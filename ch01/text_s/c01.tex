%%%%% Please note that the below listed 2 lines needs to be moved to %%%%
%%%%% a new file 'c01.tml' and the same should be compiled to get the  %%%%
%%%%% typeset pages                                                  %%%%
\def\xmlfile{c01.tml}
\input xmltex
%%%%% END  %%%%%%%%%%%%%%%%%%%%%%%%
<?xml version="1.0" encoding="utf-8"?>
<!DOCTYPE component SYSTEM "file://chgnsm02/macdata/Books/ptg/Books-Documents/Approved-Documents/Guidelines/WileyML3G/WileyML_3G_v2.0/Wileyml3gv20-flat.dtd">
<component version="2.0" xmlns:cms="http://www.wiley.com/namespaces/wiley" xmlns:wiley="http://www.wiley.com/namespaces/wiley/wiley" type="bookChapter" xml:lang="en" xml:id="w9781119237198c01">
<?xmltex \pgtag{\IIIProofVersionInfo{c01}}?>
<?xmltex \pgtag{\setcounter{chapter}{0}\setcounter{page}{3}}?>
<header xml:id="c01-hdr-0001">
<publicationMeta level="product">
<publisherInfo>
<publisherName>John Wiley &amp; Sons, Inc.</publisherName>
<publisherLoc>Hoboken, New Jersey</publisherLoc>
</publisherInfo>
<isbn type="print-13">9781119237198</isbn>
<titleGroup><title type="main" sort="QUANTITATIVE MOMENTUM">Quantitative Momentum</title></titleGroup>
<copyright ownership="publisher">Copyright &copy; 2016 by John Wiley &amp; Sons, Inc. All rights reserved.</copyright>
<numberingGroup>
<numbering type="edition" number="1">1st Edition</numbering>
</numberingGroup>
<creators><creator xml:id="cr-0001" creatorRole="author"><personName><givenNames>Wesley R.</givenNames> <familyName>Gray</familyName></personName></creator></creators>
<subjectInfo>
<subject href="http://psi.wiley.com/subject/ME20">n/a</subject>
</subjectInfo>
</publicationMeta>
<publicationMeta level="unit" position="20" type="chapter">
<idGroup>
<id type="unit" value="c01"/>
<id type="file" value="c01"/>
</idGroup>
<titleGroup><title type="name">Chapter</title></titleGroup>
<eventGroup>
<event type="xmlCreated" agent="SPi Global" date="2016-07-12"/>
</eventGroup>
<numberingGroup>
<numbering type="main">1</numbering>
</numberingGroup>
<objectNameGroup>
<objectName elementName="featureFixed">Extract</objectName>
</objectNameGroup>
</publicationMeta>
<contentMeta>
<titleGroup><title type="main">Less Religion; More Reason</title></titleGroup>
</contentMeta>
</header>
<body sectionsNumbered="no" xml:id="c01-body-0001"><section type="opening" xml:id="c01-sec-0001"><p xml:id="c01-para-0001"><?xmltex \OrgFixedPosition{c01-blkfxd-0001}?><blockFixed type="standFirst" xml:id="c01-blkfxd-0001"><p xml:id="c01-para-0002">Child: &ldquo;Dad, are you sure Santa brought the presents?&rdquo;</p>
<?xmltex \pgtag{\noindent}?><p xml:id="c01-para-0003">Father: &ldquo;Yes, Santa carried them on his sleigh.&rdquo;</p>
<?xmltex \pgtag{\noindent}?><p xml:id="c01-para-0004">Child: &ldquo;I guess that makes sense. He did eat the cookies and milk we left by the fireplace.&rdquo;</p>
<source>&mdash;Typical adult/child chat on Christmas Day</source></blockFixed></p></section>
<section xml:id="c01-sec-0002"><title type="main">Technical Analysis: The Market<?xmltex \pgtag{\kern1.1pt}?>&apos;s Oldest Religion</title><p xml:id="c01-para-0005">During the 1600s, the Dutch had a large merchant fleet and the port city of Amsterdam was a dominant commercial hub for trade from around the world. Based on the growing influence of the Dutch Republic, in 1602 the Dutch East India Company was founded, and its evolution into the first publicly traded global corporation drove a number of financial innovations to the Amsterdam Stock Exchange, including the subsequent listing of additional companies and even short selling.</p>
<p xml:id="c01-para-0006">In 1688, Joseph de la Vega, a successful Dutch merchant, wrote <i>Confusion De Confusiones</i>, one of the earliest known books to describe a stock exchange and stock trading. Some researchers today argue that he should be considered the father of behavioral finance. De la Vega vividly described excessive trading, overreaction, underreaction, and the disposition effect well before they were documented by modern finance journals.<link href="c01-note-0001"/></p>
<p xml:id="c01-para-0007">In his book, de la Vega describes the day&hyphen;to&hyphen;day business of the Exchange and alludes to how prices are set:<?xmltex \OrgFixedPosition{c01-feafxd-0001}?>
<featureFixed xml:id="c01-feafxd-0001" lwtype="Extract"><title type="featureFixedName">Extract</title><p xml:id="c01-para-0008">When a bull enters such a coffee&hyphen;house during the Exchange hours, he is asked the price of the shares by the people present. He adds one to two per cent to the price of the day and he produces a notebook in which he pretends to put down orders. The desire to buy shares increases; and this enhances also the apprehension that there may be a further rise (for on this point we are all alike: when the prices rise, we think that they fly up high and, when they have risen high, that they will run away from us).<link href="c01-note-0002"/></p>
</featureFixed></p>
<p xml:id="c01-para-0009">De la Vega seems to be describing how rising prices themselves can beget continued price increases. Put another way, in the words of Wes&apos;s graduate school roommate who managed a market making desk at a large Wall Street bank, &ldquo;High prices attract buyers, low prices attract sellers.&rdquo;<link href="c01-note-0003"/></p>
<p xml:id="c01-para-0010">De la Vega continues:<?xmltex \OrgFixedPosition{c01-feafxd-0002}?>
<featureFixed xml:id="c01-feafxd-0002" lwtype="Extract"><title type="featureFixedName">Extract</title><p xml:id="c01-para-0011">The fall of prices need not have a limit, and there are also unlimited possibilities for the rise&hellip;Therefore the excessively high values need not alarm you&hellip;there will always be buyers who will free you from anxiety&hellip;the bulls are optimistic with joy over the state of business affairs, which is steadily favorable to them; and their attitude is so full of [unthinking] confidence that even less favorable news does not impress them and causes no anxiety&hellip;[It seems] incompatible with philosophy that bears should sell after the reason for their sales has ceased to exist, since the philosophers teach that when the cause ceases, the effect ceases also. But if the bears obstinately go on selling, there is an effect even after the cause had disappeared.<link href="c01-note-0004"/></p>
</featureFixed></p>
<p xml:id="c01-para-0012">Here de la Vega explicitly discusses how bulls can continue buying, and bears can continue selling, even when there is no direct reason or cause for them to do so, other than the price action itself. So here we see how, even in seventeenth&hyphen;century Europe, price changes&mdash;independent of fundamentals&mdash;can affect future market prices.</p>
<p xml:id="c01-para-0013">While early technical analysis was evolving in stock trading in Europe, an even more fascinating financial experiment was taking place in Japan. During the 1600s, the peasant class, who made up the majority of the Japanese population, was forced into farming, thus supplying a tax base that could support the ruling military class, who, in turn, provided protection for agricultural land. Rice was the largest crop at that time, accounting for as much as 90 percent of government revenues, and became a staple of the Japanese economy.</p>
<p xml:id="c01-para-0014"><?xmltex \pgtag{\looseness=-1}?>The important role of rice in Japan led to the establishment of a formal exchange in 1697, and eventually to the emergence of what many believe to be the first futures market, the Dojima Rice Market. That market grew to include a network of warehouses, with established credit and clearing mechanisms.<link href="c01-note-0005"/></p>
<p xml:id="c01-para-0015">The rapidly evolving rice market in Japan was the fertile financial environment in which a young rice merchant, Munehisa Homma (1724&ndash;1803), found himself during the mid&hyphen;1700s. Homma began trading rice futures and used a private communications network to trade advantageously. Homma also used the history of prices to make predictions about the direction of future prices. But his key insight involved the psychology of the markets.</p>
<p xml:id="c01-para-0016">In 1755, Homma wrote, <i>The Fountain of Gold&mdash;The Three Monkey Record of Money</i>, which described the role of emotions and how these could affect rice prices. Homma observed, &ldquo;The psychological aspect of the <?xmltex \pgtag{\hbox\bgroup}?>market<?xmltex \pgtag{\egroup}?> was critical to [one&apos;s] trading success,&rdquo; and &ldquo;studying the emotions of the market&hellip;could help in predicting prices.&rdquo; Thus, Homma, like de la Vega, was perhaps one of the earliest documented practitioners of behavioral finance. His book was among the earliest writings covering markets and investor psychology.<link href="c01-note-0006"/></p>
<p xml:id="c01-para-0017">Homma invested on the long and the short side, and was thus an antecedent to today&apos;s hedge funds. He was so successful and became so wealthy that he inspired the adage: &ldquo;I will never become a Homma, but I would settle to be a local lord.&rdquo; He eventually became an adviser to the government, and to Japan&apos;s first sovereign wealth fund.<link href="c01-note-0007"/></p>
<p xml:id="c01-para-0018">On the other side of the globe, financial markets were also evolving. The late nineteenth and early twentieth centuries marked a time of increasing stock market participation in the United States. Among the most famous equity investors of that era was a man named Jesse Livermore. He began trading at the age of 14, and over his lifetime, he gained and lost several fortunes.</p>
<p xml:id="c01-para-0019"><?xmltex \pgtag{\changespaceskip{2.3}}?>An American author named Edwin Lefevre wrote the biography <i>Reminiscences of a Stock Operator</i>. The biography is an account of Livermore&apos;s life and experiences in the early years of 1900s. The book describes <?xmltex \pgtag{\bgroup\mbox}?>Livermore&apos;s<?xmltex \pgtag{\egroup}?> success using technical trading rules. Lefevre also described Livermore&apos;s overarching philosophy on the market:<?xmltex \OrgFixedPosition{c01-feafxd-0003}?>
<featureFixed xml:id="c01-feafxd-0003" lwtype="Extract"><title type="featureFixedName">Extract</title><p xml:id="c01-para-0020">You watch the market&hellip;with one object: to determine the direction&mdash;that is the price tendency&hellip;Nobody should be puzzled as to whether a market is a bull or a bear market after it fairly starts. The trend is evident to a man who has an open mind and reasonably clear sight&hellip;<link href="c01-note-0008"/></p>
</featureFixed></p>
<p xml:id="c01-para-0021">We gain more insight into Livermore&apos;s investment philosophy when we examine comments regarding his buy and sell decisions. We would recognize these decisions today as modern &ldquo;momentum&rdquo; strategies: &ldquo;It is surprising how many experienced traders there are who look incredulous when<?xmltex \pgtag{\nb}?> I<?xmltex \pgtag{\nb}?> tell<?xmltex \pgtag{\nb}?> them that when I buy stocks for a rise I like to pay top prices and when I sell I must sell low or not at all.&rdquo;</p>
<p xml:id="c01-para-0022">Clearly, the ideas that investors are not completely rational, and prices are related to future prices are not new ideas. Collectively, the investors discussed above&mdash;Joseph de la Vega, Munehisa Homma, and Jesse Livermore&mdash;highlight how great investors across history have recognized the role of psychology in the markets, and that historical prices can help predict future prices&mdash;in other words, technical analysis works. But fast forward to the early twentieth century, when some investors began to question whether technical analysis represented a sensible approach to investing. Many thought analysis of a company&apos;s fundamentals might be a more reasonable technique. Investors began to investigate fundamental analysis, involving a careful review of a company&apos;s financial statements, in hopes that such analysis might provide a better rationale for making investment decisions. In particular, a new investing philosophy began to gain notoriety: value investing, which involves buying stocks trading at a low price versus various fundamentals, such as earnings or cash<?xmltex \pgtag{\nobreak}?> <?xmltex \pgtag{\hbox\bgroup}?>flow.<?xmltex \pgtag{\egroup}?><?xmltex \pgtag{\vspace*{-6pt}}?></p></section>
<section xml:id="c01-sec-0003"><title type="main">A New Religion Emerges: Fundamental Analysis</title><p xml:id="c01-para-0023">Benjamin Graham is commonly known as the father of the value investing movement. Graham believed that if investors bought stocks at prices consistently below their intrinsic value, as determined by fundamental analysis, those investors could earn superior risk&hyphen;adjusted returns. Graham outlined his value&hyphen;investing framework in two of the most famous investing books of all time, <i>Security Analysis</i> and <i>The Intelligent Investor</i>.</p>
<p xml:id="c01-para-0024"><?xmltex \pgtag{\changespaceskip{2.4}}?>Graham realized that there were many adherents to the technical analysis approach, but he was clear in expressing what he thought of the discipline: bogus witchcraft. A quote from <i>The Intelligent Investor</i> summarizes his views:<?xmltex \OrgFixedPosition{c01-feafxd-0004}?>
<featureFixed xml:id="c01-feafxd-0004" lwtype="Extract"><title type="featureFixedName">Extract</title><p xml:id="c01-para-0025">The one principle that applies to nearly all these so&hyphen;called &ldquo;technical approaches&rdquo; is that one should buy because a stock or the market has gone up and one should sell because it has declined. This is the exact opposite of sound business sense everywhere else, and it is most unlikely that it can lead to lasting success on Wall Street.<link href="c01-note-0009"/></p>
</featureFixed></p>
<p xml:id="c01-para-0026">Graham&apos;s early criticism of technical analysis has been reinforced over time by other adamant adherents of the fundamental analysis religion. <?xmltex \pgtag{\bgroup\mbox}?>Graham&apos;s<?xmltex \pgtag{\egroup}?> most famous prot&eacute;g&eacute;, Warren Buffett, took the boxing gloves from Graham and continued to beat on the technical analysis crowd. A statement attributed to him demonstrates his views: &ldquo;I realized technical analysis didn&apos;t work when I turned the charts upside down and didn&apos;t get a different answer.&rdquo; A more recent quote by Burt Malkiel, who penned the popular book <i>A Random Walk Down Wall Street</i>, brings the disdain for technical methods front and center: &ldquo;The central proposition of charting is absolutely false&hellip;&rdquo;<link href="c01-note-0010"/></p>
<p xml:id="c01-para-0027">One can almost hear the laughter from the fundamental analysts. They believe they are better informed and ultimately more rational than technical investors. Another statement attributed to Buffett is, &ldquo;If past history was all there was to the game, the richest people would be librarians.&rdquo; It&apos;s pretty obvious that, in Buffett&apos;s view, only obscure and harebrained librarians turning their charts around and around would ever consider technical analysis to be a legitimate discipline. And perhaps the religious adherents of the fundamental approach thought that the use of humor and ridicule would make their arguments more compelling.</p>
<p xml:id="c01-para-0028">More recently, Seth Klarman, the billionaire founder of the Baupost Group hedge fund, has also denigrated technical analysis. In his cult&hyphen;classic value investing book <i>Margin of Safety: Risk&hyphen;Averse Value Investing Strategies for the Thoughtful Investor</i>, Klarman is clear about his views:<link href="c01-note-0011"/><?xmltex \OrgFixedPosition{c01-feafxd-0005}?>
<featureFixed xml:id="c01-feafxd-0005" lwtype="Extract"><title type="featureFixedName">Extract</title><p xml:id="c01-para-0029">Speculators&hellip;buy and sell securities based on the whether they believe those securities will next rise or fall in price. Their judgment regarding future price movements is based, not on fundamentals, but on a prediction of the behavior of others&hellip;They buy securities because they &ldquo;act&rdquo; well and sell when they don&apos;t&hellip;Many speculators attempt to predict the market direction by using technical analysis&mdash;past stock price fluctuations&mdash;as a guide. Technical analysis is based on the presumption that past share prices meanderings, rather than underlying business value, hold the key to future stock prices. In reality, no one knows what the market will do; trying to predict it is a waste of time, and investing based on that prediction is a speculative undertaking&hellip;speculators&hellip;are likely to lose money over<?xmltex \pgtag{\nobreak}?> <?xmltex \pgtag{\hbox\bgroup}?>time.<?xmltex \pgtag{\egroup}?></p>
</featureFixed></p>
<p xml:id="c01-para-0030">It is illuminating that Klarman views underlying fundamentals as the only justifiable signal for insight into future stock prices. Price action is<?xmltex \pgtag{\break}?> &ldquo;meandering&rdquo; and meaningless, and efforts to predict the behavior of <?xmltex \pgtag{\bgroup\mbox}?>others<?xmltex \pgtag{\egroup}?> are in vain. But Klarman doesn&apos;t stop here. He goes on to reject <i>any</i> systematic means of predicting future stock prices:<?xmltex \OrgFixedPosition{c01-feafxd-0006}?>
<featureFixed xml:id="c01-feafxd-0006" lwtype="Extract"><title type="featureFixedName">Extract</title><p xml:id="c01-para-0031">Some investment formulas involve technical analysis, in which past stock&hyphen;price movements are considered predictive of future prices. Other formulas incorporate investment fundamentals such as price&hyphen;to&hyphen;earnings (P/E) ratios, price&hyphen;to&hyphen;book&hyphen;value ratios, sales or profits growth rates, dividend yields, and the prevailing level of interest rates. Despite the enormous effort that has been put into devising such formulas, none has been proven to<?xmltex \pgtag{\nobreak}?> <?xmltex \pgtag{\hbox\bgroup}?>work.<?xmltex \pgtag{\egroup}?></p>
</featureFixed></p>
<p xml:id="c01-para-0032">It is perhaps surprising that Graham, Malkiel, Buffett, and Klarman would be so dismissive of technical analysis, given what seems to be a rich vein of successful historical practitioners and a stack of academic research that is arguably higher than the research that supports the merits of a fundamental, or value investing, approach. Nevertheless, these fundamental investors&apos; views are reflective of those of many in the value investing community and of fundamental practitioners in general. The value investing religion is alive and<?xmltex \pgtag{\nobreak}?> <?xmltex \pgtag{\hbox\bgroup}?>well.<?xmltex \pgtag{\egroup}?></p></section>
<section xml:id="c01-sec-0004"><title type="main">The Age of<?xmltex \pgtag{\protect\nobreak}?> Evidence&hyphen;Based Investing</title><p xml:id="c01-para-0033"><?xmltex \OrgFixedPosition{c01-blkfxd-0002}?><?xmltex \pgtag{\Secfollowedepitrue}?><blockFixed type="standFirst" xml:id="c01-blkfxd-0002"><p xml:id="c01-para-0034">&ldquo;Avoid extremely intense ideology because it ruins your mind.&rdquo;</p>
<source>&mdash;Charlie Munger, Vice Chairman, Berkshire Hathaway<link href="c01-note-0012"/></source></blockFixed></p>
<p xml:id="c01-para-0035">Why did Ben Graham, a data&hyphen;driven financial economist at heart, have a knee&hyphen;jerk distrust for technical methods? Perhaps some of this doubt relates to how technical analysis differs from fundamental analysis. For value investors, fundamentals lead, and prices follow, albeit noisily. However, for technical investors, prices lead, and perhaps even drive fundamentals, but fundamentals are not the core driver of stock movements. Moreover, the <i>technician</i> label captures a larger group of the investing public, with a much larger distribution of skills, ranging from the peon to the preeminent. This wider distribution means the average technician tends to be more subjective, less professional, and generally less sophisticated than the average fundamental investor. Thus, one criticism of technical analysis might be that investors are seeking out patterns where no patterns really exist&mdash;a reasonable concern, given what we know about<?xmltex \pgtag{\break}?> human behavior.</p>
<p xml:id="c01-para-0036">Contrast the technical analyst with the fundamental analyst. The fundamental analyst is looking at concrete data&mdash;financial statements&mdash;that are based on established conventions. For example, positive net income ratios, ample free cash flow, and low levels of debt can be considered fairly objective measures of good financial health. Additionally, the fundamental analyst must do a lot of hard work to conduct her security analysis: after all, she is trying to identify the present value of all future cash flows from a business and discount them to the present<?xmltex \pgtag{\nobreak}?> <?xmltex \pgtag{\hbox\bgroup}?>time.<?xmltex \pgtag{\egroup}?></p>
<p xml:id="c01-para-0037">The fundamental analyst is thus arguably engaged in a more thoughtful and intellectually rigorous pursuit. In this sense, she is perhaps more credible. Buying based on fundamentals seems more reasonable than examining recent price charts with a Ouija board. The technical analyst is assumed to have a simpler job because one can reasonably argue that a <?xmltex \pgtag{\bgroup\mbox}?>history<?xmltex \pgtag{\egroup}?> of prices<?xmltex \pgtag{\vadjust{\vfill\eject}}?> is a limited and simplistic signal, whereas for the <?xmltex \pgtag{\bgroup\mbox}?>fundamental<?xmltex \pgtag{\egroup}?> analyst, there is a much wider and deeper array of financial information to digest and consider.</p>
<p xml:id="c01-para-0038">But in the end, does effort and sophistication really matter? Taking a step back, the mission for long&hyphen;term active investors is to beat the market. Active investors should focus on the scientific method to address a basic question: What works? Warren Buffett obviously showed that value investing, irrespective of technical considerations, can work. But Stanley Druckenmiller, George Soros, and Paul Tudor Jones also showed that technical analysis can work just as well. An ever&hyphen;growing body of academic research formalizes the evidence that fundamental strategies (e.g., value and quality) and technical strategies (e.g., momentum and trend&hyphen;following) both seem to work.<link href="c01-note-0013"/> Many dogmatic investors, however, looking to confirm what they already believe, selectively adopt the research evidence that fits their investing religion. In contrast, an evidence&hyphen;based investor will conclude that fundamental and technical analysis strategies can work because they are two sides of the same coin. They are cousins&mdash;because they share the common objective of exploiting the poor decisions of market participants influenced by biased decision making. As Andrew Lo, an influential and forward&hyphen;looking financial economist at MIT, correctly observes about the debate between fundamental and technical traders, &ldquo;In the end we all have the same goal, which is to forecast uncertain market prices. We should be able to learn from each other.&rdquo;</p><?xmltex \pgtag{\vspace*{-5pt}}?><section xml:id="c01-sec-0005"><title type="main">We Agree: Less Religion, More Reason</title><p xml:id="c01-para-0039">The debate outlined above is merely the tip of the analysis iceberg and is meant to demonstrate the contentious debates that surround different investment philosophies. And as people become devoted to a particular philosophy, their beliefs often become more firmly established. Thus, while ascertaining the winner in these debates is impossible, one thing is certain: Once an investment strategy has gained a convert, it is nearly impossible to &ldquo;flip&rdquo; that convert to another investment religion. But why do these debates necessarily need to be so contentious? Why should value and momentum approaches be mutually exclusive? Indeed, a key aspect of the scientific method is to preserve the freedom to doubt, for without doubt we would cease to explore new ideas. We argue in Chapter <exlink href="urn:x-wiley:9781119237198:xml-component:w9781119237198c02"/> that there is an overarching framework for understanding why certain strategies work. We call our framework the <i>sustainable active investing framework</i>. This framework does not seek to identify the best investment strategy, but aims to identify the necessary conditions for any investment strategy to succeed in the future.</p></section>
</section>
<?xmltex \pgtag{\vfill\eject}?>
<section xml:id="c01-sec-0006"><title type="main">Don&apos;t Worry: This Book Is About<?xmltex \pgtag{\protect\break}?> Stock&hyphen;Selection Momentum</title><p xml:id="c01-para-0040">In this introductory chapter, we&apos;ve already discussed technical analysis, fundamental analysis, and psychology. A lot of topics in short order and no mention of how to build a momentum strategy&mdash;and we will continue to explore these important topics in the next few chapters. But we want to be clear: this book <i>is</i> about stock&hyphen;selection momentum. But in order to really understand how to build <i>any</i> active investing strategy, we need context to understand how and why this strategy will presumably work in the future. This discussion will be covered in Chapters <exlink href="urn:x-wiley:9781119237198:xml-component:w9781119237198c02"/> through <exlink href="urn:x-wiley:9781119237198:xml-component:w9781119237198c04"/>. If you are an advanced practitioner, we recommend you skip ahead to Chapter <exlink href="urn:x-wiley:9781119237198:xml-component:w9781119237198c05"/> for the cookbook details on how to create what we consider to be an effective active momentum strategy; however, if you want to understand and be successful with the momentum strategy proposed, you will want to read the chapters in the order we present them. Also, we must emphasize that the strategy we outline is <i>not for everyone</i>, primarily because it requires discipline to follow, but more explicitly because the math doesn&apos;t add up. From an equilibrium perspective, not everyone can follow our strategy because for every stock we buy, there is a seller on the other side of the trade.</p>
<p xml:id="c01-para-0041">With that disclaimer out of the way, let&apos;s outline what we mean by stock&hyphen;selection momentum. There is sometimes confusion associated with so&hyphen;called <i>momentum</i> strategies&mdash;we want to clear the muddy waters. We break momentum into two categories to differentiate between the different approaches to measure momentum:
<?xmltex \pgtag{\def\itemwd{2.}}?>
<list xml:id="c01-list-0001" style="1"><listItem xml:id="c01-li-0001"><b>Time&hyphen;series momentum:</b> Sometimes referred to as <i>absolute momentum,</i> time&hyphen;series momentum is calculated based on a stock&apos;s <i>own past return</i>, considered independently from the returns of other stocks.<link href="c01-note-0014"/></listItem>
<listItem xml:id="c01-li-0002"><b>Cross&hyphen;sectional momentum:</b> Originally referred to as <i>relative strength,</i> before academics developed a more jargon&hyphen;like term, cross&hyphen;sectional momentum is a measure of a stock&apos;s performance, <i>relative</i> to other stocks.<link href="c01-note-0015"/></listItem>
</list>
</p>
<p xml:id="c01-para-0042">A simple example will illustrate the difference. Consider a hypothetical scenario where we have two stocks in our universe: Apple and Google. Twelve months ago, Apple was &dollar;25 per share and Google was also &dollar;25 per share. Today, Apple is &dollar;100 per share and Google is &dollar;50 per share.</p>
<p xml:id="c01-para-0043">Next, we examine a simple time&hyphen;series momentum rule and a simple cross&hyphen;sectional momentum<?xmltex \pgtag{\nobreak}?> <?xmltex \pgtag{\hbox\bgroup}?>rule.<?xmltex \pgtag{\egroup}?></p>
<p xml:id="c01-para-0044">The time&hyphen;series rule will buy a stock that has positive performance over the past 12 months, and will sell a stock if the stock has negative <?xmltex \pgtag{\bgroup\mbox}?>performance<?xmltex \pgtag{\egroup}?>. Here is how our time&hyphen;series momentum&hyphen;trading rule would treat this scenario:
<list xml:id="c01-list-0002" style="bulleted"><listItem xml:id="c01-li-0003"><b>Time&hyphen;series momentum:</b> <i>Long</i> Apple and <i>long</i> Google because both stocks have strong absolute momentum.</listItem>
</list>
</p>
<p xml:id="c01-para-0045">Our cross&hyphen;sectional rule will buy a stock if the stock&apos;s past performance over the past 12 months is <i>relatively stronger</i> than the past performance of other stocks in the universe (and will sell a stock if it has poor relative performance to other stocks). Here is how our cross&hyphen;sectional momentum&hyphen;trading rule would treat this scenario:
<list xml:id="c01-list-0003" style="bulleted"><listItem xml:id="c01-li-0004"><b>Cross&hyphen;sectional momentum:</b> <i>Long</i> Apple and <i>short</i> Google because Apple is relatively stronger performing than Google.</listItem>
</list>
</p>
<p xml:id="c01-para-0046">Note that even though both stocks have increased in price (we are long both from a time&hyphen;series momentum perspective), Apple&apos;s price has gone up <i>much more</i> than Google&apos;s price; thus, Apple has stronger momentum in the cross&hyphen;section (suggesting long Apple and short Google from a cross&hyphen;sectional momentum perspective).</p>
<p xml:id="c01-para-0047">One could use elements of both types of momentum to develop a momentum strategy. For example, we could consider both momentum elements and invest based on both the time series rule <i>and</i> the cross&hyphen;sectional rules. Using our example above, we would go long Apple, because the time&hyphen;series rule says buy and the cross&hyphen;sectional rule also says buy, but we might take no position in Google because one of the rules (i.e., cross&hyphen;sectional momentum) says to sell.<link href="c01-note-0016"/></p>
<p xml:id="c01-para-0048">As outlined above, the various forms of momentum can be used to develop a stock selection methodology. We want to highlight that time&hyphen;series and cross&hyphen;sectional momentum are often used in a market&hyphen;timing or asset&hyphen;class selection context. Let us be clear: This book is not focused on market&hyphen;timing or asset class selection&mdash;we are trying to understand how different elements of momentum might be useful in the context of <i>individual stock selection</i>. This book is a stock picking book, not an asset allocation<?xmltex \pgtag{\nobreak}?> <?xmltex \pgtag{\hbox\bgroup}?>book.<?xmltex \pgtag{\egroup}?></p></section>
<section type="summary" xml:id="c01-sec-0007"><title type="main">Summary</title><p xml:id="c01-para-0049">In this chapter, we outline the long&hyphen;running debate between technical and fundamental investors. Many readers are certainly familiar with both faiths, and there are certainly zealots to be found in each camp. In many circumstances the debate between technical and fundamental investing tactics isn&apos;t a debate&mdash;it is a yelling match. We want to stop the yelling and start the research. To circumvent the yelling match, in the next chapter we will describe the sustainable active investing framework. This framework will help us better understand why certain strategies work and why others do not, independent of the dogma. Through this lens we can form testable hypotheses and have a constructive discussion. Our framework is decidedly not perfect, but we do our best to contextualize the debate. Because, let&apos;s be honest, the mission of active investing is not to argue about which investment philosophy is better&mdash;who cares&mdash;we just want to beat the market over the long term! Also, to reiterate, if you are an advanced practitioner looking to learn about the details of our proposed stock&hyphen;selection momentum strategy, feel free to skip to Chapter <exlink href="urn:x-wiley:9781119237198:xml-component:w9781119237198c05"/>.</p></section>
<?xmltex \pgtag{\tablenotecnt=6\def\itemwd{16.}}?><noteGroup xml:id="c01-ntgp-0001"><title type="main">Notes</title><note xml:id="c01-note-0001">Teresa Corzo, Margarita Prat, and Esther Vaquero, &ldquo;Behavioral Finance In Joseph de la Vega&apos;s Confusion de Confusiones,&rdquo; <i>The Journal of Behavioral Finance</i> 15 (2014): 341&ndash;350.</note>
<note xml:id="c01-note-0002">Joseph de la Vega, <i>Confusion de Confusiones</i>. An English translation of <i>Confusion de Confusiones,</i> 1688, is available via babel.hathitrust.org/cgi/pt?id&equals;uc1<?xmltex \pgtag{\break}?>.32106019504239, accessed 2/15/2015.</note>
<note xml:id="c01-note-0003">Attributed to Jared Hullick.</note>
<note xml:id="c01-note-0004">De la Vega.</note>
<note xml:id="c01-note-0005">www.ndl.go.jp/scenery/kansai/e/column/markets&uscore;in&uscore;osaka.html, accessed<?xmltex \pgtag{\hb}?> February 15, 2015.</note>
<note xml:id="c01-note-0006">Jasmina Hasanhodzic, &ldquo;Technical Analysis: Neural network based pattern recognition of technical trading indicators, statistical evaluation of their predictive value and a historical overview of the field,&rdquo; MIT Master&apos;s Thesis (1979). Accessible at hdl.handle.net/1721.1/28725.</note>
<note xml:id="c01-note-0007">Steve Nison, <i>Japanese Candlestick Charting Techniques</i> (New York: Prentice Hall Press, 2001).</note>
<note xml:id="c01-note-0008">Edwin Lefevre and Roger Lowenstein, <i>Reminiscences of a Stock Operator</i> (Hoboken, NJ: John Wiley &amp; Sons, 2006).</note>
<note xml:id="c01-note-0009">Benjamin Graham, <i>The Intelligent Investor</i> (New York: Harper, 1949).</note>
<note xml:id="c01-note-0010">Burt Malkiel, <i>A Random Walk Down Wall Street</i> (New York: W. W. Norton &amp; Company, 1996).</note>
<note xml:id="c01-note-0011">Seth Klarman, <i>Margin of Safety</i> (New York: Harper Collins, 1991).</note>
<note xml:id="c01-note-0012">Charlie Munger USC Law Commencement Speech, May 2007. www.youtube<?xmltex \pgtag{\break}?>.com/watch?v&equals;NkLHxMWAZgQ, accessed February 28, 2016.</note>
<note xml:id="c01-note-0013">See Wesley Gray and Tobias Carlisle, <i>Quantitative Value: A Practitioner&apos;s Guide to Automating Intelligent Investment and Eliminating Behavioral Errors</i> (Hoboken, NJ: John Wiley &amp; Sons, 2012), and Chris Geczy and Mikhail Samonov, &ldquo;Two Centuries of Price Return Momentum,&rdquo; <i>Financial Analysts Journal</i> (2016).</note>
<note xml:id="c01-note-0014">See Gary Antonacci, <i>Dual Momentum Investing: An Innovative Strategy for Higher Returns with Lower Risk</i> (New York: McGraw&hyphen;Hill, 2014), and Tobias Moskowitz, Yao Ooi, and Lasse Pedersen, &ldquo;Time Series Momentum,&rdquo; <i>Journal of Financial Economics</i> 104 (2012): 228&ndash;250.</note>
<note xml:id="c01-note-0015">See Andreas Clenow, &ldquo;Stocks on the Move: Beating the Market with Hedge Fund Momentum Strategies,&rdquo; self&hyphen;published, 2015, for a practitioner perspective, and see Narasimhan Jegadeesh and Sheridan Titman, &ldquo;Returns to Buying Winners and Selling Losers: Implications for Stock Market Efficiency,&rdquo; <i>The Journal of Finance</i> 48 (1993): 65&ndash;91, for an academic discussion.</note>
<note xml:id="c01-note-0016">See Antonacci&apos;s <i>Dual Momentum</i> book for a discussion of dual momentum in an asset allocation context, which is different than our context of individual stock selection. It conveys the idea of using both types of momentum in an investment system.</note></noteGroup>
</body>
</component>
