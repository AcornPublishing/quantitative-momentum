%%%%% Please note that the below listed 2 lines needs to be moved to %%%%
%%%%% a new file 'c02.tml' and the same should be compiled to get the  %%%%
%%%%% typeset pages                                                  %%%%
\def\xmlfile{c02.tml}
\input xmltex
%%%%% END  %%%%%%%%%%%%%%%%%%%%%%%%
<?xml version="1.0" encoding="utf-8"?>
<!DOCTYPE component SYSTEM "file://chgnsm02/macdata/Books/ptg/Books-Documents/Approved-Documents/Guidelines/WileyML3G/WileyML_3G_v2.0/Wileyml3gv20-flat.dtd">
<component version="2.0" xmlns:cms="http://www.wiley.com/namespaces/wiley" xmlns:wiley="http://www.wiley.com/namespaces/wiley/wiley" type="bookChapter" xml:lang="en" xml:id="w9781119237198c02">
<?xmltex \pgtag{\IIIProofVersionInfo{c02}}?>
<?xmltex \pgtag{\setcounter{chapter}{1}\setcounter{page}{14}}?>
<?xmltex \pgtag{\def\Gpath{u:/books/Wiley/Pd/E-line/Reprint/Gray37198/figures/iround}%
}?>
<header xml:id="c02-hdr-0001">
<publicationMeta level="product">
<publisherInfo>
<publisherName>John Wiley &amp; Sons, Inc.</publisherName>
<publisherLoc>Hoboken, New Jersey</publisherLoc>
</publisherInfo>
<isbn type="print-13">9781119237198</isbn>
<titleGroup><title type="main" sort="QUANTITATIVE MOMENTUM">Quantitative Momentum</title></titleGroup>
<copyright ownership="publisher">Copyright &copy; 2016 by John Wiley &amp; Sons, Inc. All rights reserved.</copyright>
<numberingGroup>
<numbering type="edition" number="1">1st Edition</numbering>
</numberingGroup>
<creators><creator xml:id="cr-0001" creatorRole="author"><personName><givenNames>Wesley R.</givenNames> <familyName>Gray</familyName></personName></creator></creators>
<subjectInfo>
<subject href="http://psi.wiley.com/subject/ME20">n/a</subject>
</subjectInfo>
</publicationMeta>
<publicationMeta level="unit" position="30" type="chapter">
<idGroup>
<id type="unit" value="c02"/>
<id type="file" value="c02"/>
</idGroup>
<titleGroup><title type="name">Chapter</title></titleGroup>
<eventGroup>
<event type="xmlCreated" agent="SPi Global" date="2016-07-13"/>
</eventGroup>
<numberingGroup>
<numbering type="main">2</numbering>
</numberingGroup>
<objectNameGroup>
<objectName elementName="featureFixed">Extract</objectName>
</objectNameGroup>
</publicationMeta>
<contentMeta>
<titleGroup><title type="main">Why Can Active Investment Strategies Work?</title></titleGroup>
</contentMeta>
</header>
<body sectionsNumbered="no" xml:id="c02-body-0001"><section type="opening" xml:id="c02-sec-0001"><p xml:id="c02-para-0001"><?xmltex \OrgFixedPosition{c02-blkfxd-0001}?>
<blockFixed type="standFirst" xml:id="c02-blkfxd-0001"><p xml:id="c02-para-0002">&ldquo;The worst thing I can be is the same as everybody else.&rdquo;</p>
<source>&mdash;Attributed to Arnold Schwarzenegger</source>
</blockFixed></p>
<p xml:id="c02-para-0003"><?xmltex \pgtag{\firstlet}?>The debate over active investing versus passive investing is akin to other classic conflicts, such as Philadelphia Eagles versus Dallas Cowboys or Coke versus Pepsi. In short, once our preference for one style over the other is established, it often becomes a proven fact or incontrovertible reality in our minds. Psychology research describes the notion of &ldquo;confirmation bias,&rdquo; in which people prefer evidence that supports their earlier conclusions, and ignore disconfirming evidence.</p>
<p xml:id="c02-para-0004">The following discussion is not meant to convert a passive investor into an active investor; however, we do explain why we believe <i>some</i> active investing approaches, given certain characteristics, might logically beat other investment strategies over a reasonably long time horizon. In other words, what drove the success of Munehisa Homma, Jesse Livermore, and Ben Graham, when all three active investors had dramatically different investment philosophies? Perhaps it is all just luck, but we believe there might have been something<?xmltex \pgtag{\nobreak}?> <?xmltex \pgtag{\hbox\bgroup}?>more.<?xmltex \pgtag{\egroup}?></p>
<?xmltex \pgtag{\enlargethispage{1pc}}?>
<p xml:id="c02-para-0005">A key theme that seems to underlie all of their approaches is the exploitation of irrational investor behaviors. But if understanding behavior were the Holy Grail, why aren&apos;t psychologists running the capital markets? Or perhaps Homma, Livermore, and Graham were just smarter than everyone else? Being smarter does not seem to be the correct answer either, since investors with the highest IQs do not control the market. Perhaps the most famous case is that of Sir Isaac Newton&mdash;the genius who developed modern physics. The great physicist and mathematician famously went broke trading the stock of the South Sea Company in the early<?xmltex \pgtag{\break}?> eighteenth<?xmltex \pgtag{\nb}?> century.</p>
<p xml:id="c02-para-0006"><?xmltex \pgtag{\looseness=-1}?>Thus far there does not seem to be a &ldquo;silver bullet&rdquo; explanation to describe how active investors beat the market. Being smart, understanding behavioral bias, or amassing an army of PhDs to crunch data is only half the battle. Even with those tools, an active investor is still only one shark in a tank filled with other sharks. All sharks are smart and all sharks know how to analyze a company and how to read and understand financial charts. Maintaining an edge in these shark&hyphen;infested waters is no small feat, and one that only a handful of investors have consistently accomplished. So what&apos;s the answer? We still aren&apos;t sure, and we are always learning. Our best working theory is that there are two components that drive sustainable success for active investors:
<list xml:id="c02-list-0001" style="bulleted"><listItem xml:id="c02-li-0001">A keen understanding of human psychology, and</listItem>
<listItem xml:id="c02-li-0002">A thorough grasp of &ldquo;smart money&rdquo; incentives.</listItem>
</list><?xmltex \pgtag{\vspace*{-12pt}}?></p></section>
<section xml:id="c02-sec-0002"><title type="main">Into the<?xmltex \pgtag{\protect\nobreak}?> Lion&apos;s Den</title><p xml:id="c02-para-0007">Wes entered the University of Chicago Finance PhD program in 2002. It was the beginning of a painful, but highly enlightening journey into the world of advanced finance. For context, the University of Chicago finance department maintains a rich legacy associated with having established, and successfully defended, the Efficient Market Hypothesis (EMH). PhD students in the department spend their first two years in grueling, graduate&hyphen;level finance courses infused with highly technical mathematics and statistics. The final two to four years are dedicated to dissertation research. The best way to describe the scene is as follows: sweatshop factory meets international mathematics competition. In short, the program is tough.</p>
<?xmltex \pgtag{\enlargethispage{1pc}}?>
<p xml:id="c02-para-0008">After surviving his first two years of intellectual waterboarding, Wes needed a break. He took a unique &ldquo;sabbatical,&rdquo; and decided to join the United States Marine Corps for four years. To make a long story short: He wanted to serve, and he wasn&apos;t getting any younger. Wes returned to the PhD program in 2008 to finish his dissertation. His time in the Marines taught him a lot of things, but one lesson stood out from the rest: &ldquo;<b>Make Bold Moves.</b>&rdquo;<link href="c02-note-0001"/> And of course, what is the boldest move one can do at the University of Chicago?</p>
<p xml:id="c02-para-0009"><i>Focus on research that questions the efficient market hypothesis.</i><?xmltex \pgtag{\vspace*{-6pt}}?></p><section xml:id="c02-sec-0003"><title type="main">Inefficient Market Mavericks: Value Investors</title><p xml:id="c02-para-0010">Wes wanted to determine if fundamental investors, or &ldquo;value&rdquo; investors, could beat the market. He had been religiously following a value investing strategy with his own account for over 10 years. He was a tried&hyphen;and&hyphen;true believer in the Ben Graham fundamentals&hyphen;focused value investing religion (he still considered technical trading ideas to be heresy). The story that active value investing could beat the market was compelling, but much of the rhetoric in academic circles, and the research published in top&hyphen;tier academic journals, suggested otherwise.</p>
<p xml:id="c02-para-0011"><?xmltex \pgtag{\looseness=-1}?>The <i>value</i> debate was reinvigorated by a highly cited Eugene Fama and Ken French paper titled &ldquo;The Cross&hyphen;Section of Expected Stock Returns.&rdquo;<link href="c02-note-0002"/> The paper sparked a debate over whether or not the so&hyphen;called <i>value premium,</i> or the large spread in historical returns between cheap stocks and expensive stocks, was due to extra risk or to mispricing. Were the excess returns of value stocks a reward for added economic risk factors borne by shareholders, or were these stocks simply mispriced? For Eugene Fama and Ken French, the answer was clear: The value premium must be attributed to higher risk if the market was efficient. The risk&hyphen;based argument for the value premium seemed far&hyphen;fetched to Wes, who was a Ben Graham aficionado. Graham and his disciple Warren Buffett were famous for beating the market over long periods of time by buying cheap stocks. Their claim was that &ldquo;Mr. Market,&rdquo; who represented the broad market, was characterized as a manic&hyphen;depressive person with deep psychological problems: Mr. Market would sometimes offer stocks for prices below their fundamental value (e.g., the trough of the 2008 financial crisis) or above their fundamental value (e.g., during the Internet bubble of the late 1990s). And if a value investor purchased cheap, eventually Mr. Market would agree. But could it be the case that the stocks these value investors bought had high returns, not because they outsmarted Mr. Market, but because they were buying more risk and got lucky? Wes began digging.</p>
<p xml:id="c02-para-0012">Wes started collecting data on nearly 4,000 investment picks submitted by top fund experts, asset managers, and value enthusiasts to Joel Greenblatt&apos;s website, ValueInvestorsClub.com. This club wasn&apos;t just any club. This club was highly selective, with members screened for quality, and was regarded as one of the best sites on the web for market ideas. Members tended to be heavy hitters in the value investing arena.</p>
<p xml:id="c02-para-0013">After a year of toil and anguish, Wes compiled all the members&apos; stock recommendations into a database so he could conduct a thorough analysis. The results were extremely compelling&mdash;there was strong evidence that these &ldquo;varsity value investors&rdquo; exhibited significant stock&hyphen;picking skills.</p>
<p xml:id="c02-para-0014">Excited to share his new findings, Wes eagerly drafted a paper, which included the following sentence at the end of the abstract:<?xmltex \OrgFixedPosition{c02-feafxd-0001}?>
<featureFixed xml:id="c02-feafxd-0001" lwtype="Extract"><title type="featureFixedName">Extract</title><p xml:id="c02-para-0015"><?xmltex \pgtag{\changespaceskip{2.1}}?>Analyzing buy&hyphen;and&hyphen;hold abnormal returns and calendar&hyphen;time portfolio regressions, I conclude that value investors have stock&hyphen; picking skills.</p>
</featureFixed></p>
<?xmltex \pgtag{\vfill\eject}?>
<p xml:id="c02-para-0016">Pleased with his work, Wes sent his draft dissertation to his adviser, Dr. Eugene Fama, who by then was widely recognized as the &ldquo;father of modern finance,&rdquo; and was closely identified with the efficient market hypothesis (&ldquo;EMH&rdquo;). Dr. Fama would go on to win the 2013 Nobel Prize in Economics. Dr. Fama was a strong&mdash;perhaps the strongest&mdash;supporter of EMH. Because Dr. Fama reviewed the results of Wes&apos;s research personally, Wes&apos;s draft was sure to be rigorously scrutinized. The response Wes received was less than ideal:<?xmltex \OrgFixedPosition{c02-feafxd-0002}?><featureFixed xml:id="c02-feafxd-0002" lwtype="Extract"><title type="featureFixedName">Extract</title><p xml:id="c02-para-0017">&ldquo;Your conclusion has to be false&hellip;&rdquo;</p></featureFixed></p>
<p xml:id="c02-para-0018">Wes sped down to Dr. Fama&apos;s office to get some clarification. The last thing Wes wanted was a year&apos;s worth of blood, sweat, and tears to get tossed out the window. Wes&apos;s evidence seemed solid. Was Dr. Fama simply being dogmatic? Wes had to know exactly why Dr. Fama disagreed. Sweating profusely, with the prospect of the PhD degree slowly slipping away, he asked one of the world&apos;s most famous financial economists for clarification. Fama responded that the data and analysis were sound, but that Wes simply couldn&apos;t say that value investors had stock&hyphen;picking skills. Always a stickler for detail, Dr. Fama insisted that Wes qualify the abstract by adding two clarifying words to the concluding statement from the paper: &ldquo;<i>The sample.</i>&rdquo; So instead of saying that &ldquo;value investors have stock picking skills&rdquo; the final sentence needed to say that &ldquo;<i>the sample</i> of value investors have stock&hyphen;picking skills.&rdquo;<link href="c02-note-0003"/></p>
<p xml:id="c02-para-0019">Wes sat back, <i>relieved,</i> and relearned what he had been taught by his mother as a young child: words matter. The eminent Fama was, not surprisingly, correct: Wes&apos;s findings did not suggest that <i>all</i> value investors have skill, merely that the sample he was investigating had skill&mdash;a subtle, yet important distinction. Crisis averted.</p>
<p xml:id="c02-para-0020">Wes graduated the following year, with his research affirming, at least for him, if not for Dr. Fama, that markets were not perfectly efficient and value investors had an edge. Soon thereafter, Wes took a job as a finance professor at Drexel University and met Jack Vogel, who was a finance PhD student at the time. Jack would go on to publish his dissertation, which suggested the extra returns associated with value stocks were likely driven by mispricing and not additional<?xmltex \pgtag{\nobreak}?> <?xmltex \pgtag{\hbox\bgroup}?>risk.<?xmltex \pgtag{\egroup}?></p>
<p xml:id="c02-para-0021">But nagging questions abounded: What gives a certain investor &ldquo;edge&rdquo;? What characteristics drive alpha? Why can one active investor (the winner) systematically take money from other investors (the losers)?</p></section>
<?xmltex \pgtag{\vfill\eject}?>
<section xml:id="c02-sec-0004"><title type="main">Enter Behavioral Finance</title><p xml:id="c02-para-0022"><?xmltex \OrgFixedPosition{c02-blkfxd-0002}?><?xmltex \pgtag{\Secfollowedepitrue}?><blockFixed type="standFirst" xml:id="c02-blkfxd-0002"><p xml:id="c02-para-0023">&ldquo;[Behavioral finance] has two building blocks: limits to arbitrage&hellip;and psychology.&rdquo;</p>
<source>&mdash;Nick Barberis and Richard Thaler<link href="c02-note-0004"/></source>
</blockFixed></p>
<p xml:id="c02-para-0024">As Wes plowed through thousands of stock&hyphen;picking proposals, one key takeaway became clear. These analysts were <i>good</i>. Collectively, they had skill. They were smart. They all made compelling cases that statistically outperformed in the aggregate. But Jack&apos;s dissertation research also found that harnessing the power of a computer to buy generically cheap stocks with strong fundamentals performed about equally well as the fundamental stock pickers that Wes had investigated in his dissertation. Value investing, whether driven by a human or a computer, beat the market. But why?</p>
<?xmltex \pgtag{\enlargethispage{1pc}}?>
<p xml:id="c02-para-0025">As mentioned, many in the market are smart and capable&mdash;intellect alone cannot be the driver of superior returns. What enabled value investors to buy low and sell high, and <i>why was the efficient market hypothesis not stopping them</i>?</p>
<p xml:id="c02-para-0026">John Maynard Keynes was a groundbreaking early&hyphen;twentieth&hyphen;century economist. He also spent many years as a professional investor, and may have had the answer. Keynes was a shrewd observer of financial markets and a successful investor in his time. But even Keynes struggled as an investor. At one point, Keynes was nearly wiped out while speculating on leveraged currencies (despite otherwise being a highly successful investor). His downfall led him to share one of the greatest investing mantras of all time:<link href="c02-note-0005"/><?xmltex \OrgFixedPosition{c02-feafxd-0003}?>
<featureFixed xml:id="c02-feafxd-0003" lwtype="Extract"><title type="featureFixedName">Extract</title><p xml:id="c02-para-0027">&ldquo;Markets can remain irrational longer than you can remain<?xmltex \pgtag{\break}?> solvent.&rdquo;</p>
</featureFixed></p>
<p xml:id="c02-para-0028">Keynes&apos;s quip highlights two key elements of real world markets that the efficient market hypothesis (EMH) doesn&apos;t consider: Investors can be irrational and the attempt to exploit market mispricing, or <i>arbitrage,</i> is risky. We can break Keynes&apos;s quote into academic parlance: First, the phrase &ldquo;&hellip; longer than you can remain solvent&rdquo; speaks to the fact that arbitrage is risky and is referred to by academics as &ldquo;limits to arbitrage.&rdquo; Second, the &ldquo;Markets can remain irrational&hellip;&rdquo; component speaks to investor psychology, which is an area of research that has been well developed by professional psychologists. These two elements&mdash;limits to arbitrage and investor psychology&mdash;are the building blocks for so&hyphen;called behavioral finance (depicted in Figure<?xmltex \pgtag{\nobreak}?> <link href="c02-fig-0001"/>).</p>
<?xmltex \OrgFixedPosition{c02-fig-0001}?>
<figure xml:id="c02-fig-0001">
<mediaResource href="urn:x-wiley:9781119237198:media:w9781119237198c02:c02f001" alt="image"/>
<caption>The Two Pillars of Behavioral Finance</caption>
<?xmltex \pgtag{\bgroup\FloatPositionToptrue\putfigure{1}{c02/c02f001.eps}{}{}{}\egroup}?></figure><section xml:id="c02-sec-0005"><title type="main">Limits to<?xmltex \pgtag{\protect\nobreak}?> Arbitrage</title><p xml:id="c02-para-0029">The efficient market hypothesis predicts that prices quickly reflect fundamental value. Why? Smart investors are greedy and any mispricing in the market is an opportunity to make a quick profit. As the logic goes, price dislocations are ephemeral because they are immediately rectified by the proverbial &ldquo;smart money.&rdquo; In the real world, true arbitrage opportunities&mdash;where profits are earned with zero risk after all possible costs&mdash;rarely, if ever, exist. Most &ldquo;arbitrage&rdquo; is really <i>risk arbitrage</i> that involves some form of cost that doesn&apos;t exist in a theoretical pricing model. Let&apos;s look at a simple example of exploiting mispricing opportunities in the orange market. Our basic assumptions are listed below:
<list xml:id="c02-list-0002" style="bulleted"><listItem xml:id="c02-li-0003">Oranges in Florida sell for &dollar;1 each.</listItem>
<listItem xml:id="c02-li-0004">Oranges in California sell for &dollar;2 each.</listItem>
<listItem xml:id="c02-li-0005">The fundamental value of an orange is &dollar;1.</listItem>
</list>
</p>
<p xml:id="c02-para-0030">The EMH suggests arbitrageurs will buy oranges in Florida and immediately sell oranges in California until California orange prices are driven to their fundamental value, which is &dollar;1. In a vacuum, the situation above is an arbitrage. However, there are obvious costs to conduct this arbitrage. For example, what if it costs &dollar;1 to ship oranges from Florida to California? Prices are decidedly not correct&mdash;the fundamental value of an orange is &dollar;1&mdash;but there is no free lunch, since the shipping costs are a limit to arbitrage. Savvy arbitrageurs will be prevented from exploiting the opportunity (in this case, due to &ldquo;frictional&rdquo; shipping costs).</p></section>
<?xmltex \pgtag{\enlargethispage{1pc}}?>
<section xml:id="c02-sec-0006"><title type="main">Investor Psychology</title><p xml:id="c02-para-0031">News flash: Human beings are not rational 100 percent of the time. To anyone who has driven without wearing a seat belt, or hit the snooze button on an alarm clock, this should be pretty clear. The literature from top psychologists is overwhelming for the remaining naysayers. Daniel Kahneman, the Nobel&hyphen;prize winning psychologist and author of the <i>New<?xmltex \pgtag{\nb}?> York Times</i> bestseller <i>Thinking, Fast and Slow</i>, tells a story of two modes of thinking: System 1 and System 2.<link href="c02-note-0006"/> System 1 is the &ldquo;think fast, survive in the jungle&rdquo; portion of the human brain. When we start to run away from a poisonous snake, even if later on, it turns out to be a stick, we are relying on our trusty System 1. System 2 is the analytic and calculating portion of the brain that is slower, but always rational. When we are comparing the costs and benefits of refinancing a mortgage, we are likely using System<?xmltex \pgtag{\nobreak}?> 2.</p>
<p xml:id="c02-para-0032">System 1 keeps us alive in the jungle; System 2 helps us make rational decisions for long&hyphen;term benefit. Both serve their purpose; however, sometimes one system can muscle onto the turf of the other. When System 1 starts making System 2 decisions, we can get in a lot of trouble. For example, do any of these sound familiar?
<list xml:id="c02-list-0003" style="bulleted"><listItem xml:id="c02-li-0006">&ldquo;That diamond bracelet was so beautiful; I just had to buy it.&rdquo;</listItem>
<listItem xml:id="c02-li-0007">&ldquo;Dessert comes free with dinner; of course I had to have some.&rdquo;</listItem>
<listItem xml:id="c02-li-0008">&ldquo;Home prices never seem to go down; we&apos;ve got to buy!&rdquo;</listItem>
</list>
</p>
<p xml:id="c02-para-0033">Unfortunately, the efficiency of System 1 comes with drawbacks&mdash;what keeps us alive in the jungle isn&apos;t necessarily what saves us from ourselves in financial markets.</p>
<p xml:id="c02-para-0034">Now, let&apos;s combine our irrational investors (System 1 types) with the limits of arbitrage, or market frictions, that we discussed above. We&apos;re in a situation where smart investors can&apos;t take advantage of the System 1 types for some reason. Combining bad investor behaviors with the frictions that smart people run into, could create compelling investment opportunities for uniquely situated investors.</p>
<?xmltex \pgtag{\enlargethispage{1pc}}?>
<p xml:id="c02-para-0035">For example, consider the concept of &ldquo;noise traders:&rdquo; think day traders that ignore fundamentals and trade on &ldquo;gut&rdquo;&mdash;classic System 1 types. These irrational noise traders can dislocate prices from fundamentals, but because these traders are irrational, arbitrageurs have a hard time pinning down the timing and duration of these irrational trades. Thus, going back to the idea that markets can remain irrational longer than you can remain solvent, an element of risk arises when an arbitrageur tries to exploit a noise trader. Sure, noise traders are irrational now, but perhaps they will be even more irrational tomorrow? Brad DeLong, Andrei Shleifer, Larry Summers, and Robert Waldmann described this phenomenon in &ldquo;Noise Trader Risk in Financial Markets,&rdquo; in the <i>Journal of Political Economy</i> in 1990.<link href="c02-note-0007"/> Here is an abridged abstract from the paper:<?xmltex \OrgFixedPosition{c02-feafxd-0004}?>
<featureFixed xml:id="c02-feafxd-0004" lwtype="Extract"><title type="featureFixedName">Extract</title><p xml:id="c02-para-0036">The unpredictability of noise traders&apos; beliefs creates a risk in the price of the asset that deters rational arbitragers from aggressively betting against them. As a result, prices can diverge significantly from fundamental values even in the absence of fundamental risk&hellip;</p>
</featureFixed></p>
<p xml:id="c02-para-0037"><?xmltex \pgtag{\looseness=-1}?>Let&apos;s translate this into English: Day traders mess up prices, and although these people are idiots, you don&apos;t know the extent of their idiocy, and you can&apos;t really time the strategy of an idiot anyway, so most smart people don&apos;t even try to take advantage of them. Consequently, prices move around a lot more than they should because no one is stopping the idiots. It&apos;s too risky! Moreover, since prices move around a lot more, the returns can be higher, so some lucky idiots may think they are actually good at timing markets, which incentivizes more idiots to do more idiotic things. This combination of bad behavior and market frictions describes what behavioral finance is all about: <b>Behavioral bias &plus; Market frictions &equals; Mispriced assets.</b></p>
<?xmltex \pgtag{\enlargethispage{1pc}}?>
<p xml:id="c02-para-0038">And while this working definition of behavioral finance may seem simple, the debate surrounding behavioral finance is far from settled. In one corner, the efficient market clergy claims that behavioral finance is heresy, reserved for those economists who have lost their way and diverted from the &ldquo;truth.&rdquo; In their view, prices always reflect fundamental value. Some in the efficient market camp point to the evidence that active managers can&apos;t beat the market in the aggregate and incorrectly conclude that prices are always efficient as a result. In the other corner, practitioners that leverage &ldquo;behavioral bias&rdquo; suggest that they have an edge because they exploit investors with behavioral bias. Yet, practitioners who make these claims often have terrible performance.<link href="c02-note-0008"/></p>
<p xml:id="c02-para-0039"><i>So where is the disconnect?</i></p>
<p xml:id="c02-para-0040">The disconnect lies in the fact that both sides of the argument fail to assess mispricing opportunities <i>and</i> the limits to arbitrage, simultaneously. The efficient market believers correctly identify that practitioners often lose to the market, but fail to consider the limits to arbitrage, which suggest that prices can deviate from fundamentals, but still not be profitable for active managers. Practitioners acknowledge mispricing opportunities, but they ignore the limits of arbitrage, which make mispricing opportunities too costly to profitably exploit. In other words, behavioral finance is a possible answer to everyone&apos;s problems. Behavioral finance can explain why we observe <i>inefficient market prices</i> and why we observe that most <i>active managers can&apos;t beat the market</i>.<link href="c02-note-0009"/></p></section>
</section>
</section>
<section xml:id="c02-sec-0007"><title type="main">Good Investing Is Like Good Poker:<?xmltex \pgtag{\protect\break}?> Pick the<?xmltex \pgtag{\protect\nobreak}?> Right Table</title><p xml:id="c02-para-0041">Behavioral finance hints at a framework for being a successful active investor:
<?xmltex \pgtag{\def\itemwd{3.}}?>
<list xml:id="c02-list-0004" style="1"><listItem xml:id="c02-li-0009">Identify market situations where behavioral bias is driving prices from fundamentals (e.g., identify market opportunity).</listItem>
<listItem xml:id="c02-li-0010">Identify the actions/incentives of the smartest market participants and understand their arbitrage costs.</listItem>
<listItem xml:id="c02-li-0011">Find situations where mispricing is high and arbitrage costs are high for the majority of arbitrage capital, but the costs are low for an active investor with low arbitrage costs.</listItem>
</list>
</p>
<p xml:id="c02-para-0042">One can think of the situation outlined above as analogous to a poker player seeking to find a winnable poker game. And in the context of poker, picking the right table is critical for success:
<?xmltex \pgtag{\def\itemwd{3.}}?>
<list xml:id="c02-list-0005" style="1"><listItem xml:id="c02-li-0012">Know the fish at the table (opportunity is high).</listItem>
<listItem xml:id="c02-li-0013">Know the sharks at the table (opportunity is low).</listItem>
<listItem xml:id="c02-li-0014">Find a table with a lot of fish and few sharks.</listItem>
</list>
</p>
<p xml:id="c02-para-0043">Following the poker analogy, in Figure<?xmltex \pgtag{\nobreak}?> <link href="c02-fig-0002"/>, the graphic outlines the questions we must ask as an active investor in the marketplace:
<?xmltex \pgtag{\def\itemwd{2.}}?>
<list xml:id="c02-list-0006" style="1"><listItem xml:id="c02-li-0015">Who is the worst player at the table?</listItem>
<listItem xml:id="c02-li-0016">Who is the best player at the table?</listItem>
</list>
</p>
<?xmltex \OrgFixedPosition{c02-fig-0002}?>
<figure xml:id="c02-fig-0002">
<mediaResource href="urn:x-wiley:9781119237198:media:w9781119237198c02:c02f002" alt="image"/>
<caption>Identifying Opportunity in the Market</caption>
<?xmltex \pgtag{\bgroup\FloatPositionBottrue\putfigure{2}{c02/c02f002.eps}{}{}{}\egroup}?></figure>
<p xml:id="c02-para-0044">To be successful over the long haul, an active investor needs to be good at identifying market opportunities created by poor investors, but also skilled at identifying situations where savvy market participants are unable or unwilling to act because their arbitrage costs are too<?xmltex \pgtag{\nobreak}?> <?xmltex \pgtag{\hbox\bgroup}?>high.<?xmltex \pgtag{\egroup}?></p><section xml:id="c02-sec-0008"><title type="main">Understanding the<?xmltex \pgtag{\protect\nobreak}?> Worst Players</title><p xml:id="c02-para-0045">All human beings suffer from behavioral bias, and these biases are magnified in stressful situations. After all, we&apos;re only human.</p>
<p xml:id="c02-para-0046">We laundry list a plethora of biases that can affect investment decisions on the financial battlefield:
<list xml:id="c02-list-0007" style="bulleted"><listItem xml:id="c02-li-0017">Overconfidence (&ldquo;I&apos;ve been right before&hellip;&rdquo;)</listItem>
<listItem xml:id="c02-li-0018">Optimism (&ldquo;Markets always go up.&rdquo;)</listItem>
<listItem xml:id="c02-li-0019">Self&hyphen;attribution bias (&ldquo;I called that stock price increase&hellip;&rdquo;)</listItem>
<listItem xml:id="c02-li-0020">Endowment effect (&ldquo;I have worked with this manager for 25 years; he has to be good.&rdquo;)</listItem>
<listItem xml:id="c02-li-0021">Anchoring (&ldquo;The market was up 50 percent last year; I think it will return between 45 and 55 percent this year.&rdquo;)</listItem>
<listItem xml:id="c02-li-0022">Availability (&ldquo;You see the terrible results last quarter? This stock is a total dog!&rdquo;)</listItem>
<listItem xml:id="c02-li-0023">Framing (&ldquo;Do you prefer a bond that has a 99 percent chance of paying its promised yield or one with a 1 percent chance of default?&rdquo;&mdash;hint, it&apos;s the same bond.)</listItem>
</list>
</p>
<p xml:id="c02-para-0047">The psychology research is clear: humans are flawed decision makers, especially under duress. But even if we identify poor investor behavior, that identification does not necessarily imply that an exploitable market opportunity exists. As discussed previously, other smarter investors will surely be privy to the mispricing situation before we are aware of the opportunity. They will attempt immediately to exploit the opportunity, eliminating our ability to profitably take advantage of mispricing caused by biased market participants. We want to avoid competition, but to avoid competition we need to understand the competition.</p></section>
<section xml:id="c02-sec-0009"><title type="main">Understanding the<?xmltex \pgtag{\protect\nobreak}?> Best Poker Players</title><p xml:id="c02-para-0048">In the context of financial markets, the best pokers players are often those investors managing the largest amounts of money. These market participants are exemplified by the hedge funds with all&hyphen;star managers or institutional titans running massive fund complexes. The resources available to these investors are remarkable and vast. One can rarely overpower this sort of opponent. Thankfully, overwhelming strength isn&apos;t the only way to slay Goliath. One can outmaneuver these titans because many top players are hamstrung by perverse economic incentives.</p>
<p xml:id="c02-para-0049">Before we dive into the incentives of these savvy players, let&apos;s quickly review the concept of arbitrage. The textbook definition of <i>arbitrage</i> involves a costless investment that generates riskless profits, by taking advantage of mispricings across different instruments representing the same security (think back to our orange example). In reality, arbitrage entails costs as well as the assumption of risk, and for these reasons there are limits to the effectiveness of arbitrage. There is ample evidence for such limits to arbitrage. Examples include the following:
<list style="plain" xml:id="c02-list-0008"><listItem xml:id="c02-li-0024"><b>Fundamental Risk.</b> Arbitragers may identify a mispricing of a security that does not have a perfect substitute that enables riskless arbitrage. If a piece of bad news affects the substitute security involved in hedging, the arbitrager may be subject to unanticipated losses. An example would be Ford and GM&mdash;similar stocks, but they are not the same company.</listItem>
<listItem xml:id="c02-li-0025"><b>Noise Trader Risk.</b> Once a position is taken, noise traders may drive prices farther from fundamental value, and the arbitrageur may be forced to invest additional capital, which may not be available, forcing an early liquidation of the position.</listItem>
<listItem xml:id="c02-li-0026"><b>Implementation Costs.</b> Short selling is often used in the arbitrage process, although it can be expensive because of the &ldquo;short rebate,&rdquo; which represents the costs to borrow the stock to be sold short. In some cases, such borrowing costs may exceed potential profits. For example, if short rebate fees are 10 percent and the expected arbitrage profits are 9 percent, there is no way to profit from the mispricing.</listItem>
</list>
</p>
<?xmltex \pgtag{\enlargethispage{1.2pc}}?>
<p xml:id="c02-para-0050">The three market frictions mentioned are important. There are potentially many others, but the biggest risk for most smart players is the balance they must strike between long&hyphen;term expected performance and career risk. An explanation is in order. The biggest short&hyphen;circuit to the arbitrage process are the limits imposed on smart fund managers that face short&hyphen;term focused performance assessments. Consider the pressures produced by <i>tracking error,</i> or the tendency of returns to deviate from a standard benchmark. Say a professional investor has a job investing the pensions of 100,000 firemen. They have a choice of investment strategies. They can invest in the following options:
<list xml:id="c02-list-0009" style="bulleted"><listItem xml:id="c02-li-0027"><b>Strategy A:</b> A strategy that they know (by some magical means) will beat the market by 1 percent per year over 25 years. But, they also know that this strategy will never underperform the index by more than 1 percent in a given year; or</listItem>
<listItem xml:id="c02-li-0028"><?xmltex \pgtag{\changespaceskip{2.3}}?><b>Strategy B:</b> An arbitrage strategy that the investor knows (again, by some magical means) will outperform the market, on average, by 5 percent per year over the next 25 years. The catch is that the investor also knows that they will have a 5&hyphen;year period where they underperform by 5 percent per<?xmltex \pgtag{\nb}?> year.</listItem>
</list>
</p>
<p xml:id="c02-para-0051">Which strategy does the investment professional choose? If they are being hired on behalf of 100,000 firemen, the choice is often obvious, despite being sub&hyphen;optimal for their investors: choose Strategy A and avoid getting fired!</p>
<p xml:id="c02-para-0052"><i>Why choose A?</i> This strategy is a bad long&hyphen;term strategy relative to B. The incentives of an investment manager are complex. Fund managers are not the owners of the capital, but work on behalf of someone who does. Financial mercenaries, if you will. These managers sometimes make decisions that increase the odds of them keeping their job, but will not necessarily maximize risk&hyphen;adjusted returns for their investors. For these managers, relative performance is everything and tracking error is dangerous. In the example above, the tracking error on Strategy B is just too painful to digest. Those firemen are going to start screaming bloody murder during the five years of underperformance, and the manager will not be around long enough to see the rebound when it occurs after year 5. But if the manager follows Strategy A, he can avoid career risk and the fireman&apos;s pension will not endure the stress of a prolonged downturn.</p>
<?xmltex \pgtag{\enlargethispage{1pc}}?>
<p xml:id="c02-para-0053">Over long time frames, a mispricing opportunity may be a mile wide&mdash;you could drive a proverbial truck through it. But this agency problem&mdash;the fact that the owners of the capital can, in the short&hyphen;term, begin to doubt the abilities of the arbitrageur and pull their capital&mdash;precludes smart managers from taking advantage of the long&hyphen;term mispricing opportunities that are highly volatile.</p>
<p xml:id="c02-para-0054">The threat of short&hyphen;term tracking&hyphen;error is very real. Consider the commonly cited example of Ken Heebner&apos;s CGM Focus Fund.<sup>10</sup> A <i>Wall Street Journal</i> (WSJ) article offers some facts relating to Ken&apos;s fund performance:<?xmltex \OrgFixedPosition{c02-feafxd-0005}?>
<featureFixed xml:id="c02-feafxd-0005" lwtype="Extract"><title type="featureFixedName">Extract</title><p xml:id="c02-para-0055">&ldquo;Ken Heebner&apos;s &dollar;3.7 billion CGM Focus Fund, rose more than 18&percnt; annually and outpaced its closest rival by more than three percentage points.&rdquo;</p>
</featureFixed></p>
<p xml:id="c02-para-0056">Next, the WSJ lays out additional facts relating to the performance of investors in Ken&apos;s fund:<?xmltex \OrgFixedPosition{c02-feafxd-0006}?>
<featureFixed xml:id="c02-feafxd-0006" lwtype="Extract"><title type="featureFixedName">Extract</title><p xml:id="c02-para-0057">&ldquo;Too bad investors weren&apos;t around to enjoy much of those gains. The typical CGM Focus shareholder lost 11&percnt; annually in the 10<?xmltex \pgtag{\nb}?> years ending Nov. 30&hellip;&rdquo;</p>
</featureFixed></p>
<p xml:id="c02-para-0058">Ken&apos;s fund compounded at 18 percent a year, and yet, the investors in the fund lost 11 percent a year, a reflection of the typical investor&apos;s inability to time effectively in and out of Ken&apos;s fund (see Figure<?xmltex \pgtag{\nobreak}?> <link href="c02-fig-0003"/>).<link href="c02-note-0011"/> When Ken&apos;s fund was underperforming (and the opportunity was high), they pulled capital; when his fund was outperforming (and opportunity was low), they invested more capital. On net, Ken looks like a genius, but few investors actually benefited from Ken&apos;s ability&mdash;a lose&hyphen;lose proposition.</p>
<?xmltex \OrgFixedPosition{c02-fig-0003}?>
<figure xml:id="c02-fig-0003">
<mediaResource href="urn:x-wiley:9781119237198:media:w9781119237198c02:c02f003" alt="image"/>
<caption>CGM Focus Fund from 1999 to 2009</caption>
<?xmltex \pgtag{\bgroup\FloatPositionToptrue\figbotskip=6pt\putfigure{3}{c02/c02f003.eps}{}{}{}\egroup}?></figure>
<p xml:id="c02-para-0059">Ken&apos;s Heebner&apos;s experience highlights this conflict of interest problem for asset managers. The dynamics of this problem are explored in an illuminating 1997 <i>Journal of Finance</i> paper by Andrei Shleifer and Robert Vishny, appropriately called &ldquo;The Limits of Arbitrage.&rdquo;<link href="c02-note-0012"/> The takeaway from Ken Heebner&apos;s experience and Shleifer and Vishny&apos;s insights is as follows: Smart managers <b>avoid long&hyphen;term market opportunities</b> if their investors are <b>focused on short&hyphen;term performance.</b></p>
<?xmltex \pgtag{\enlargethispage{1.2pc}}?>
<p xml:id="c02-para-0060">And can you blame the managers? If their careers depend on their relative performance over a month, a year, or even every five years, then asset managers will clearly care more about short&hyphen;term relative performance than about long&hyphen;term expected risk&hyphen;adjusted returns. Whether they are proactively protecting their jobs or the clients are actively driving the conversation around near&hyphen;sighted metrics, the end result is the same. Fund investors lose, and prices are not always efficient.</p></section>
<section xml:id="c02-sec-0010"><title type="main">Keys to<?xmltex \pgtag{\protect\nobreak}?> Long&hyphen;Term Active Management Success</title><p xml:id="c02-para-0061"><?xmltex \OrgFixedPosition{c02-blkfxd-0003}?><?xmltex \pgtag{\Secfollowedepitrue}?><blockFixed type="standFirst" xml:id="c02-blkfxd-0003"><p xml:id="c02-para-0062">&ldquo;There are a lot of smart people&hellip;so it&apos;s not easy to win.&rdquo;</p>
<source>&mdash;Charlie Munger, Vice Chairman Berkshire Hathaway<link href="c02-note-0013"/></source>
</blockFixed></p>
<p xml:id="c02-para-0063">We&apos;ve outlined a few elements of the marketplace. First, some investors are probably making poor investment decisions, and second, some managers are unable to exploit genuine market opportunities due to incentives. We encapsulate these elements in a simple equation for sustainable long&hyphen;term performance in Figure<?xmltex \pgtag{\nobreak}?> <link href="c02-fig-0004"/>.</p>
<?xmltex \OrgFixedPosition{c02-fig-0004}?>
<figure xml:id="c02-fig-0004">
<mediaResource href="urn:x-wiley:9781119237198:media:w9781119237198c02:c02f004" alt="image"/>
<caption>The Long&hyphen;Term Performance Equation</caption>
<?xmltex \pgtag{\bgroup\FloatPositionToptrue\figbotskip=-6pt\putfigure{4}{c02/c02f004.eps}{}{}{}\egroup}?></figure>
<p xml:id="c02-para-0064">The long&hyphen;term performance equation has two core elements:
<list xml:id="c02-list-0010" style="bulleted"><listItem xml:id="c02-li-0029">Sustainable alpha</listItem>
<listItem xml:id="c02-li-0030">Sustainable investors</listItem>
</list>
</p><?xmltex \pgtag{\enlargethispage*{1pc}}?>
<p xml:id="c02-para-0065">Sustainable alpha refers to an active stock selection process that systematically exploits mispricings caused by behavioral bias in the marketplace (i.e., finds the worst poker players). In order for this &ldquo;edge&rdquo; to be sustainable, it cannot be arbitraged away in the long run. Typically, sustainable edges are driven by strategies that require a long&hyphen;horizon and indifference to short&hyphen;term relative performance in order to be successful. That requirement brings us to our second element of the long&hyphen;term performance equation: sustainable investors. Sustainable investors cannot fall victim to the siren song of short&hyphen;term underperformance. If they do fall prey to short&hyphen;termism, these <i>un</i>sustainable investors will greatly enhance the arbitrage costs for their delegated asset manager, and will thus prevent the investors from profitably exploiting mispricing opportunities.</p>
<?xmltex \pgtag{\enlargethispage{1.5pc}}?>
<p xml:id="c02-para-0066">Based on the equation, if one can identify a process with an established edge (i.e., sustainable alpha) that requires long&hyphen;term discipline to exploit (i.e., requires sustainable investors), it is likely that this process will serve as a promising long&hyphen;term strategy that will beat the market over<?xmltex \pgtag{\nobreak}?> <?xmltex \pgtag{\hbox\bgroup}?>time.<?xmltex \pgtag{\egroup}?></p><section xml:id="c02-sec-0011"><title type="main">Moving from Theory to<?xmltex \pgtag{\protect\nobreak}?> Practice</title><p xml:id="c02-para-0067">Much of this discussion outlines an intellectual framework for successful active investing. There is no discussion of whether value investing is better than growth investing, or if high&hyphen;frequency trading is better than investing in pork belly futures. However, the building blocks to identify sustainable performance are simple to follow:
<?xmltex \pgtag{\vspace*{-3pt}}?>
<list xml:id="c02-list-0011" style="bulleted"><listItem xml:id="c02-li-0031">Identify a sustainable alpha process that can exploit bad players.</listItem>
<listItem xml:id="c02-li-0032">Understand the limitations of good players.</listItem>
<listItem xml:id="c02-li-0033">Exploit the opportunity by pairing a good process (that good players avoid exploiting)  with sustainable capital.</listItem></list></p>
<p xml:id="c02-para-0068">To put a little bit of meat on the bone, we provide an example of how this construct works in the &ldquo;value versus growth&rdquo; debate, which is a familiar discussion for most readers. To keep things simple and in line with academic research practices, we consider <i>value investing</i> to be approximated roughly by the practice of purchasing portfolios of firms with low prices to some fundamental price metric (e.g., a high book&hyphen;to&hyphen;market or B/M ratio). <i>Growth investing</i> is the opposite approach&mdash;purchase firms with high prices relative to fundamentals, with the expectation that fundamentals will grow rapidly. Using Ken French&apos;s data,<link href="c02-note-0014"/> we examine the returns from January 1, 1927, to December 31, 2014, for a value portfolio (high B/M decile, value&hyphen;weighted returns), a growth portfolio (low B/M decile, value&hyphen;weighted returns), and the S&amp;P 500 total return index. By <i>value&hyphen;weight,</i> we mean that each stock is given its weight in the portfolio, depending on the size of the firm. Results are shown in Table<?xmltex \pgtag{\nobreak}?> <link href="c02-tbl-0001"/>. All returns are total returns and include the reinvestment of distributions (e.g., dividends). Results are gross of<?xmltex \pgtag{\nobreak}?> <?xmltex \pgtag{\hbox\bgroup}?>fees.<?xmltex \pgtag{\egroup}?></p>
<?xmltex \OrgFixedPosition{c02-tbl-0001}?>
<?xmltex \pgtag{\bgroup\tabbotskip=-3pt\FloatPositionBottrue}?><tabular xml:id="c02-tbl-0001"><title type="main">Value versus Growth (1927 to 2014)</title><table pgwide="1" frame="topbot" rowsep="0" colsep="0"><tgroup cols="4"><colspec colnum="1" colname="col1" align="left"/><colspec colnum="2" colname="col2" align="center"/><colspec colnum="3" colname="col3" align="center"/><colspec colnum="4" colname="col4" align="center"/><lwtablebody><?xmltex \pgtag{\tabcolsep=0pt\begin{tabular*}{\textwidth}{@{\extracolsep\fill}ld{4}d{4}d{4}@{\extracolsep\fill}}\firsttablerule}?><thead valign="bottom"><!--<row rowsep="1">--><?xmltex \pgtag{\icolcnt=1\relax}?><entry colname="col1" align="center" xml:id="c02-ent-0001"></entry><entry colname="col2" xml:id="c02-ent-0002" align="center" lwPstyle="TabularHead">Value</entry><entry colname="col3" xml:id="c02-ent-0003" align="center" lwPstyle="TabularHead">Growth</entry><entry colname="col4" xml:id="c02-ent-0004" align="center" lwPstyle="TabularHead">SP500</entry><!--</row>--></thead><!--<tbody valign="top">--><!--<row>--><?xmltex \\\tablerule\pgtag{\icolcnt=1\relax}?><entry colname="col1" xml:id="c02-ent-0005"><b>CAGR</b></entry>
<entry colname="col2" xml:id="c02-ent-0006">12.41&percnt;</entry>
<entry colname="col3" xml:id="c02-ent-0007">8.70&percnt;</entry>
<entry colname="col4" xml:id="c02-ent-0008">9.95&percnt;</entry><!--</row>-->
<!--<row>--><?xmltex \\\pgtag{\icolcnt=1\relax}?><entry colname="col1" xml:id="c02-ent-0009"><b>Standard Deviation</b></entry>
<entry colname="col2" xml:id="c02-ent-0010">31.92&percnt;</entry>
<entry colname="col3" xml:id="c02-ent-0011">19.95&percnt;</entry>
<entry colname="col4" xml:id="c02-ent-0012">19.09&percnt;</entry><!--</row>-->
<!--<row>--><?xmltex \\\pgtag{\icolcnt=1\relax}?><entry colname="col1" xml:id="c02-ent-0013"><b>Downside Deviation</b></entry>
<entry colname="col2" xml:id="c02-ent-0014">21.34&percnt;</entry>
<entry colname="col3" xml:id="c02-ent-0015">14.41&percnt;</entry>
<entry colname="col4" xml:id="c02-ent-0016">14.22&percnt;</entry><!--</row>-->
<!--<row>--><?xmltex \\\pgtag{\icolcnt=1\relax}?><entry colname="col1" xml:id="c02-ent-0017"><b>Sharpe Ratio</b></entry>
<entry colname="col2" xml:id="c02-ent-0018">0.41</entry>
<entry colname="col3" xml:id="c02-ent-0019">0.35</entry>
<entry colname="col4" xml:id="c02-ent-0020">0.41</entry><!--</row>-->
<!--<row>--><?xmltex \\\pgtag{\icolcnt=1\relax}?><entry colname="col1" xml:id="c02-ent-0021"><b>Sortino Ratio (MAR</b> <math display="inline" overflow="scroll" xmlns="http://www.w3.org/1998/Math/MathML" xmlns:xlink="http://www.w3.org/1999/xlink"><mrow><mo mathvariant="bold">=</mo></mrow></math> <b>5&percnt;)</b></entry>
<entry colname="col2" xml:id="c02-ent-0022">0.54</entry>
<entry colname="col3" xml:id="c02-ent-0023">0.37</entry>
<entry colname="col4" xml:id="c02-ent-0024">0.45</entry><!--</row>-->
<!--<row>--><?xmltex \\\pgtag{\icolcnt=1\relax}?><entry colname="col1" xml:id="c02-ent-0025"><b>Worst Drawdown</b></entry>
<entry colname="col2" xml:id="c02-ent-0026">&ndash;91.67&percnt;</entry>
<entry colname="col3" xml:id="c02-ent-0027">&ndash;85.01&percnt;</entry>
<entry colname="col4" xml:id="c02-ent-0028">&ndash;84.59&percnt;</entry><!--</row>-->
<!--<row>--><?xmltex \\\pgtag{\icolcnt=1\relax}?><entry colname="col1" xml:id="c02-ent-0029"><b>Worst Month Return</b></entry>
<entry colname="col2" xml:id="c02-ent-0030">&ndash;43.98&percnt;</entry>
<entry colname="col3" xml:id="c02-ent-0031">&ndash;30.65&percnt;</entry>
<entry colname="col4" xml:id="c02-ent-0032">&ndash;28.73&percnt;</entry><!--</row>-->
<!--<row>--><?xmltex \\\pgtag{\icolcnt=1\relax}?><entry colname="col1" xml:id="c02-ent-0033"><b>Best Month Return</b></entry>
<entry colname="col2" xml:id="c02-ent-0034">98.65&percnt;</entry>
<entry colname="col3" xml:id="c02-ent-0035">42.16&percnt;</entry>
<entry colname="col4" xml:id="c02-ent-0036">41.65&percnt;</entry><!--</row>-->
<!--<row>--><?xmltex \\\pgtag{\icolcnt=1\relax}?><entry colname="col1" xml:id="c02-ent-0037"><b>Profitable Months</b></entry>
<entry colname="col2" xml:id="c02-ent-0038">60.51&percnt;</entry>
<entry colname="col3" xml:id="c02-ent-0039">59.09&percnt;</entry>
<entry colname="col4" xml:id="c02-ent-0040">61.74&percnt;</entry><!--</row>-->
<?xmltex \pgtag{\\ \lasttablerule\end{tabular*}}?><!--</tbody>-->
</lwtablebody></tgroup>
</table>
</tabular><?xmltex \pgtag{\egroup}?>
<p xml:id="c02-para-0069">The historical evidence is clear: value stocks from 1927 to 2014 have outperformed growth stocks&mdash;by a wide margin. The portfolio of value stocks earns a compound annual growth rate of 12.41 percent per year, whereas, the growth stock portfolio earns 8.70 percent per year&mdash;approximately a 4 percent annual spread in performance. This historical spread in returns, which has been repeatedly and consistently observed over time, has been labeled the <i>value anomaly</i> by academic researchers. Of course, academics argue over the reasons why the spread is large (e.g., value investing might earn higher returns because it is simply more risky or because of mispricing, as discussed earlier). This debate is best captured by a 2008 interview with Eugene Fama where he describes a personal conversation with Andrei Shleifer over a glass of wine.<link href="c02-note-0015"/> Fama highlights that Andrei believes the value premium is due to mispricing, whereas Fama attributes the value premium to higher risk. Bottom line: Great minds can disagree on the explanation, but nobody can dispute the empirical fact that value stocks have outperformed growth stocks by a wide margin over<?xmltex \pgtag{\nobreak}?> <?xmltex \pgtag{\hbox\bgroup}?>time.<?xmltex \pgtag{\egroup}?></p></section>
<section xml:id="c02-sec-0012"><title type="main">We Have the<?xmltex \pgtag{\protect\nobreak}?> Facts. Next Step: Identify Bad Players</title><p xml:id="c02-para-0070">The data highlight that value investing has higher expected returns than growth investing. But to better understand whether value will beat growth in the future we need to look through the sustainable active investing prism and identify if the spread is due to risk (the efficient market explanation) or mispricing (the behavioral finance explanation). For a valid mispricing argument, we need to identify if there are market participants making systematically poor decisions with respect to the purchase of value and growth stocks.</p>
<?xmltex \OrgFixedPosition{c02-fig-0005}?>
<figure xml:id="c02-fig-0005">
<mediaResource href="urn:x-wiley:9781119237198:media:w9781119237198c02:c02f005" alt="image"/>
<caption>Investors Extrapolate Past Growth Rates into the Future</caption>
<?xmltex \pgtag{\bgroup\FloatPositionBottrue\putfigure{5}{c02/c02f005.eps}{}{}{}\egroup}?></figure>
<p xml:id="c02-para-0071">Lakonishok, Shleifer, and Vishny (LSV) explore this question in their paper &ldquo;Contrarian Investment, Extrapolation, and Risk.&rdquo;<link href="c02-note-0016"/> LSV hypothesize that investors suffer from representative bias, a situation where investors naively extrapolate past growth rates too far into the future. Figure<?xmltex \pgtag{\nobreak}?> <link href="c02-fig-0005"/> highlights the concept from the LSV paper using updated data from Dechow and Sloan&apos;s 1997 paper, &ldquo;Returns to Contrarian Investment Strategies: Tests of Naive Expectations Hypothesis.&rdquo;<link href="c02-note-0017"/> The horizontal axis represents cheapness and sorts securities into buckets, from left to right, based on whether stocks are expensive (low book&hyphen;to&hyphen;market ratios) or cheap (high book&hyphen;to&hyphen;market ratios). The vertical axis represents <i>past</i> five&hyphen;year earnings growth rates for the respective valuation buckets. Stocks in Bucket 10 are the cheapest, and they exhibited (on average) a <i>negative</i> 1 percent earnings growth over the preceding five years.</p>
<p xml:id="c02-para-0072">The relationship is almost perfectly linear. Cheap stocks have terrible past earnings growth, whereas expensive stocks have had wonderful earnings growth over the past five years. No real surprise there, but it is interesting to see how well the data fits this relationship.</p>
<p xml:id="c02-para-0073">Figure<?xmltex \pgtag{\nobreak}?> <link href="c02-fig-0005"/> underscores the general market expectation that past earnings growth rates will continue into the future. Growth firms are expensive because market participants believe past growth rates will continue. Otherwise, why would they pay so much for these stocks? Meanwhile, value stocks are cheap for what seems like a good a reason&mdash;the market believes their poor past growth rates will continue as<?xmltex \pgtag{\nobreak}?> <?xmltex \pgtag{\hbox\bgroup}?>well.<?xmltex \pgtag{\egroup}?></p>
<p xml:id="c02-para-0074">But does this really happen? Do cheap stocks have poor realized future earnings growth and do expensive stocks have strong realized future earnings growth? This is an empirical question that can be tested with an experiment. Do growth firms continue to grow faster, on average, <i>or</i> is there a systematic flaw in market expectations?</p>
<p xml:id="c02-para-0075">In Figure<?xmltex \pgtag{\nobreak}?> <link href="c02-fig-0006"/>, we look at what happens to earnings growth over the <i>next five years</i>. Specifically, did the value stocks continue to exhibit <?xmltex \pgtag{\bgroup\mbox}?>terrible<?xmltex \pgtag{\egroup}?> <?xmltex \pgtag{\bgroup\mbox}?>earnings<?xmltex \pgtag{\egroup}?> growth as predicted? Did growth stocks maintain their terrific earnings growth?</p>
<?xmltex \OrgFixedPosition{c02-fig-0006}?>
<figure xml:id="c02-fig-0006">
<mediaResource href="urn:x-wiley:9781119237198:media:w9781119237198c02:c02f006" alt="image"/>
<caption>Realized Growth Rates Systematically Mean&hyphen;Revert</caption>
<?xmltex \pgtag{\bgroup\FloatPositionBottrue\putfigure{6}{c02/c02f006.eps}{}{}{}\egroup}?></figure>
<p xml:id="c02-para-0076">No, they did not. The chart is evidence of systematically poor poker playing. The realized earnings growth (dark bars) systematically reverts to the average growth rate across the universe. Value stocks outperform earnings growth expectations and growth stocks underperform their expectations, <i>systematically</i>. Take a moment to study this profound observation. This unexpected deviation from expectations leads to price movements that are favorable for cheap &ldquo;value&rdquo; stocks, and unfavorable for expensive &ldquo;growth&rdquo; stocks. This deviation explains, at least in part, why expensive stock investors underperform, cheap stock investors outperform, and passive investors receive something in between.</p>
<p xml:id="c02-para-0077">To summarize: Markets, on average, throw value stocks under the bus and clamor for growth stocks. From a poker playing perspective, buying growth stocks and selling value stocks is an example of a systematically poor strategy. Assuming that a great hand from the last round equals a winning hand in the next round is a losing approach. But what are the best poker players doing about this value anomaly situation, and can these poker players easily exploit the poor poker players?</p></section>
<section xml:id="c02-sec-0013"><title type="main">Next Step: Identify the<?xmltex \pgtag{\protect\nobreak}?> Actions of<?xmltex \pgtag{\protect\nobreak}?> the Best Players</title><p xml:id="c02-para-0078">It is unlikely that we will ever be the smartest investors in the world. For example, George Soros, Julian Robertson, Leon Cooperman, and Paul Tudor Jones will always be smarter than we are. But if we aren&apos;t going to be the best player at the investing table, how can we win against these high&hyphen;powered investors? We can win by finding those market opportunities where the smartest investors are reluctant to participate. But why would a smart investor <i>not</i> want to participate in a straightforward way to beat the market, such as through value<?xmltex \pgtag{\nb}?> investing?</p>
<p xml:id="c02-para-0079">As mentioned previously, smart investors often get endowed with large amounts of capital from a large group of diverse investors (again, think George Soros, Julian Robertson, Paul Tudor Jones, but also large institutions such as BlackRock, Fidelity, and so forth). This makes sense on many levels&mdash;investors want to give their money to smart people. The challenge is that the really smart investors are often managing money on behalf of investors that suffer from behavioral biases (System 1 thinkers). Shleifer and Vishny highlight, and the Ken Heebner example confirms, that many smart market participants are hamstrung by the short&hyphen;term performance measures imposed on them by their investors. &ldquo;How did you perform against the benchmark this quarter? What do your results look like year to date? What macroeconomic trends are you exploiting this month?&rdquo; All of these questions are commonplace in the market. The threat of being fired and replaced with a passive portfolio of Vanguard funds is an implied threat. When job security and client expectations trump long&hyphen;term value creation, funny things<?xmltex \pgtag{\nb}?> happen.</p>
<p xml:id="c02-para-0080">A remarkable paper by Markus Brunnermeier and Stefan Nagel, &ldquo;Hedge Funds and the Technology Bubble,&rdquo; highlights the warped incentives faced by the smartest investors who deal with other people&apos;s money.<link href="c02-note-0018"/> Contrary to all textbook teachings related to efficient price formation, the smart money sometimes can be incentivized to <i>enhance mispricing</i>, not trade against it! Brunnermeier and Nagel find that many hedge fund managers didn&apos;t try to capitalize on the mispricing between value and growth stocks in the Internet Bubble of the late 1990s&mdash;they actually bought growth stocks and sold value stocks. This action enabled them to more closely track the index&mdash;for a time. Meanwhile, the poor hedge funds that stuck to their value investing guns, for example, Julian Robertson of Tiger Funds, ended up with no assets under management and a busted business model.</p>
<p xml:id="c02-para-0081">But Julian Robertson wasn&apos;t the only famous value investor to lose his proverbial shirt during the 1994 to 1999 time period. Around this time, Barron&apos;s famously stated the following regarding Warren Buffett&apos;s relative performance:<link href="c02-note-0019"/><?xmltex \OrgFixedPosition{c02-feafxd-0007}?>
<featureFixed xml:id="c02-feafxd-0007" lwtype="Extract"><title type="featureFixedName">Extract</title><p xml:id="c02-para-0082">&ldquo;Warren Buffett may be losing his magic touch.&rdquo;</p>
</featureFixed></p>
<p xml:id="c02-para-0083">Barron&apos;s observation was, in many respects, fully warranted. Value investors as a group were destroyed by the market in the late 1990s. Generic value investing (shown in Figure<?xmltex \pgtag{\nobreak}?> <link href="c02-fig-0007"/>) underperformed the broader market by a large margin for six long years!</p>
<?xmltex \OrgFixedPosition{c02-fig-0007}?>
<figure xml:id="c02-fig-0007">
<mediaResource href="urn:x-wiley:9781119237198:media:w9781119237198c02:c02f007" alt="image"/>
<caption>Value Investing Can Underperform</caption>
<?xmltex \pgtag{\bgroup\FloatPositionBottrue\putfigure{7}{c02/c02f007.eps}{}{}{}\egroup}?></figure>
<p xml:id="c02-para-0084">Obviously, being a value investor requires a patience and faith that few investors possess. In theory, value investing is easy&mdash;buy and hold cheap stocks for the long haul&mdash;but in practice, true value investing <i>is almost impossible.</i></p>
<p xml:id="c02-para-0085">Using Ken French&apos;s data, we examined just how painful it was to be a value investor in the late 1990s. We examine the returns from January 1, 1994, to December 31, 1999, for a value portfolio (high book&hyphen;to&hyphen;market decile, value&hyphen;weighted portfolio returns), a growth portfolio (low book&hyphen;to&hyphen;market decile, value&hyphen;weighted portfolio returns), the S&amp;P 500 total return index (SP500), and the Russell 2000 total return index (R2K), a small&hyphen;cap index. Results are shown in Table<?xmltex \pgtag{\nobreak}?> <link href="c02-tbl-0002"/>. All returns are total returns and include the reinvestment of distributions (e.g., dividends). Results are gross of<?xmltex \pgtag{\nobreak}?> <?xmltex \pgtag{\hbox\bgroup}?>fees.<?xmltex \pgtag{\egroup}?></p>
<?xmltex \OrgFixedPosition{c02-tbl-0002}?>
<?xmltex \pgtag{\bgroup\tabtopskip=-6pt\FloatPositionBottrue}?><tabular xml:id="c02-tbl-0002"><title type="main">Value Investing Can Underperform (1994&ndash;1999)</title><table pgwide="1" frame="topbot" rowsep="0" colsep="0"><tgroup cols="5"><colspec colnum="1" colname="col1" align="left"/><colspec colnum="2" colname="col2" align="center"/><colspec colnum="3" colname="col3" align="center"/><colspec colnum="4" colname="col4" align="center"/><colspec colnum="5" colname="col5" align="center"/><lwtablebody><?xmltex \pgtag{\tabcolsep=0pt\begin{tabular*}{\textwidth}{@{\extracolsep\fill}ld{5}d{5}d{5}d{5}@{\extracolsep\fill}}\firsttablerule}?><thead valign="bottom"><!--<row rowsep="1">--><?xmltex \pgtag{\icolcnt=1\relax}?><entry colname="col1" align="center" xml:id="c02-ent-0041"></entry><entry colname="col2" xml:id="c02-ent-0042" align="center" lwPstyle="TabularHead">Value</entry><entry colname="col3" align="center" xml:id="c02-ent-0043" lwPstyle="TabularHead">Growth</entry><entry colname="col4" align="center" xml:id="c02-ent-0044" lwPstyle="TabularHead">SP500</entry><entry colname="col5" align="center" xml:id="c02-ent-0045" lwPstyle="TabularHead">R2K</entry><!--</row>--></thead><!--<tbody valign="top">--><!--<row>--><?xmltex \\\tablerule\pgtag{\icolcnt=1\relax}?><entry colname="col1" xml:id="c02-ent-0046"><b>CAGR</b></entry>
<entry colname="col2" xml:id="c02-ent-0047">18.35&percnt;</entry>
<entry colname="col3" xml:id="c02-ent-0048">27.71&percnt;</entry>
<entry colname="col4" xml:id="c02-ent-0049">23.84&percnt;</entry>
<entry colname="col5" xml:id="c02-ent-0050">13.39&percnt;</entry><!--</row>-->
<!--<row>--><?xmltex \\\pgtag{\icolcnt=1\relax}?><entry colname="col1" xml:id="c02-ent-0051"><b>Standard Deviation</b></entry>
<entry colname="col2" xml:id="c02-ent-0052">11.79&percnt;</entry>
<entry colname="col3" xml:id="c02-ent-0053">16.53&percnt;</entry>
<entry colname="col4" xml:id="c02-ent-0054">13.63&percnt;</entry>
<entry colname="col5" xml:id="c02-ent-0055">16.96&percnt;</entry><!--</row>-->
<!--<row>--><?xmltex \\\pgtag{\icolcnt=1\relax}?><entry colname="col1" xml:id="c02-ent-0056"><b>Downside Deviation</b></entry>
<entry colname="col2" xml:id="c02-ent-0057">7.59&percnt;</entry>
<entry colname="col3" xml:id="c02-ent-0058">11.25&percnt;</entry>
<entry colname="col4" xml:id="c02-ent-0059">10.50&percnt;</entry>
<entry colname="col5" xml:id="c02-ent-0060">14.27&percnt;</entry><!--</row>-->
<!--<row>--><?xmltex \\\pgtag{\icolcnt=1\relax}?><entry colname="col1" xml:id="c02-ent-0061"><b>Sharpe Ratio</b></entry>
<entry colname="col2" xml:id="c02-ent-0062">1.09</entry>
<entry colname="col3" xml:id="c02-ent-0063">1.28</entry>
<entry colname="col4" xml:id="c02-ent-0064">1.30</entry>
<entry colname="col5" xml:id="c02-ent-0065">0.55</entry><!--</row>-->
<!--<row>--><?xmltex \\\pgtag{\icolcnt=1\relax}?><entry colname="col1" xml:id="c02-ent-0066"><b>Sortino Ratio (MAR</b> <math display="inline" overflow="scroll" xmlns="http://www.w3.org/1998/Math/MathML" xmlns:xlink="http://www.w3.org/1999/xlink"><mrow><mo mathvariant="bold">=</mo></mrow></math> <b>5&percnt;)</b></entry>
<entry colname="col2" xml:id="c02-ent-0067">1.66</entry>
<entry colname="col3" xml:id="c02-ent-0068">1.87</entry>
<entry colname="col4" xml:id="c02-ent-0069">1.67</entry>
<entry colname="col5" xml:id="c02-ent-0070">0.64</entry><!--</row>-->
<!--<row>--><?xmltex \\\pgtag{\icolcnt=1\relax}?><entry colname="col1" xml:id="c02-ent-0071"><b>Worst Drawdown</b></entry>
<entry colname="col2" xml:id="c02-ent-0072">&ndash;11.58&percnt;</entry>
<entry colname="col3" xml:id="c02-ent-0073">&ndash;16.33&percnt;</entry>
<entry colname="col4" xml:id="c02-ent-0074">&ndash;15.18&percnt;</entry>
<entry colname="col5" xml:id="c02-ent-0075">&ndash;29.78&percnt;</entry><!--</row>-->
<!--<row>--><?xmltex \\\pgtag{\icolcnt=1\relax}?><entry colname="col1" xml:id="c02-ent-0076"><b>Worst Month Return</b></entry>
<entry colname="col2" xml:id="c02-ent-0077">&ndash;8.62&percnt;</entry>
<entry colname="col3" xml:id="c02-ent-0078">&ndash;14.92&percnt;</entry>
<entry colname="col4" xml:id="c02-ent-0079">&ndash;14.31&percnt;</entry>
<entry colname="col5" xml:id="c02-ent-0080">&ndash;19.42&percnt;</entry><!--</row>-->
<!--<row>--><?xmltex \\\pgtag{\icolcnt=1\relax}?><entry colname="col1" xml:id="c02-ent-0081"><b>Best Month Return</b></entry>
<entry colname="col2" xml:id="c02-ent-0082">8.05&percnt;</entry>
<entry colname="col3" xml:id="c02-ent-0083">10.69&percnt;</entry>
<entry colname="col4" xml:id="c02-ent-0084">8.04&percnt;</entry>
<entry colname="col5" xml:id="c02-ent-0085">11.32&percnt;</entry><!--</row>-->
<!--<row>--><?xmltex \\\pgtag{\icolcnt=1\relax}?><entry colname="col1" xml:id="c02-ent-0086"><b>Profitable Months</b></entry>
<entry colname="col2" xml:id="c02-ent-0087">68.06&percnt;</entry>
<entry colname="col3" xml:id="c02-ent-0088">70.83&percnt;</entry>
<entry colname="col4" xml:id="c02-ent-0089">73.61&percnt;</entry>
<entry colname="col5" xml:id="c02-ent-0090">66.67&percnt;</entry><!--</row>-->
<?xmltex \pgtag{\\ \lasttablerule\end{tabular*}}?><!--</tbody>-->
</lwtablebody></tgroup>
</table>
</tabular><?xmltex \pgtag{\egroup}?>
<p xml:id="c02-para-0086">The returns to the value portfolio were not bad on an absolute basis, but on a relative basis, value was horrific. Looking at the annual returns (shown in Table<?xmltex \pgtag{\nobreak}?> <link href="c02-tbl-0003"/>), value investing lost almost every year to a simple passive market allocation!</p>
<?xmltex \OrgFixedPosition{c02-tbl-0003}?>
<?xmltex \pgtag{\bgroup\tabbotskip=-3pt\tabtopskip=-7pt\FloatPositionBottrue}?><tabular xml:id="c02-tbl-0003"><title type="main">Annual Returns</title><table pgwide="1" frame="topbot" rowsep="0" colsep="0"><tgroup cols="5"><colspec colnum="1" colname="col1" align="center"/><colspec colnum="2" colname="col2" align="center"/><colspec colnum="3" colname="col3" align="center"/><colspec colnum="4" colname="col4" align="center"/><colspec colnum="5" colname="col5" align="center"/><lwtablebody><?xmltex \pgtag{\tabcolsep=0pt\begin{tabular*}{\textwidth}{@{\extracolsep\fill}cd{4}d{4}d{4}d{4}@{\extracolsep\fill}}\firsttablerule}?><thead valign="bottom"><!--<row rowsep="1">--><?xmltex \pgtag{\icolcnt=1\relax}?><entry colname="col1" align="center" xml:id="c02-ent-0091"></entry><entry colname="col2" align="center" xml:id="c02-ent-0092" lwPstyle="TabularHead">Value</entry><entry colname="col3" align="center" xml:id="c02-ent-0093" lwPstyle="TabularHead">Growth</entry><entry colname="col4" align="center" xml:id="c02-ent-0094" lwPstyle="TabularHead">SP500</entry><entry colname="col5" align="center"  xml:id="c02-ent-0095" lwPstyle="TabularHead">R2K</entry><!--</row>--></thead><!--<tbody valign="top">--><!--<row>--><?xmltex \\\tablerule\pgtag{\icolcnt=1\relax}?><entry colname="col1" xml:id="c02-ent-0096"><b>1994</b></entry>
<entry colname="col2" xml:id="c02-ent-0097">&ndash;2.83&percnt;</entry>
<entry colname="col3" xml:id="c02-ent-0098">2.53&percnt;</entry>
<entry colname="col4" xml:id="c02-ent-0099">1.35&percnt;</entry>
<entry colname="col5" xml:id="c02-ent-0100">&ndash;1.82&percnt;</entry><!--</row>-->
<!--<row>--><?xmltex \\\pgtag{\icolcnt=1\relax}?><entry colname="col1" xml:id="c02-ent-0101"><b>1995</b></entry>
<entry colname="col2" xml:id="c02-ent-0102">36.47&percnt;</entry>
<entry colname="col3" xml:id="c02-ent-0103">35.47&percnt;</entry>
<entry colname="col4" xml:id="c02-ent-0104">37.64&percnt;</entry>
<entry colname="col5" xml:id="c02-ent-0105">28.45&percnt;</entry><!--</row>-->
<!--<row>--><?xmltex \\\pgtag{\icolcnt=1\relax}?><entry colname="col1" xml:id="c02-ent-0106"><b>1996</b></entry>
<entry colname="col2" xml:id="c02-ent-0107">14.22&percnt;</entry>
<entry colname="col3" xml:id="c02-ent-0108">23.20&percnt;</entry>
<entry colname="col4" xml:id="c02-ent-0109">23.23&percnt;</entry>
<entry colname="col5" xml:id="c02-ent-0110">16.49&percnt;</entry><!--</row>-->
<!--<row>--><?xmltex \\\pgtag{\icolcnt=1\relax}?><entry colname="col1" xml:id="c02-ent-0111"><b>1997</b></entry>
<entry colname="col2" xml:id="c02-ent-0112">32.52&percnt;</entry>
<entry colname="col3" xml:id="c02-ent-0113">31.15&percnt;</entry>
<entry colname="col4" xml:id="c02-ent-0114">33.60&percnt;</entry>
<entry colname="col5" xml:id="c02-ent-0115">22.36&percnt;</entry><!--</row>-->
<!--<row>--><?xmltex \\\pgtag{\icolcnt=1\relax}?><entry colname="col1" xml:id="c02-ent-0116"><b>1998</b></entry>
<entry colname="col2" xml:id="c02-ent-0117">29.75&percnt;</entry>
<entry colname="col3" xml:id="c02-ent-0118">44.23&percnt;</entry>
<entry colname="col4" xml:id="c02-ent-0119">29.32&percnt;</entry>
<entry colname="col5" xml:id="c02-ent-0120">&ndash;2.55&percnt;</entry><!--</row>-->
<!--<row>--><?xmltex \\\pgtag{\icolcnt=1\relax}?><entry colname="col1" xml:id="c02-ent-0121"><b>1999</b></entry>
<entry colname="col2" xml:id="c02-ent-0122">5.45&percnt;</entry>
<entry colname="col3" xml:id="c02-ent-0123">33.90&percnt;</entry>
<entry colname="col4" xml:id="c02-ent-0124">21.35&percnt;</entry>
<entry colname="col5" xml:id="c02-ent-0125">21.26&percnt;</entry><!--</row>-->
<?xmltex \pgtag{\\ \lasttablerule\end{tabular*}}?><!--</tbody>-->
</lwtablebody></tgroup>
</table>
</tabular><?xmltex \pgtag{\egroup}?>
<p xml:id="c02-para-0087">A plain&hyphen;vanilla index fund (SP500) outperforms value five out of six years in a row, sometimes by double&hyphen;digit figures! To simulate what these value managers went through, ask yourself this question:<?xmltex \OrgFixedPosition{c02-feafxd-0008}?>
<featureFixed xml:id="c02-feafxd-0008" lwtype="Extract"><title type="featureFixedName">Extract</title><p xml:id="c02-para-0088">If your asset managers underperformed a benchmark for five out of six years, at times by double digits, would you fire them?</p>
</featureFixed></p>
<p xml:id="c02-para-0089">For 99.9 percent of investors, that answer would be a resounding, &ldquo;Yes!&rdquo; (and giving someone a six&hyphen;year trial period is probably out of the question to begin with). Most&mdash;if not all&mdash;professional asset managers would be fired, given this underperformance. Asset managers know this evidence intuitively, and internalize the results by avoiding purist value investing endeavors that could make them look like fools in the short<?xmltex \pgtag{\nobreak}?> <?xmltex \pgtag{\hbox\bgroup}?>run.<?xmltex \pgtag{\egroup}?></p>
<p xml:id="c02-para-0090">After viewing the six&hyphen;year value investing pain train, we can identify two key takeaways:
<?xmltex \pgtag{\def\itemwd{2.}}?>
<list xml:id="c02-list-0012" style="1"><listItem xml:id="c02-li-0034">For a long&hyphen;term investor, a six&hyphen;year stretch of pain is a truly great thing. Why? Because this will limit competition from the best pokers players, for whom career risks trump performance considerations, and the weak hands will be shaken out of the competition.</listItem>
<listItem xml:id="c02-li-0035">Sustainable active investing requires special investors. It requires that investors be disciplined, have a long&hyphen;term horizon, and be indifferent to short&hyphen;term relative performance. These unique investors are what we had previously labeled <i>sustainable investors</i> in Figure<?xmltex \pgtag{\nobreak}?> <link href="c02-fig-0004"/>.</listItem>
</list>
</p>
<p xml:id="c02-para-0091">Now, suspend reality for a moment and let&apos;s imagine that an active value manager had clients that didn&apos;t run for the exits in 1999. What would their hypothetical returns look like over the long run? As one can see in Table<?xmltex \pgtag{\nobreak}?> <link href="c02-tbl-0004"/>, value quickly recovers and handily outperforms over the entire time period thereafter. Table<?xmltex \pgtag{\nobreak}?> <link href="c02-tbl-0004"/> shows the returns to the same portfolios from January 1, 2000, to December 31, 2014, the 15 years following the six&hyphen;year period of underperformance.</p>
<?xmltex \OrgFixedPosition{c02-tbl-0004}?>
<?xmltex \pgtag{\bgroup\FloatPositionBottrue}?><tabular xml:id="c02-tbl-0004"><title type="main">Summary Statistics (2000&ndash;2014)</title><table pgwide="1" frame="topbot" rowsep="0" colsep="0"><tgroup cols="5"><colspec colnum="1" colname="col1" align="left"/><colspec colnum="2" colname="col2" align="center"/><colspec colnum="3" colname="col3" align="center"/><colspec colnum="4" colname="col4" align="center"/><colspec colnum="5" colname="col5" align="center"/><lwtablebody><?xmltex \pgtag{\tabcolsep=0pt\begin{tabular*}{\textwidth}{@{\extracolsep\fill}ld{3.4}d{3.4}d{3.4}d{3.4}@{\extracolsep\fill}}\firsttablerule}?><thead valign="bottom"><!--<row rowsep="1">--><?xmltex \pgtag{\icolcnt=1\relax}?><entry colname="col1" align="center" xml:id="c02-ent-0126"></entry><entry colname="col2" xml:id="c02-ent-0127" align="center" lwPstyle="TabularHead">Value</entry><entry colname="col3" align="center" xml:id="c02-ent-0128" lwPstyle="TabularHead">Growth</entry><entry colname="col4" align="center" xml:id="c02-ent-0129" lwPstyle="TabularHead">SP500</entry><entry colname="col5" align="center" xml:id="c02-ent-0130" lwPstyle="TabularHead">R2K</entry><!--</row>--></thead><!--<tbody valign="top">--><!--<row>--><?xmltex \\\tablerule\pgtag{\icolcnt=1\relax}?><entry colname="col1" xml:id="c02-ent-0131"><b>CAGR</b></entry>
<entry colname="col2" xml:id="c02-ent-0132">9.12&percnt;</entry>
<entry colname="col3" xml:id="c02-ent-0133">2.75&percnt;</entry>
<entry colname="col4" xml:id="c02-ent-0134">4.45&percnt;</entry>
<entry colname="col5" xml:id="c02-ent-0135">7.38&percnt;</entry><!--</row>-->
<!--<row>--><?xmltex \\\pgtag{\icolcnt=1\relax}?><entry colname="col1" xml:id="c02-ent-0136"><b>Standard Deviation</b></entry>
<entry colname="col2" xml:id="c02-ent-0137">24.05&percnt;</entry>
<entry colname="col3" xml:id="c02-ent-0138">16.90&percnt;</entry>
<entry colname="col4" xml:id="c02-ent-0139">15.22&percnt;</entry>
<entry colname="col5" xml:id="c02-ent-0140">20.42&percnt;</entry><!--</row>-->
<!--<row>--><?xmltex \\\pgtag{\icolcnt=1\relax}?><entry colname="col1" xml:id="c02-ent-0141"><b>Downside Deviation</b></entry>
<entry colname="col2" xml:id="c02-ent-0142">17.73&percnt;</entry>
<entry colname="col3" xml:id="c02-ent-0143">12.50&percnt;</entry>
<entry colname="col4" xml:id="c02-ent-0144">11.42&percnt;</entry>
<entry colname="col5" xml:id="c02-ent-0145">13.77&percnt;</entry><!--</row>-->
<!--<row>--><?xmltex \\\pgtag{\icolcnt=1\relax}?><entry colname="col1" xml:id="c02-ent-0146"><b>Sharpe Ratio</b></entry>
<entry colname="col2" xml:id="c02-ent-0147">0.41</entry>
<entry colname="col3" xml:id="c02-ent-0148">0.14</entry>
<entry colname="col4" xml:id="c02-ent-0149">0.24</entry>
<entry colname="col5" xml:id="c02-ent-0150">0.36</entry><!--</row>-->
<!--<row>--><?xmltex \\\pgtag{\icolcnt=1\relax}?><entry colname="col1" xml:id="c02-ent-0151"><b>Sortino Ratio (MAR</b> <math display="inline" overflow="scroll" xmlns="http://www.w3.org/1998/Math/MathML" xmlns:xlink="http://www.w3.org/1999/xlink"><mrow><mo mathvariant="bold">=</mo></mrow></math> <b>5&percnt;)</b></entry>
<entry colname="col2" xml:id="c02-ent-0152">0.37</entry>
<entry colname="col3" xml:id="c02-ent-0153">&ndash;0.07</entry>
<entry colname="col4" xml:id="c02-ent-0154">0.05</entry>
<entry colname="col5" xml:id="c02-ent-0155">0.31</entry><!--</row>-->
<!--<row>--><?xmltex \\\pgtag{\icolcnt=1\relax}?><entry colname="col1" xml:id="c02-ent-0156"><b>Worst Drawdown</b></entry>
<entry colname="col2" xml:id="c02-ent-0157">&ndash;64.47&percnt;</entry>
<entry colname="col3" xml:id="c02-ent-0158">&ndash;58.21&percnt;</entry>
<entry colname="col4" xml:id="c02-ent-0159">&ndash;50.21&percnt;</entry>
<entry colname="col5" xml:id="c02-ent-0160">&ndash;52.89&percnt;</entry><!--</row>-->
<!--<row>--><?xmltex \\\pgtag{\icolcnt=1\relax}?><entry colname="col1" xml:id="c02-ent-0161"><b>Worst Month Return</b></entry>
<entry colname="col2" xml:id="c02-ent-0162">&ndash;28.07&percnt;</entry>
<entry colname="col3" xml:id="c02-ent-0163">&ndash;16.13&percnt;</entry>
<entry colname="col4" xml:id="c02-ent-0164">&ndash;16.70&percnt;</entry>
<entry colname="col5" xml:id="c02-ent-0165">&ndash;20.80&percnt;</entry><!--</row>-->
<!--<row>--><?xmltex \\\pgtag{\icolcnt=1\relax}?><entry colname="col1" xml:id="c02-ent-0166"><b>Best Month Return</b></entry>
<entry colname="col2" xml:id="c02-ent-0167">36.64&percnt;</entry>
<entry colname="col3" xml:id="c02-ent-0168">11.21&percnt;</entry>
<entry colname="col4" xml:id="c02-ent-0169">10.93&percnt;</entry>
<entry colname="col5" xml:id="c02-ent-0170">16.51&percnt;</entry><!--</row>-->
<!--<row>--><?xmltex \\\pgtag{\icolcnt=1\relax}?><entry colname="col1" xml:id="c02-ent-0171"><b>Profitable Months</b></entry>
<entry colname="col2" xml:id="c02-ent-0172">58.89&percnt;</entry>
<entry colname="col3" xml:id="c02-ent-0173">56.67&percnt;</entry>
<entry colname="col4" xml:id="c02-ent-0174">60.56&percnt;</entry>
<entry colname="col5" xml:id="c02-ent-0175">58.89&percnt;</entry><!--</row>-->
<?xmltex \pgtag{\\ \lasttablerule\end{tabular*}}?><!--</tbody>-->
</lwtablebody></tgroup>
</table>
</tabular><?xmltex \pgtag{\egroup}?>
<p xml:id="c02-para-0092">Sticking with the value strategy, although painful, was richly rewarded with almost a 5 percent edge&mdash;per year&mdash;over the market benchmark (S&amp;P 500) from 2000 to 2014.</p>
<p xml:id="c02-para-0093">Over the entire cycle, patient and disciplined investors were rewarded. Table<?xmltex \pgtag{\nobreak}?> <link href="c02-tbl-0005"/> shows the results over the entire time period, measured from<?xmltex \pgtag{\break}?> January 1, 1994, to December 31, 2014.</p>
<?xmltex \OrgFixedPosition{c02-tbl-0005}?>
<?xmltex \pgtag{\bgroup\tabbotskip=-3pt\FloatPositionToptrue}?><tabular xml:id="c02-tbl-0005"><title type="main">Summary Statistics (1994&ndash;2014)</title><table pgwide="1" frame="topbot" rowsep="0" colsep="0"><tgroup cols="5"><colspec colnum="1" colname="col1" align="left"/><colspec colnum="2" colname="col2" align="center"/><colspec colnum="3" colname="col3" align="center"/><colspec colnum="4" colname="col4" align="center"/><colspec colnum="5" colname="col5" align="center"/><lwtablebody><?xmltex \pgtag{\tabcolsep=0pt\begin{tabular*}{\textwidth}{@{\extracolsep\fill}ld{4}d{4}d{4}d{4}@{\extracolsep\fill}}\firsttablerule}?><thead valign="bottom"><!--<row rowsep="1">--><?xmltex \pgtag{\icolcnt=1\relax}?><entry colname="col1" align="center" xml:id="c02-ent-0176"></entry><entry colname="col2" xml:id="c02-ent-0177" align="center" lwPstyle="TabularHead">Value</entry><entry colname="col3" align="center" xml:id="c02-ent-0178" lwPstyle="TabularHead">Growth</entry><entry colname="col4" xml:id="c02-ent-0179" align="center" lwPstyle="TabularHead">SP500</entry><entry colname="col5" align="center" xml:id="c02-ent-0180" lwPstyle="TabularHead">R2K</entry><!--</row>--></thead><!--<tbody valign="top">--><!--<row>--><?xmltex \\\tablerule\pgtag{\icolcnt=1\relax}?><entry colname="col1" xml:id="c02-ent-0181"><b>CAGR</b></entry>
<entry colname="col2" xml:id="c02-ent-0182">11.68&percnt;</entry>
<entry colname="col3" xml:id="c02-ent-0183">9.33&percnt;</entry>
<entry colname="col4" xml:id="c02-ent-0184">9.65&percnt;</entry>
<entry colname="col5" xml:id="c02-ent-0185">9.06&percnt;</entry><!--</row>-->
<!--<row>--><?xmltex \\\pgtag{\icolcnt=1\relax}?><entry colname="col1" xml:id="c02-ent-0186"><b>Standard Deviation</b></entry>
<entry colname="col2" xml:id="c02-ent-0187">21.27&percnt;</entry>
<entry colname="col3" xml:id="c02-ent-0188">17.00&percnt;</entry>
<entry colname="col4" xml:id="c02-ent-0189">14.92&percnt;</entry>
<entry colname="col5" xml:id="c02-ent-0190">19.48&percnt;</entry><!--</row>-->
<!--<row>--><?xmltex \\\pgtag{\icolcnt=1\relax}?><entry colname="col1" xml:id="c02-ent-0191"><b>Downside Deviation</b></entry>
<entry colname="col2" xml:id="c02-ent-0192">16.23&percnt;</entry>
<entry colname="col3" xml:id="c02-ent-0193">12.25&percnt;</entry>
<entry colname="col4" xml:id="c02-ent-0194">11.19&percnt;</entry>
<entry colname="col5" xml:id="c02-ent-0195">13.97&percnt;</entry><!--</row>-->
<!--<row>--><?xmltex \\\pgtag{\icolcnt=1\relax}?><entry colname="col1" xml:id="c02-ent-0196"><b>Sharpe Ratio</b></entry>
<entry colname="col2" xml:id="c02-ent-0197">0.50</entry>
<entry colname="col3" xml:id="c02-ent-0198">0.45</entry>
<entry colname="col4" xml:id="c02-ent-0199">0.51</entry>
<entry colname="col5" xml:id="c02-ent-0200">0.41</entry><!--</row>-->
<!--<row>--><?xmltex \\\pgtag{\icolcnt=1\relax}?><entry colname="col1" xml:id="c02-ent-0201"><b>Sortino Ratio (MAR</b> <math display="inline" overflow="scroll" xmlns="http://www.w3.org/1998/Math/MathML" xmlns:xlink="http://www.w3.org/1999/xlink"><mrow><mo mathvariant="bold">=</mo></mrow></math> <b>5&percnt;)</b></entry>
<entry colname="col2" xml:id="c02-ent-0202">0.51</entry>
<entry colname="col3" xml:id="c02-ent-0203">0.44</entry>
<entry colname="col4" xml:id="c02-ent-0204">0.48</entry>
<entry colname="col5" xml:id="c02-ent-0205">0.40</entry><!--</row>-->
<!--<row>--><?xmltex \\\pgtag{\icolcnt=1\relax}?><entry colname="col1" xml:id="c02-ent-0206"><b>Worst Drawdown</b></entry>
<entry colname="col2" xml:id="c02-ent-0207">&ndash;64.47&percnt;</entry>
<entry colname="col3" xml:id="c02-ent-0208">&ndash;58.21&percnt;</entry>
<entry colname="col4" xml:id="c02-ent-0209">&ndash;50.21&percnt;</entry>
<entry colname="col5" xml:id="c02-ent-0210">&ndash;52.89&percnt;</entry><!--</row>-->
<!--<row>--><?xmltex \\\pgtag{\icolcnt=1\relax}?><entry colname="col1" xml:id="c02-ent-0211"><b>Worst Month Return</b></entry>
<entry colname="col2" xml:id="c02-ent-0212">&ndash;28.07&percnt;</entry>
<entry colname="col3" xml:id="c02-ent-0213">&ndash;16.13&percnt;</entry>
<entry colname="col4" xml:id="c02-ent-0214">&ndash;16.70&percnt;</entry>
<entry colname="col5" xml:id="c02-ent-0215">&ndash;20.80&percnt;</entry><!--</row>-->
<!--<row>--><?xmltex \\\pgtag{\icolcnt=1\relax}?><entry colname="col1" xml:id="c02-ent-0216"><b>Best Month Return</b></entry>
<entry colname="col2" xml:id="c02-ent-0217">36.64&percnt;</entry>
<entry colname="col3" xml:id="c02-ent-0218">11.21&percnt;</entry>
<entry colname="col4" xml:id="c02-ent-0219">10.93&percnt;</entry>
<entry colname="col5" xml:id="c02-ent-0220">16.51&percnt;</entry><!--</row>-->
<!--<row>--><?xmltex \\\pgtag{\icolcnt=1\relax}?><entry colname="col1" xml:id="c02-ent-0221"><b>Profitable Months</b></entry>
<entry colname="col2" xml:id="c02-ent-0222">61.51&percnt;</entry>
<entry colname="col3" xml:id="c02-ent-0223">60.71&percnt;</entry>
<entry colname="col4" xml:id="c02-ent-0224">64.29&percnt;</entry>
<entry colname="col5" xml:id="c02-ent-0225">61.11&percnt;</entry><!--</row>-->
<?xmltex \pgtag{\\ \lasttablerule\end{tabular*}}?><!--</tbody>-->
</lwtablebody></tgroup>
</table>
</tabular><?xmltex \pgtag{\egroup}?>
<p xml:id="c02-para-0094">What&apos;s the bottom line? For a long&hyphen;term investor, value investing was the optimal decision relative to growth investing, but for many of the smartest asset managers in the world, including the great Julian Robertson, value investing was simply not feasible as a business model. These professionals were often forced via the threat of investor redemptions to &ldquo;diworsify&rdquo; their portfolios with overpriced growth stocks during the Internet Bubble. They needed to keep up with the market and did so by doing what everyone else was doing. This decision helped them keep their jobs, but prevented their investors from maximizing their chances for success, even if some truly did have a long&hyphen;horizon, and discipline.</p></section>
</section>
<section xml:id="c02-sec-0014"><title type="main">Putting It All Together</title><p xml:id="c02-para-0095"><?xmltex \pgtag{\changespaceskip{2.4}}?>We&apos;ve used value and growth investing as a laboratory to highlight how the sustainable active investing framework can identify long&hyphen;term winning strategies. Value investing fits nicely in this paradigm, but has serious warts, notably stretches of horrendous underperformance. The lesson from value investing is that successful active investing is simple, but not easy. If active investing were easy, everyone would do it, and if everyone were doing it, it probably would not generate outsized risk&hyphen;adjusted returns over the long<?xmltex \pgtag{\nobreak}?> <?xmltex \pgtag{\hbox\bgroup}?>haul.<?xmltex \pgtag{\egroup}?></p>
<p xml:id="c02-para-0096">In summary, our long&hyphen;term performance equation from Figure<?xmltex \pgtag{\nobreak}?> <link href="c02-fig-0004"/> highlights two required elements for sustainable performance:
<?xmltex \pgtag{\def\itemwd{2.}}?>
<list xml:id="c02-list-0013" style="1"><listItem xml:id="c02-li-0036">The sustainable process exploits systematic investor expectation errors.</listItem>
<listItem xml:id="c02-li-0037">The sustainable investor has a long horizon and a willingness to be<?xmltex \pgtag{\break}?> different.</listItem>
</list>
</p>
<p xml:id="c02-para-0097">These two pieces of the puzzle map back to the classic lessons of poker:
<?xmltex \pgtag{\def\itemwd{2.}}?>
<list xml:id="c02-list-0014" style="1"><listItem xml:id="c02-li-0038">Identify the worst poker player at the table.</listItem>
<listItem xml:id="c02-li-0039">Identify the best poker players at the table.</listItem>
</list>
</p>
<p xml:id="c02-para-0098">And these classic lessons map into the two pillars of behavioral finance:
<?xmltex \pgtag{\def\itemwd{2.}}?>
<list xml:id="c02-list-0015" style="1"><listItem xml:id="c02-li-0040">Understand behavioral bias and how investors form expectations.</listItem>
<listItem xml:id="c02-li-0041">Understand market frictions and how they affect market participants.</listItem>
</list>
</p>
<p xml:id="c02-para-0099">So the next time you hear a market participant suggest that one strategy is better than another strategy, simply ask two basic questions: (1) Why are the securities selected by this process mispriced? and (2) Why aren&apos;t other smart investors already exploiting the mispricing opportunity? Without solid answers to both questions, it is unlikely that the investment process<?xmltex \pgtag{\break}?> is<?xmltex \pgtag{\nb}?> sustainable.</p></section>
</section>
<section xml:id="c02-sec-0015"><title type="main">Growth Investing Stinks, So Why Do It?</title><p xml:id="c02-para-0100">We talked in the last section about how value stocks outperform growth stocks, and showed that buying and holding growth stocks is a bad relative bet. And yet, most fund complexes divide the investable universe into value and growth stocks. To highlight the prevalence of the value/growth mindset in the marketplace, Figure<?xmltex \pgtag{\nobreak}?> <link href="c02-fig-0008"/> is an example of the classic three&hyphen;by&hyphen;three diagram, which splits the stock universe into nine buckets. The two axes are size (vertical axis from large to small) and value (horizontal axis from value to growth).</p>
<?xmltex \OrgFixedPosition{c02-fig-0008}?>
<figure xml:id="c02-fig-0008">
<mediaResource href="urn:x-wiley:9781119237198:media:w9781119237198c02:c02f008" alt="image"/>
<caption>Value and Growth Chart</caption>
<?xmltex \pgtag{\bgroup\FloatPositionBottrue\putfigure{8}{c02/c02f008.eps}{}{}{}\egroup}?></figure>
<?xmltex \OrgFixedPosition{c02-tbl-0006}?>
<?xmltex \pgtag{\bgroup\tabbotskip=-3pt\FloatPositionBottrue}?><tabular xml:id="c02-tbl-0006"><title type="main">Combining Value and Growth Lowers Volatility (1994&ndash;1999)</title><table pgwide="1" frame="topbot" rowsep="0" colsep="0"><tgroup cols="5"><colspec colnum="1" colname="col1" align="left"/><colspec colnum="2" colname="col2" align="center"/><colspec colnum="3" colname="col3" align="center"/><colspec colnum="4" colname="col4" align="center"/><colspec colnum="5" colname="col5" align="center"/><lwtablebody><?xmltex \pgtag{\tabcolsep=0pt\begin{tabular*}{\textwidth}{@{\extracolsep\fill}ld{4}d{4}d{6}d{4}@{\extracolsep\fill}}\firsttablerule}?><thead valign="bottom"><!--<row rowsep="1">--><?xmltex \pgtag{\icolcnt=1\relax}?><entry colname="col1" align="center" xml:id="c02-ent-0226"></entry><entry colname="col2" align="center" xml:id="c02-ent-0227" lwPstyle="TabularHead">Value</entry><entry colname="col3" align="center" xml:id="c02-ent-0228" lwPstyle="TabularHead">Growth</entry><entry colname="col4" align="center" xml:id="c02-ent-0229" lwPstyle="TabularHead">50&percnt; Value,<?xmltex \pgtag{\\}?> 50&percnt; Growth</entry><entry colname="col5" align="center" xml:id="c02-ent-0230" lwPstyle="TabularHead">SP500</entry><!--</row>--></thead><!--<tbody valign="top">--><!--<row>--><?xmltex \\\tablerule\pgtag{\icolcnt=1\relax}?><entry colname="col1" xml:id="c02-ent-0231"><b>CAGR</b></entry>
<entry colname="col2" xml:id="c02-ent-0232">18.35&percnt;</entry>
<entry colname="col3" xml:id="c02-ent-0233">27.71&percnt;</entry>
<entry colname="col4" xml:id="c02-ent-0234">23.19&percnt;</entry>
<entry colname="col5" xml:id="c02-ent-0235">23.84&percnt;</entry><!--</row>-->
<!--<row>--><?xmltex \\\pgtag{\icolcnt=1\relax}?><entry colname="col1" xml:id="c02-ent-0236"><b>Standard Deviation</b></entry>
<entry colname="col2" xml:id="c02-ent-0237">11.79&percnt;</entry>
<entry colname="col3" xml:id="c02-ent-0238">16.53&percnt;</entry>
<entry colname="col4" xml:id="c02-ent-0239">12.86&percnt;</entry>
<entry colname="col5" xml:id="c02-ent-0240">13.63&percnt;</entry><!--</row>-->
<!--<row>--><?xmltex \\\pgtag{\icolcnt=1\relax}?><entry colname="col1" xml:id="c02-ent-0241"><b>Downside Deviation</b></entry>
<entry colname="col2" xml:id="c02-ent-0242">7.59&percnt;</entry>
<entry colname="col3" xml:id="c02-ent-0243">11.25&percnt;</entry>
<entry colname="col4" xml:id="c02-ent-0244">9.49&percnt;</entry>
<entry colname="col5" xml:id="c02-ent-0245">10.50&percnt;</entry><!--</row>-->
<!--<row>--><?xmltex \\\pgtag{\icolcnt=1\relax}?><entry colname="col1" xml:id="c02-ent-0246"><b>Sharpe Ratio</b></entry>
<entry colname="col2" xml:id="c02-ent-0247">1.09</entry>
<entry colname="col3" xml:id="c02-ent-0248">1.28</entry>
<entry colname="col4" xml:id="c02-ent-0249">1.32</entry>
<entry colname="col5" xml:id="c02-ent-0250">1.30</entry><!--</row>-->
<!--<row>--><?xmltex \\\pgtag{\icolcnt=1\relax}?><entry colname="col1" xml:id="c02-ent-0251"><b>Sortino Ratio (MAR</b> <math display="inline" overflow="scroll" xmlns="http://www.w3.org/1998/Math/MathML" xmlns:xlink="http://www.w3.org/1999/xlink"><mrow><mo mathvariant="bold">=</mo></mrow></math> <b>5&percnt;)</b></entry>
<entry colname="col2" xml:id="c02-ent-0252">1.66</entry>
<entry colname="col3" xml:id="c02-ent-0253">1.87</entry>
<entry colname="col4" xml:id="c02-ent-0254">1.78</entry>
<entry colname="col5" xml:id="c02-ent-0255">1.67</entry><!--</row>-->
<!--<row>--><?xmltex \\\pgtag{\icolcnt=1\relax}?><entry colname="col1" xml:id="c02-ent-0256"><b>Worst Drawdown</b></entry>
<entry colname="col2" xml:id="c02-ent-0257">&ndash;11.58&percnt;</entry>
<entry colname="col3" xml:id="c02-ent-0258">&ndash;16.33&percnt;</entry>
<entry colname="col4" xml:id="c02-ent-0259">&ndash;13.93&percnt;</entry>
<entry colname="col5" xml:id="c02-ent-0260">&ndash;15.18&percnt;</entry><!--</row>-->
<!--<row>--><?xmltex \\\pgtag{\icolcnt=1\relax}?><entry colname="col1" xml:id="c02-ent-0261"><b>Worst Month Return</b></entry>
<entry colname="col2" xml:id="c02-ent-0262">&ndash;8.62&percnt;</entry>
<entry colname="col3" xml:id="c02-ent-0263">&ndash;14.92&percnt;</entry>
<entry colname="col4" xml:id="c02-ent-0264">&ndash;11.77&percnt;</entry>
<entry colname="col5" xml:id="c02-ent-0265">&ndash;14.31&percnt;</entry><!--</row>-->
<!--<row>--><?xmltex \\\pgtag{\icolcnt=1\relax}?><entry colname="col1" xml:id="c02-ent-0266"><b>Best Month Return</b></entry>
<entry colname="col2" xml:id="c02-ent-0267">8.05&percnt;</entry>
<entry colname="col3" xml:id="c02-ent-0268">10.69&percnt;</entry>
<entry colname="col4" xml:id="c02-ent-0269">7.97&percnt;</entry>
<entry colname="col5" xml:id="c02-ent-0270">8.04&percnt;</entry><!--</row>-->
<!--<row>--><?xmltex \\\pgtag{\icolcnt=1\relax}?><entry colname="col1" xml:id="c02-ent-0271"><b>Profitable Months</b></entry>
<entry colname="col2" xml:id="c02-ent-0272">68.06&percnt;</entry>
<entry colname="col3" xml:id="c02-ent-0273">70.83&percnt;</entry>
<entry colname="col4" xml:id="c02-ent-0274">70.83&percnt;</entry>
<entry colname="col5" xml:id="c02-ent-0275">73.61&percnt;</entry><!--</row>-->
<?xmltex \pgtag{\\ \lasttablerule\end{tabular*}}?><!--</tbody>-->
</lwtablebody></tgroup>
</table>
</tabular><?xmltex \pgtag{\egroup}?>
<p xml:id="c02-para-0101">Figure<?xmltex \pgtag{\nobreak}?> <link href="c02-fig-0008"/>, or some derivation of it, is used by almost every major investment firm in the United States. But if growth is a suboptimal investment approach, why bother with a framework that suggests we consider growth stocks as part of a portfolio? One answer to this question is likely related to the fact that growth stocks provide some diversification benefits for a portfolio, even though they provide poor relative returns. We explicitly investigate the diversification benefits of growth in Table<?xmltex \pgtag{\nobreak}?> <link href="c02-tbl-0006"/>, in the context of the late 1990s (a period we examined in Table<?xmltex \pgtag{\nobreak}?> <link href="c02-tbl-0002"/>), which was a time when value underperformed versus growth. We examine the performance of a monthly rebalanced portfolio that invests half of the portfolio in the value portfolio and half in the growth portfolio over the 1994 to 1999 time period.</p>
<p xml:id="c02-para-0102">At a high level, being a combo investor (value and growth) was a much smarter career move over this period in the 1990s than being a pure value investor. The combo investor did not achieve the performance of the pure growth portfolio, but the results were closer to the broader market and the probability of getting fired was muted. The annual return figures in Table<?xmltex \pgtag{\nobreak}?> <link href="c02-tbl-0007"/> bring this point<?xmltex \pgtag{\nobreak}?> <?xmltex \pgtag{\hbox\bgroup}?>home.<?xmltex \pgtag{\egroup}?></p>
<?xmltex \OrgFixedPosition{c02-tbl-0007}?>
<?xmltex \pgtag{\bgroup\tabbotskip=-3pt\FloatPositionBottrue}?><tabular xml:id="c02-tbl-0007"><title type="main">Annual Returns for Combo Portfolio</title><table pgwide="1" frame="topbot" rowsep="0" colsep="0"><tgroup cols="5"><colspec colnum="1" colname="col1" align="center"/><colspec colnum="2" colname="col2" align="center"/><colspec colnum="3" colname="col3" align="center"/><colspec colnum="4" colname="col4" align="center"/><colspec colnum="5" colname="col5" align="center"/><lwtablebody><?xmltex \pgtag{\tabcolsep=0pt\begin{tabular*}{\textwidth}{@{\extracolsep\fill}cd{3.4}d{3.4}d{3.4}d{3.4}@{\extracolsep\fill}}\firsttablerule}?><thead valign="bottom"><!--<row rowsep="1">--><?xmltex \pgtag{\icolcnt=1\relax}?><entry colname="col1" align="left" xml:id="c02-ent-0276"></entry><entry colname="col2" align="center" xml:id="c02-ent-0277" lwPstyle="TabularHead">Value</entry><entry colname="col3" align="center" xml:id="c02-ent-0278" lwPstyle="TabularHead">Growth</entry><entry colname="col4" align="center" xml:id="c02-ent-0279"  lwPstyle="TabularHead">50&percnt; Value,<?xmltex \pgtag{\\}?> 50&percnt; Growth</entry><entry colname="col5" align="center" xml:id="c02-ent-0280" lwPstyle="TabularHead">SP500</entry><!--</row>--></thead><!--<tbody valign="top">--><!--<row>--><?xmltex \\\tablerule\pgtag{\icolcnt=1\relax}?><entry colname="col1" xml:id="c02-ent-0281"><b>1994</b></entry>
<entry colname="col2" xml:id="c02-ent-0282">&ndash;2.83&percnt;</entry>
<entry colname="col3" xml:id="c02-ent-0283">2.53&percnt;</entry>
<entry colname="col4" xml:id="c02-ent-0284">&ndash;0.09&percnt;</entry>
<entry colname="col5" xml:id="c02-ent-0285">1.35&percnt;</entry><!--</row>-->
<!--<row>--><?xmltex \\\pgtag{\icolcnt=1\relax}?><entry colname="col1" xml:id="c02-ent-0286"><b>1995</b></entry>
<entry colname="col2" xml:id="c02-ent-0287">36.47&percnt;</entry>
<entry colname="col3" xml:id="c02-ent-0288">35.47&percnt;</entry>
<entry colname="col4" xml:id="c02-ent-0289">36.07&percnt;</entry>
<entry colname="col5" xml:id="c02-ent-0290">37.64&percnt;</entry><!--</row>-->
<!--<row>--><?xmltex \\\pgtag{\icolcnt=1\relax}?><entry colname="col1" xml:id="c02-ent-0291"><b>1996</b></entry>
<entry colname="col2" xml:id="c02-ent-0292">14.22&percnt;</entry>
<entry colname="col3" xml:id="c02-ent-0293">23.20&percnt;</entry>
<entry colname="col4" xml:id="c02-ent-0294">18.77&percnt;</entry>
<entry colname="col5" xml:id="c02-ent-0295">23.23&percnt;</entry><!--</row>-->
<!--<row>--><?xmltex \\\pgtag{\icolcnt=1\relax}?><entry colname="col1" xml:id="c02-ent-0296"><b>1997</b></entry>
<entry colname="col2" xml:id="c02-ent-0297">32.52&percnt;</entry>
<entry colname="col3" xml:id="c02-ent-0298">31.15&percnt;</entry>
<entry colname="col4" xml:id="c02-ent-0299">32.08&percnt;</entry>
<entry colname="col5" xml:id="c02-ent-0300">33.60&percnt;</entry><!--</row>-->
<!--<row>--><?xmltex \\\pgtag{\icolcnt=1\relax}?><entry colname="col1" xml:id="c02-ent-0301"><b>1998</b></entry>
<entry colname="col2" xml:id="c02-ent-0302">29.75&percnt;</entry>
<entry colname="col3" xml:id="c02-ent-0303">44.23&percnt;</entry>
<entry colname="col4" xml:id="c02-ent-0304">37.15&percnt;</entry>
<entry colname="col5" xml:id="c02-ent-0305">29.32&percnt;</entry><!--</row>-->
<!--<row>--><?xmltex \\\pgtag{\icolcnt=1\relax}?><entry colname="col1" xml:id="c02-ent-0306"><b>1999</b></entry>
<entry colname="col2" xml:id="c02-ent-0307">5.45&percnt;</entry>
<entry colname="col3" xml:id="c02-ent-0308">33.90&percnt;</entry>
<entry colname="col4" xml:id="c02-ent-0309">19.37&percnt;</entry>
<entry colname="col5" xml:id="c02-ent-0310">21.35&percnt;</entry><!--</row>-->
<?xmltex \pgtag{\\ \lasttablerule\end{tabular*}}?><!--</tbody>-->
</lwtablebody></tgroup>
</table>
</tabular><?xmltex \pgtag{\egroup}?>
<p xml:id="c02-para-0103">Unlike the pure value portfolio, which was a guaranteed ticket to the unemployment line in 1999, the combo portfolio, while underwhelming relative to the market, would have been at least a salvageable situation in a client meeting. Of course, we already know how this story ends. The <?xmltex \pgtag{\bgroup\mbox}?>benefits<?xmltex \pgtag{\egroup}?> of combining the growth strategy with the value strategy over this unique time period offered a great benefit: diversification. The combo reduced the pain versus a pure value approach.</p>
<p xml:id="c02-para-0104">Likewise, as seen in Table<?xmltex \pgtag{\nobreak}?> <link href="c02-tbl-0008"/>, the combo portfolio served an investment manager well over the longer 1994 to 2014 period (examined previously in Table<?xmltex \pgtag{\nobreak}?> <link href="c02-tbl-0005"/>), delivering higher risk&hyphen;adjusted returns than the S&amp;P 500 benchmark. For the period 1994 to 2014, the combo portfolio reduced the pain versus a pure growth approach.</p>
<?xmltex \OrgFixedPosition{c02-tbl-0008}?>
<?xmltex \pgtag{\bgroup\FloatPositionBottrue}?><tabular xml:id="c02-tbl-0008"><title type="main">Combining Value and Growth Lowers Volatility (1994&ndash;2014)</title><table pgwide="1" frame="topbot" rowsep="0" colsep="0"><tgroup cols="5"><colspec colnum="1" colname="col1" align="left"/><colspec colnum="2" colname="col2" align="center"/><colspec colnum="3" colname="col3" align="center"/><colspec colnum="4" colname="col4" align="center"/><colspec colnum="5" colname="col5" align="center"/><lwtablebody><?xmltex \pgtag{\tabcolsep=0pt\begin{tabular*}{\textwidth}{@{\extracolsep\fill}ld{3.4}d{3.4}d{3.4}d{3.4}@{\extracolsep\fill}}\firsttablerule}?><thead valign="bottom"><!--<row rowsep="1">--><?xmltex \pgtag{\icolcnt=1\relax}?><entry colname="col1" align="center" xml:id="c02-ent-0311"></entry><entry colname="col2" align="center" xml:id="c02-ent-0312" lwPstyle="TabularHead">Value</entry><entry colname="col3" align="center" xml:id="c02-ent-0313" lwPstyle="TabularHead">Growth</entry><entry colname="col4" align="center" xml:id="c02-ent-0314" lwPstyle="TabularHead">50&percnt; Value,<?xmltex \pgtag{\\}?> 50&percnt; Growth</entry><entry colname="col5" align="center" xml:id="c02-ent-0315" lwPstyle="TabularHead">SP500</entry><!--</row>--></thead><!--<tbody valign="top">--><!--<row>--><?xmltex \\\tablerule\pgtag{\icolcnt=1\relax}?><entry colname="col1" xml:id="c02-ent-0316"><b>CAGR</b></entry>
<entry colname="col2" xml:id="c02-ent-0317">11.68&percnt;</entry>
<entry colname="col3" xml:id="c02-ent-0318">9.33&percnt;</entry>
<entry colname="col4" xml:id="c02-ent-0319">10.86&percnt;</entry>
<entry colname="col5" xml:id="c02-ent-0320">9.65&percnt;</entry><!--</row>-->
<!--<row>--><?xmltex \\\pgtag{\icolcnt=1\relax}?><entry colname="col1" xml:id="c02-ent-0321"><b>Standard Deviation</b></entry>
<entry colname="col2" xml:id="c02-ent-0322">21.27&percnt;</entry>
<entry colname="col3" xml:id="c02-ent-0323">17.00&percnt;</entry>
<entry colname="col4" xml:id="c02-ent-0324">17.42&percnt;</entry>
<entry colname="col5" xml:id="c02-ent-0325">14.92&percnt;</entry><!--</row>-->
<!--<row>--><?xmltex \\\pgtag{\icolcnt=1\relax}?><entry colname="col1" xml:id="c02-ent-0326"><b>Downside Deviation</b></entry>
<entry colname="col2" xml:id="c02-ent-0327">16.23&percnt;</entry>
<entry colname="col3" xml:id="c02-ent-0328">12.25&percnt;</entry>
<entry colname="col4" xml:id="c02-ent-0329">12.87&percnt;</entry>
<entry colname="col5" xml:id="c02-ent-0330">11.19&percnt;</entry><!--</row>-->
<!--<row>--><?xmltex \\\pgtag{\icolcnt=1\relax}?><entry colname="col1" xml:id="c02-ent-0331"><b>Sharpe Ratio</b></entry>
<entry colname="col2" xml:id="c02-ent-0332">0.50</entry>
<entry colname="col3" xml:id="c02-ent-0333">0.45</entry>
<entry colname="col4" xml:id="c02-ent-0334">0.53</entry>
<entry colname="col5" xml:id="c02-ent-0335">0.51</entry><!--</row>-->
<!--<row>--><?xmltex \\\pgtag{\icolcnt=1\relax}?><entry colname="col1" xml:id="c02-ent-0336"><b>Sortino Ratio (MAR</b> <math display="inline" overflow="scroll" xmlns="http://www.w3.org/1998/Math/MathML" xmlns:xlink="http://www.w3.org/1999/xlink"><mrow><mo mathvariant="bold">=</mo></mrow></math> <b>5&percnt;)</b></entry>
<entry colname="col2" xml:id="c02-ent-0337">0.51</entry>
<entry colname="col3" xml:id="c02-ent-0338">0.44</entry>
<entry colname="col4" xml:id="c02-ent-0339">0.53</entry>
<entry colname="col5" xml:id="c02-ent-0340">0.48</entry><!--</row>-->
<!--<row>--><?xmltex \\\pgtag{\icolcnt=1\relax}?><entry colname="col1" xml:id="c02-ent-0341"><b>Worst Drawdown</b></entry>
<entry colname="col2" xml:id="c02-ent-0342">&ndash;64.47&percnt;</entry>
<entry colname="col3" xml:id="c02-ent-0343">&ndash;58.21&percnt;</entry>
<entry colname="col4" xml:id="c02-ent-0344">&ndash;56.63&percnt;</entry>
<entry colname="col5" xml:id="c02-ent-0345">&ndash;50.21&percnt;</entry><!--</row>-->
<!--<row>--><?xmltex \\\pgtag{\icolcnt=1\relax}?><entry colname="col1" xml:id="c02-ent-0346"><b>Worst Month Return</b></entry>
<entry colname="col2" xml:id="c02-ent-0347">&ndash;28.07&percnt;</entry>
<entry colname="col3" xml:id="c02-ent-0348">&ndash;16.13&percnt;</entry>
<entry colname="col4" xml:id="c02-ent-0349">&ndash;22.10&percnt;</entry>
<entry colname="col5" xml:id="c02-ent-0350">&ndash;16.70&percnt;</entry><!--</row>-->
<!--<row>--><?xmltex \\\pgtag{\icolcnt=1\relax}?><entry colname="col1" xml:id="c02-ent-0351"><b>Best Month Return</b></entry>
<entry colname="col2" xml:id="c02-ent-0352">36.64&percnt;</entry>
<entry colname="col3" xml:id="c02-ent-0353">11.21&percnt;</entry>
<entry colname="col4" xml:id="c02-ent-0354">23.28&percnt;</entry>
<entry colname="col5" xml:id="c02-ent-0355">10.93&percnt;</entry><!--</row>-->
<!--<row>--><?xmltex \\\pgtag{\icolcnt=1\relax}?><entry colname="col1" xml:id="c02-ent-0356"><b>Profitable Months</b></entry>
<entry colname="col2" xml:id="c02-ent-0357">61.51&percnt;</entry>
<entry colname="col3" xml:id="c02-ent-0358">60.71&percnt;</entry>
<entry colname="col4" xml:id="c02-ent-0359">62.30&percnt;</entry>
<entry colname="col5" xml:id="c02-ent-0360">64.29&percnt;</entry><!--</row>-->
<?xmltex \pgtag{\\ \lasttablerule\end{tabular*}}?><!--</tbody>-->
</lwtablebody></tgroup>
</table>
</tabular><?xmltex \pgtag{\egroup}?>
<p xml:id="c02-para-0105">The other benefit for these active managers is that they maintained their careers through the tech bubble. Of course, the downside of this approach was lower absolute returns due to the inclusion of the growth component, which diluted the performance of an active value strategy throughout the<?xmltex \pgtag{\nb}?> cycle.</p><section xml:id="c02-sec-0016"><title type="main">But Can We Identify a Better Diversifier?</title><p xml:id="c02-para-0106">As outlined above, investors and professional fund managers appreciate the benefits of including growth in a portfolio&mdash;especially during the period under discussion&mdash;because value and growth had relatively low correlations and thus created a portfolio with less benchmark drift and manageable volatility. However, the inclusion of growth, while providing portfolio diversification benefits, has costs in the form of lower expected portfolio returns. Growth investing is not a sustainable active strategy. In fact, it is just the opposite&mdash;a sustainably poor strategy. But what is an investor to do? Ideally, one could capture the diversification benefits of a growth portfolio, but accomplish the diversification benefits with an active stock selection <?xmltex \pgtag{\bgroup\mbox}?>methodology<?xmltex \pgtag{\egroup}?> that had characteristics that were more in line with the sustainable active framework.</p>
<p xml:id="c02-para-0107">Fortunately, there is a potential solution to this problem: momentum investing. In the early 1990s, academics such as Narasimhan Jegadeesh and Sheridan Titman, in their 1993 paper &ldquo;Returns to Buying Winners and Selling Losers: Implications for Market Efficiency,&rdquo; began to refocus on the old concept of <i>momentum,</i> which refers to a general class of strategies in which past returns can predict future returns.<link href="c02-note-0020"/> That is, if a stock has performed relatively well over the past year, it will continue to perform relatively well in the future. Researchers have done follow&hyphen;on studies that find the momentum effect persists even when controlling for company size and value factors. And the effect appears to hold over a 200&hyphen;year time sample,<link href="c02-note-0021"/> and across multiple asset classes, such as commodities, currencies, and even bonds.<sup>22</sup> Moreover, researchers find that momentum is relatively uncorrelated with value, thus providing diversification benefits. In short, it appears the evidence for momentum is pervasive and provides similar diversification benefits to growth investing.</p>
<p xml:id="c02-para-0108">And while momentum investment strategies are well established in the academic literature, these strategies are not commonly used in actively <?xmltex \pgtag{\bgroup\mbox}?>managed<?xmltex \pgtag{\egroup}?> funds, especially when compared with the large number of &ldquo;growth&rdquo; funds found in the market. In fact, the immediate gut reaction of most people to &ldquo;momentum,&rdquo; is that momentum investing <i>IS</i> growth investing. Unfortunately, this reaction reflects a misconception in the market. Momentum and growth, while sometimes related, are certainly not the same. Moreover, we believe that momentum, unlike growth, fits nicely in the sustainable active framework, thus making it a much better diversifier alongside value, which is another sustainable strategy. The goal of the next chapter is to explain why momentum investing, which is purely focused on prices, is a better alternative to growth investing, which considers both fundamentals and prices. Our mission is to convince the reader that the evidence supports a move to a new style&hyphen;box paradigm (Figure<?xmltex \pgtag{\nobreak}?> <link href="c02-fig-0009"/>) that replaces &ldquo;growth&rdquo; with &ldquo;momentum.&rdquo;</p>
<?xmltex \OrgFixedPosition{c02-fig-0009}?>
<figure xml:id="c02-fig-0009">
<mediaResource href="urn:x-wiley:9781119237198:media:w9781119237198c02:c02f009" alt="image"/>
<caption>New Style Box Paradigm</caption>
<?xmltex \pgtag{\bgroup\FloatPositionBottrue\figbotskip=-3pt\putfigure{9}{c02/c02f009.eps}{}{}{}\egroup}?></figure></section>
</section>
<section type="summary" xml:id="c02-sec-0017"><title type="main">Summary</title><p xml:id="c02-para-0109">In order to assess the sustainability of an active strategy, we outlined the sustainable active investing framework to better understand why certain strategies work and why others do not. We then reviewed the classic value versus growth debate, but viewed this argument through the lens of the sustainable active framework. We discussed that value investing works, not because Ben Graham said it would work, but because (1) it systematically captures a mispricing in the market associated with poor expectations, and (2) taking advantage of the mispricing is difficult.</p>
<p xml:id="c02-para-0110">Next, we addressed the question of why investors would ever rationally invest in growth, given the long&hyphen;term evidence of growth&apos;s historical underperformance. We highlighted a unique time period in the markets&mdash;the Internet Bubble&mdash;where growth investing outperformed value and prevented many professional investment managers from losing their jobs. Next, there was a brief discussion of the benefits (diversification) and the costs (poor long&hyphen;term performance) associated with growth investing. Finally, we ended the chapter by proposing that investors replace growth portfolios with momentum portfolios. The hope is that momentum can provide similar diversification benefits to a value&hyphen;focused portfolio as a growth portfolio does, and that momentum does so without hurting the long&hyphen;term expected performance of the portfolio. In the next chapter, we describe momentum investing, highlight how it is different from growth investing, and then describe why momentum may be a better complement to value than growth.</p></section>
<?xmltex \pgtag{\tablenotecnt=6\def\itemwd{16.}}?><noteGroup xml:id="c02-ntgp-0001"><title type="main">Notes</title><note xml:id="c02-note-0001">Wesley Gray, <i>Embedded: A Marine Corps Advisor Inside the Iraqi Army</i>, Naval Institute Press, Annapolis, 2008.</note>
<note xml:id="c02-note-0002">Eugene F. Fama and Kenneth R. French, &ldquo;The Cross Section of Expected Stock Returns,&rdquo; <i>The Journal of Finance</i> 47 (1992): 427&ndash;465.</note>
<note xml:id="c02-note-0003">Steven S. Crawford, Wesley R. Gray, and Andrew E. Kern, &ldquo;Why Do Fund Managers Identify and Share Profitable Ideas,&rdquo; <i>Journal of Financial and Quantitative Analysis</i>, Forthcoming.</note>
<note xml:id="c02-note-0004">Nick Barberis and Richard Thaler, &ldquo;A Survey of Behavioral Finance,&rdquo; in:<?xmltex \pgtag{\break{}}?> G. M. Constantinides &amp; M. Harris &amp; R. M. Stulz (ed.), <i>Handbook of the Economics of Finance</i>, 1st ed., volume 1, (North Holland: Elsevier, 2003), chapter <exlink href="urn:x-wiley:9781119237198:xml-component:w9781119237198c18"/>, pp.<?xmltex \pgtag{\nobreak}?> 1053&ndash;1128.</note>
<note xml:id="c02-note-0005">Although widely attributed to Keynes, there is little evidence he actually made this statement. See Jason Zweig, &ldquo;Keynes: He Didn&apos;t Say Half of<?xmltex \pgtag{\vadjust{\vfill\eject}}?> What He Said. Or Did He?&rdquo; <i>Wall Street Journal</i> (February 11, 2011), blogs.wsj.com<?xmltex \pgtag{\break}?>/marketbeat/2011/02/11/keynes&hyphen;he&hyphen;didnt&hyphen;say&hyphen;half&hyphen;of&hyphen;what&hyphen;he&hyphen;said&hyphen;or&hyphen;did&hyphen;he/, accessed 2/28/2016.</note>
<note xml:id="c02-note-0006">Daniel Kahneman, <i>Thinking, Fast and Slow</i> (New York: Farrar, Straus and Giroux, 2011).</note>
<note xml:id="c02-note-0007">J. Bradford DeLong, Andrei Shleifer, Lawrence Summers, and Robert Waldmann, &ldquo;Noise Trader Risk in Financial Markets,&rdquo; <i>Journal of Political Economy</i> 98 (1990): 703&ndash;738.</note>
<note xml:id="c02-note-0008">Larry Swedroe, &ldquo;Behavioral Finance Falls Short,&rdquo; April 24, 2015, ETF.com .www.etf.com/sections/index&hyphen;investor&hyphen;corner/swedroe&hyphen;behavioral&hyphen;finance&hyphen;falls<?xmltex \pgtag{\break}?>&hyphen;short?nopaging&equals;1, accessed 2/28/2016.</note>
<note xml:id="c02-note-0009">Barberis and Thaler. Of course, Bill Sharpe&apos;s argument regarding the arithmetic of active management also plays a role in explaining why active managers, as a whole, can&apos;t beat the market.</note>
<note xml:id="c02-note-0010">E. Laise, &ldquo;Best Stock Fund of the Decade: CGM Focus,&rdquo; <i>Wall Street Journal</i> (December 31, 2009), www.wsj.com/articles/SB1000142405274870487680457<?xmltex \pgtag{\break}?>4628561609012716, accessed 12/29/2015.</note>
<note xml:id="c02-note-0011">Some of the spread in returns between dollar&hyphen;weight returns and buy&hyphen;and&hyphen;hold returns could be attributable to the return sequence experienced by the CGM Focus Fund. See Michael Kitces&apos;s October 3, 2012, piece on &ldquo;Does the DALBAR Study Grossly Overstate the Behavior Gap&rdquo; for more information. https://www.kitces.com/blog/does&hyphen;the&hyphen;dalbar&hyphen;study&hyphen;grossly&hyphen;overstate&hyphen;the<?xmltex \pgtag{\hb}?>&hyphen;behavior&hyphen;gap&hyphen;guest&hyphen;post/, accessed 2/28/2016.</note>
<note xml:id="c02-note-0012">Andrei Shleifer and Robert W. Vishny, &ldquo;The Limits of Arbitrage,&rdquo; <i>The Journal of Finance</i> 52 (1997): 35&ndash;55.</note>
<note xml:id="c02-note-0013">Charlie Munger, May 4, 2005, Wesco Financial Annual Meeting. Based on notes from Whitney Tilson. www.tilsonfunds.com/wscmtg05notes.pdf, accessed 2/28/2016.</note>
<note xml:id="c02-note-0014">mba.tuck.dartmouth.edu/pages/faculty/ken.french/data&uscore;library.html, accessed 12/30/2015.</note>
<note xml:id="c02-note-0015">Eugene Fama Interview, 2008, American Finance Association. www.afajof.org<?xmltex \pgtag{\break}?>/details/video/2870921/Eugene&hyphen;Fama&hyphen;Interview.html, accessed 12/29/2015</note>
<note xml:id="c02-note-0016">Josef Lakonishok, Andrei Shleifer, and Robert W. Vishny, &ldquo;Contrarian Investment, Extraploation, and Risk,&rdquo; <i>The Journal of Finance</i> 49(5) (1994): 1541&ndash;1578.</note>
<note xml:id="c02-note-0017">Patricia M. Dechow and Rishard G. Sloan, &ldquo;Returns to Contrarian Investment Strategies: Tests of the Na&iuml;ve Expectations Hypothesis,&rdquo; <i>Journal of Financial Economics</i> 43 (1997): 3&ndash;27. As an aside, Dechow and Sloan 1997 argue that the value anomaly is not driven by naive extrapolation by irrational investors as LSV 1994 suggested, but rather, the outperformance of value stocks is driven by market participants&apos; flawed faith in analysts&apos; forecasts, which are systematically overoptimistic. We also note that the data associated with the valuation metric used in Figure<?xmltex \pgtag{\nobreak}?> <link href="c02-fig-0005"/> is book&hyphen;to&hyphen;market. Other valuation metrics, such as price&hyphen;to&hyphen;earnings, don&apos;t paint a similarly compelling picture, but deeper analysis into alternative valuation metrics still suggests that the value anomaly is partially driven by mispricing caused by investor expectation errors and cannot be fully explained by additional risk.</note>
<note xml:id="c02-note-0018">Markus Brunnermeier and Stefan Nagel, &ldquo;Hedge Funds and the Technology Bubble,&rdquo; <i>The Journal of Finance</i> 59 (1993): 2013&ndash;2040.</note>
<note xml:id="c02-note-0019">Andrew Bary, 1999, &ldquo;What&apos;s Wrong, Warren?&rdquo; <i>Barron&apos;s</i> (1999), www.barrons<?xmltex \pgtag{\break}?>.com/articles/SB945992010127068546, accessed 12/29/2015.</note>
<note xml:id="c02-note-0020">Narasimhan Jegadeesh and Sheridan Titman, &ldquo;Returns to Buying Winners and Selling Losers: Implications for Stock Market Efficiency,&rdquo; <i>The Journal of Finance</i> 48 (1993), 65&ndash;91.</note>
<note xml:id="c02-note-0021">Chris Geczy and Mikhail Samonov, &ldquo;Two Centuries of Price Return Momentum,&rdquo; <i>Financial Analysts Journal</i> (2016).</note>
<note xml:id="c02-note-0022">Clifford S. Asness, Tobias J. Moskowitz, and Lasse H. Pedersen, &ldquo;Value and Momentum Everywhere,&rdquo; <i>The Journal of Finance</i> 68(3) (2013): 929&ndash;985.</note>
</noteGroup>
</body>
</component>
