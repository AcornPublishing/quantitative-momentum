%%%%% Please note that the below listed 2 lines needs to be moved to %%%%
%%%%% a new file 'c09.tml' and the same should be compiled to get the  %%%%
%%%%% typeset pages                                                  %%%%
\def\xmlfile{c09.tml}
\input xmltex
%%%%% END  %%%%%%%%%%%%%%%%%%%%%%%%
<?xml version="1.0" encoding="utf-8"?>
<!DOCTYPE component SYSTEM "file://chgnsm02/macdata/Books/ptg/Books-Documents/Approved-Documents/Guidelines/WileyML3G/WileyML_3G_v2.0/Wileyml3gv20-flat.dtd">
<component version="2.0" xmlns:cms="http://www.wiley.com/namespaces/wiley" xmlns:wiley="http://www.wiley.com/namespaces/wiley/wiley" type="bookChapter" xml:lang="en" xml:id="w9781119237198c09">
<?xmltex \pgtag{\IIIProofVersionInfo{c09}}?>
<?xmltex \pgtag{\setcounter{chapter}{8}\setcounter{page}{144}}?>
<?xmltex \pgtag{\def\Gpath{u:/books/Wiley/Pd/E-line/Reprint/Gray37198/figures/iround}
}?>
<header xml:id="c09-hdr-0001">
<publicationMeta level="product">
<publisherInfo>
<publisherName>John Wiley &amp; Sons, Inc.</publisherName>
<publisherLoc>Hoboken, New Jersey</publisherLoc>
</publisherInfo>
<isbn type="print-13">9781119237198</isbn>
<titleGroup><title type="main" sort="QUANTITATIVE MOMENTUM">Quantitative Momentum</title></titleGroup>
<copyright ownership="publisher">Copyright &copy; 2016 by John Wiley &amp; Sons, Inc. All rights reserved.</copyright>
<numberingGroup>
<numbering type="edition" number="1">1st Edition</numbering>
</numberingGroup>
<creators><creator xml:id="cr-0001" creatorRole="author"><personName><givenNames>Wesley R.</givenNames> <familyName>Gray</familyName></personName></creator></creators>
<subjectInfo>
<subject href="http://psi.wiley.com/subject/ME20">n/a</subject>
</subjectInfo>
</publicationMeta>
<publicationMeta level="unit" position="100" type="chapter">
<idGroup>
<id type="unit" value="c09"/>
<id type="file" value="c09"/>
</idGroup>
<titleGroup><title type="name">Chapter</title></titleGroup>
<eventGroup>
<event type="xmlCreated" agent="SPi Global" date="2016-07-13"/>
</eventGroup>
<numberingGroup>
<numbering type="main">9</numbering>
</numberingGroup>
<objectNameGroup>
<objectName elementName="featureFixed">Extract</objectName>
</objectNameGroup>
</publicationMeta>
<contentMeta>
<titleGroup><title type="main">Making Momentum Work in Practice</title></titleGroup>
</contentMeta>
</header>
<body sectionsNumbered="no" xml:id="c09-body-0001"><section type="opening" xml:id="c09-sec-0001"><p xml:id="c09-para-0001"><?xmltex \OrgFixedPosition{c09-blkfxd-0001}?>
<blockFixed type="standFirst" xml:id="c09-blkfxd-0001"><p xml:id="c09-para-0002">&ldquo;Everyone has a plan until they get punched in the mouth.&rdquo;</p>
<source>&mdash;Attributed to &ldquo;Iron&rdquo; Mike Tyson</source>
</blockFixed></p>
<p xml:id="c09-para-0003"><?xmltex \pgtag{\firstlet}?>In the real world, going all&hyphen;in on quantitative momentum is sure to try the patience of even the most dedicated investor. Almost nobody has the discipline to stick with the program&mdash;including us. We don&apos;t even invest our own capital this way. But we aren&apos;t recommending that investors replace their entire equity portfolio with a high&hyphen;conviction momentum strategy. Momentum is merely a component of a diversified equity portfolio. And as alluded to in Chapter <exlink href="urn:x-wiley:9781119237198:xml-component:w9781119237198c04"/>, momentum portfolios are best used in combination with high&hyphen;conviction value portfolios. The value and momentum combination portfolio shortens stretches of multiyear relative underperformance associated with both stand&hyphen;alone strategies and allows an investor to stick with an equity investment program. Dedicating oneself to pure value investing or to pure momentum investing is akin to sitting on a one&hyphen;legged stool. So why not sit on a stool with multiple legs? Identify a great value investment approach; identify a promising momentum investment approach; and combine the two efforts to serve as your all&hyphen;weather equity portfolio.</p></section>
<section xml:id="c09-sec-0002"><title type="main">A Two&hyphen;Legged Stool: Value &plus; Momentum</title><p xml:id="c09-para-0004">To make the value and momentum combination portfolio more tangible, we examine the approach we use for our own investment capital. We combine the quantitative momentum algorithm outlined in this book with an equally rigorously tested value strategy outlined in Wes&apos;s book on the subject of building systematic value strategies: <i>Quantitative Value</i>.<link href="c09-note-0001"/> Put simply, the quantitative value algorithm seeks to buy cheap, high&hyphen;quality value stocks. Each strategy typically holds around 40 momentum stocks and 40 value stocks, leaving the investor with a high&hyphen;conviction&mdash;but diversified&mdash;portfolio of approximately 80 stocks. One could expand to international markets to increase the portfolio size and enhance diversification, but we shelve that discussion to keep the analysis short and to the<?xmltex \pgtag{\nb}?> point.</p>
<p xml:id="c09-para-0005">To assess performance of our quantitative value and momentum portfolio, we examine a mid&hyphen; to large&hyphen;cap US traded universe and we focus our analysis on the long&hyphen;only portfolios. The portfolios are quarterly rebalanced and <b>equal&hyphen;weighted</b>&mdash;here we deviate from the value&hyphen;weight portfolios shown in Chapter <exlink href="urn:x-wiley:9781119237198:xml-component:w9781119237198c08"/>. We examine the returns from January 1, 1974, to December 31, 2014, which is the time period when the historical data available for the quantitative momentum and quantitative value algorithm overlap. The combination portfolio weights are annually rebalanced on January 1 each year and equally allocated across value and momentum (a more sophisticated investor could volatility&hyphen;weight the exposures). All returns are net of 2 percent in total annual fees, which is a rough estimate of management fees, commissions, and market impact costs associated with rebalancing within and across the strategies.<link href="c09-note-0002"/></p>
<p xml:id="c09-para-0006">The results of the combination portfolio are presented in Table<?xmltex \pgtag{\nobreak}?> <link href="c09-tbl-0001"/>.</p>
<?xmltex \OrgFixedPosition{c09-tbl-0001}?>
<?xmltex \pgtag{\bgroup\FloatPositionBottrue}?><tabular xml:id="c09-tbl-0001"><title type="main">Combining Quantitative Value and Quantitative Momentum</title><table pgwide="1" frame="topbot" rowsep="0" colsep="0"><tgroup cols="5"><colspec colnum="1" colname="col1" align="left"/><colspec colnum="2" colname="col2" align="center"/><colspec colnum="3" colname="col3" align="center"/><colspec colnum="4" colname="col4" align="center"/><colspec colnum="5" colname="col5" align="center"/><lwtablebody><?xmltex \pgtag{\tabcolsep=0pt\begin{tabular*}{\textwidth}{@{\extracolsep\fill}ld{3.4}d{3.4}d{3.4}d{3.4}@{\extracolsep\fill}}\firsttablerule}?><thead valign="bottom"><!--<row rowsep="1">--><?xmltex \pgtag{\icolcnt=1\relax}?><entry colname="col1" align="center" xml:id="c09-ent-0001"></entry><entry colname="col2" align="center" xml:id="c09-ent-0002" lwPstyle="TabularHead">Combination<?xmltex \pgtag{\\}?> Portfolio (Net)</entry><entry colname="col3" align="center" xml:id="c09-ent-0003" lwPstyle="TabularHead">Quantitative<?xmltex \pgtag{\\}?> Momentum<?xmltex \pgtag{\\}?> (Net)</entry><entry colname="col4" align="center" xml:id="c09-ent-0004" lwPstyle="TabularHead">Quantitative<?xmltex \pgtag{\\}?> Value<?xmltex \pgtag{\\}?> (Net)</entry><entry colname="col5" align="center" xml:id="c09-ent-0005" lwPstyle="TabularHead">S&amp;P 500<?xmltex \pgtag{\\}?> TR Index</entry><!--</row>--></thead><!--<tbody valign="top">--><!--<row>--><?xmltex \\\tablerule\pgtag{\icolcnt=1\relax}?><entry colname="col1" xml:id="c09-ent-0006"><b>CAGR</b></entry>
<entry colname="col2"  xml:id="c09-ent-0007">18.10&percnt;</entry>
<entry colname="col3" xml:id="c09-ent-0008">17.38&percnt;</entry>
<entry colname="col4" xml:id="c09-ent-0009">16.98&percnt;</entry>
<entry colname="col5" xml:id="c09-ent-0010">11.16&percnt;</entry><!--</row>-->
<!--<row>--><?xmltex \\\pgtag{\icolcnt=1\relax}?><entry colname="col1" xml:id="c09-ent-0011"><b>Standard Deviation</b></entry>
<entry colname="col2"  xml:id="c09-ent-0012">21.38&percnt;</entry>
<entry colname="col3" xml:id="c09-ent-0013">25.59&percnt;</entry>
<entry colname="col4" xml:id="c09-ent-0014">18.58&percnt;</entry>
<entry colname="col5" xml:id="c09-ent-0015">15.45&percnt;</entry><!--</row>-->
<!--<row>--><?xmltex \\\pgtag{\icolcnt=1\relax}?><entry colname="col1" xml:id="c09-ent-0016"><b>Downside Deviation</b></entry>
<entry colname="col2"  xml:id="c09-ent-0017">14.96&percnt;</entry>
<entry colname="col3" xml:id="c09-ent-0018">18.09&percnt;</entry>
<entry colname="col4" xml:id="c09-ent-0019">12.71&percnt;</entry>
<entry colname="col5" xml:id="c09-ent-0020">11.05&percnt;</entry><!--</row>-->
<!--<row>--><?xmltex \\\pgtag{\icolcnt=1\relax}?><entry colname="col1" xml:id="c09-ent-0021"><b>Sharpe Ratio</b></entry>
<entry colname="col2"  xml:id="c09-ent-0022">0.66</entry>
<entry colname="col3" xml:id="c09-ent-0023">0.57</entry>
<entry colname="col4" xml:id="c09-ent-0024">0.68</entry>
<entry colname="col5" xml:id="c09-ent-0025">0.45</entry><!--</row>-->
<!--<row>--><?xmltex \\\pgtag{\icolcnt=1\relax}?><entry colname="col1" xml:id="c09-ent-0026"><b>Sortino Ratio (MAR</b> <math display="inline" overflow="scroll" xmlns="http://www.w3.org/1998/Math/MathML" xmlns:xlink="http://www.w3.org/1999/xlink"><mrow><mo mathvariant="bold">=</mo></mrow></math> <b>5&percnt;)</b></entry>
<entry colname="col2"  xml:id="c09-ent-0027">0.94</entry>
<entry colname="col3" xml:id="c09-ent-0028">0.80</entry>
<entry colname="col4" xml:id="c09-ent-0029">0.98</entry>
<entry colname="col5" xml:id="c09-ent-0030">0.62</entry><!--</row>-->
<!--<row>--><?xmltex \\\pgtag{\icolcnt=1\relax}?><entry colname="col1" xml:id="c09-ent-0031"><b>Worst Drawdown</b></entry>
<entry colname="col2"  xml:id="c09-ent-0032">&ndash;60.16&percnt;</entry>
<entry colname="col3" xml:id="c09-ent-0033">&ndash;67.72&percnt;</entry>
<entry colname="col4" xml:id="c09-ent-0034">&ndash;51.91&percnt;</entry>
<entry colname="col5" xml:id="c09-ent-0035">&ndash;50.21&percnt;</entry><!--</row>-->
<!--<row>--><?xmltex \\\pgtag{\icolcnt=1\relax}?><entry colname="col1" xml:id="c09-ent-0036"><b>Worst Month Return</b></entry>
<entry colname="col2"  xml:id="c09-ent-0037">&ndash;26.56&percnt;</entry>
<entry colname="col3" xml:id="c09-ent-0038">&ndash;30.33&percnt;</entry>
<entry colname="col4" xml:id="c09-ent-0039">&ndash;25.62&percnt;</entry>
<entry colname="col5" xml:id="c09-ent-0040">&ndash;21.58&percnt;</entry><!--</row>-->
<!--<row>--><?xmltex \\\pgtag{\icolcnt=1\relax}?><entry colname="col1" xml:id="c09-ent-0041"><b>Best Month Return</b></entry>
<entry colname="col2"  xml:id="c09-ent-0042">28.69&percnt;</entry>
<entry colname="col3" xml:id="c09-ent-0043">34.67&percnt;</entry>
<entry colname="col4" xml:id="c09-ent-0044">25.36&percnt;</entry>
<entry colname="col5" xml:id="c09-ent-0045">16.81&percnt;</entry><!--</row>-->
<!--<row>--><?xmltex \\\pgtag{\icolcnt=1\relax}?><entry colname="col1" xml:id="c09-ent-0046"><b>Profitable Months</b></entry>
<entry colname="col2"  xml:id="c09-ent-0047">61.18&percnt;</entry>
<entry colname="col3" xml:id="c09-ent-0048">61.79&percnt;</entry>
<entry colname="col4" xml:id="c09-ent-0049">62.60&percnt;</entry>
<entry colname="col5" xml:id="c09-ent-0050">61.59&percnt;</entry><!--</row>-->
<?xmltex \pgtag{\\ \lasttablerule\end{tabular*}}?><!--</tbody>-->
</lwtablebody></tgroup>
</table>
</tabular><?xmltex \pgtag{\egroup}?>
<p xml:id="c09-para-0013">The combination portfolio has higher returns than either the stand&hyphen;alone value or momentum portfolios. On a risk&hyphen;adjusted basis, the combination portfolio is essentially equivalent to the quantitative value strategy. However, the summary statistics do not capture the <i>survivability</i> of a strategy. To assess survivability, which we loosely define as the degree to which an investor could hold onto a portfolio without &ldquo;giving up,&rdquo; we review the rolling five&hyphen;year CAGRs relative to the passive S&amp;P 500 total return index. This analysis gives us a sense for how holding value and momentum can minimize the frequency of long periods of underperformance associated with stand&hyphen;alone value or momentum.</p>
<p xml:id="c09-para-0014">Figure<?xmltex \pgtag{\nobreak}?> <link href="c09-fig-0001"/> highlights the benefit of combining value and momentum to minimize the length and depth of five&hyphen;year relative underperformance periods. For example, quantitative value endures a deep and extended period of poor relative performance in the late 1990s during the Internet bubble. On the flip side, post&ndash;financial crisis, quantitative momentum has had a long bout of severe underperformance. To be clear, quantitative momentum, on a standalone basis, had a period of underperforming by about 15 percent on a CAGR basis over five years (occurs in June 2013, so the 2008&ndash;2009 financial crisis is in this five&hyphen;year previous period). Imagine having that conversation with your clients!</p>
<?xmltex \OrgFixedPosition{c09-fig-0001}?>
<figure xml:id="c09-fig-0001">
<mediaResource href="urn:x-wiley:9781119237198:media:w9781119237198c09:c09f001" alt="image"/>
<caption>Rolling Five&hyphen;Year Spreads</caption>
<?xmltex \pgtag{\bgroup\FloatPositionPagetrue\putfigure{1}{c09/c09f001.eps}{}{}{}\egroup}?></figure>
<p xml:id="c09-para-0015">However, by combining the two strategies (represented by the solid black line in Figure<?xmltex \pgtag{\nobreak}?> <link href="c09-fig-0001"/>), an investor is able to shorten the length and depth of long&hyphen;term underperformance to a level that is more digestible to the average investor. Another way to look at this problem is via a histogram analysis. Figure<?xmltex \pgtag{\nobreak}?> <link href="c09-fig-0002"/> shows the histogram of five&hyphen;year relative performance measured by CAGR for the pure momentum strategy and the combination portfolio. There is a relatively frequent probability of losing to the index over a five&hyphen;year window when invested in a pure momentum strategy; however, the combination portfolio substantially limits the chance for a long&hyphen;winded underperformance streak.</p>
<?xmltex \OrgFixedPosition{c09-fig-0002}?>
<figure xml:id="c09-fig-0002">
<mediaResource href="urn:x-wiley:9781119237198:media:w9781119237198c09:c09f002" alt="image"/>
<caption>Histogram of Five&hyphen;Year Spreads</caption>
<?xmltex \pgtag{\bgroup\FloatPositionToptrue\putfigure{2}{c09/c09f002.eps}{}{}{}\egroup}?></figure>
<p xml:id="c09-para-0016">For the long&hyphen;horizon investor, replacing a passive equity portfolio with a high&hyphen;conviction value and momentum system seems like a reasonable approach that can deliver strong expected returns relative to a passive index. We leave the reader with an easy to remember rule of thumb:<?xmltex \OrgFixedPosition{c09-feafxd-0001}?>
<featureFixed xml:id="c09-feafxd-0001" lwtype="Extract"><title type="featureFixedName">Extract</title><p xml:id="c09-para-0017">Buy &apos;em cheap; buy &apos;em strong; and hold &apos;em<?xmltex \pgtag{\nobreak}?> <?xmltex \pgtag{\hbox\bgroup}?>long.<?xmltex \pgtag{\egroup}?></p>
</featureFixed></p><section xml:id="c09-sec-0003"><title type="main">An Important Note on<?xmltex \pgtag{\protect\nobreak}?> Portfolio Construction</title><p xml:id="c09-para-0018">The road to success with active value and momentum will obviously be hair&hyphen;raising, primarily because the possibility of poor long&hyphen;term relative performance prevents large pools of capital from exploiting the opportunity. With that truth in hand, we must emphasize that the expected benefits outlined are associated with <i>high&hyphen;conviction</i> value and momentum portfolios, because these high&hyphen;conviction portfolios drive the relative performance risk. And if there is no extreme relative performance pain, there is no extreme expected performance gain. So&hyphen;called &ldquo;smart beta&rdquo; funds, which hold large diversified portfolios that tilt towards a characteristic like value or momentum, are unlikely to deliver on their promise to achieve outperformance after fees. These funds are nothing more than closet indexing structures that don&apos;t deliver enough active exposure benefits to outweigh their expected costs.</p>
<p xml:id="c09-para-0019"><?xmltex \pgtag{\looseness=-1}?>But why avoid closet&hyphen;indexing? Recall that the academic research and internal analysis we&apos;ve conducted throughout this book are associated with portfolios that are concentrated on stocks with a desirable characteristic (e.g., high momentum). The portfolios we analyze are typically designed to hold less than 50 stocks to minimize &ldquo;diworsification,&rdquo; which occurs when a portfolio is constructed to behave more like a passive index and less like a concentrated characteristic&hyphen;centric portfolio. We highlighted the negative effects of diworsification in Chapter <exlink href="urn:x-wiley:9781119237198:xml-component:w9781119237198c05"/> when we examined how portfolio construction parameters, such as the number of holdings and rebalance frequency, affect expected performance. The results from that analysis were clear for those who wanted to capture the expected returns associated with active momentum strategies: buy concentrated frequently rebalanced portfolios.</p>
<p xml:id="c09-para-0020">So why don&apos;t we see more truly active funds in the market? Unfortunately, the interests of fund sponsors are not aligned with fund investors. Above a certain fund size, additional fund assets erode performance as portfolios move towards closet&hyphen;indexing formations, but also grow manager fees. This creates a conflict of interest between investors, who want to maximize performance, and managers, who just want more assets, <i>even when this hurts their performance</i>. Closet&hyphen;indexers are easy to spot&mdash;their portfolios <?xmltex \pgtag{\bgroup\mbox}?>typically<?xmltex \pgtag{\egroup}?> have over 100 holdings, have market&hyphen;cap weighted construction, and have low frequency rebalancing. These portfolios constructs accommodate scale and facilitate asset collection efforts on behalf of the fund sponsor, but they are unlikely to deliver the higher expected returns documented throughout this book. The implicatons for active investors are clear: If one is going to deviate from a passive index, and pay extra management fees, embrace active risk and pay up for concentration, not closet&hyphen;indexing.</p></section>
</section>
<section xml:id="c09-sec-0004"><title type="main">A Three&hyphen;Legged Stool: Combo &plus; Trend</title><p xml:id="c09-para-0021">But wait a minute: Even a two&hyphen;legged stool isn&apos;t completely stable! The quantitative value and momentum portfolio still suffer large drawdowns that go hand&hyphen;in&hyphen;hand with buy&hyphen;and&hyphen;hold equity investments. For many investors, with a long&hyphen;horizon and a preference for simplicity, holding the combination value and momentum portfolio is a great equity solution. But for those investors concerned about massive drawdowns, a buy&hyphen;and&hyphen;hold value and momentum approach may not be appropriate. And to be clear, the large drawdowns identified in the value and momentum approach outlined above are not unique to this particular portfolio&mdash;the drawdown issue is associated with <i>all</i> long&hyphen;only stock portfolios.</p>
<p xml:id="c09-para-0022">To address the drawdown issue we discuss a basic way in which an investor can create a more stable stool via a third leg&mdash;trend following. The simplest trend&hyphen;following rule is the long&hyphen;term simple moving average rule. To give the reader a taste for how this can work, consider the following rule:
<list xml:id="c09-list-0001" style="bulleted"><listItem xml:id="c09-li-0001">Moving average (12) &equals; Average 12 month prices</listItem>
<listItem xml:id="c09-li-0002">If S&amp;P 500 TR Index&nbsp;&ndash;&nbsp;12&hyphen;month moving average (S&amp;P 500 TR Index) &gt; 0, go long the combination portfolio. Otherwise, go long safety (T&hyphen;bills).</listItem>
</list>
</p>
<p xml:id="c09-para-0023">The results of applying a simple trend&hyphen;following risk management overlay to the quantitative value and momentum portfolio are tabulated in Table<?xmltex \pgtag{\nobreak}?> <link href="c09-tbl-0002"/>.</p>
<?xmltex \OrgFixedPosition{c09-tbl-0002}?>
<?xmltex \pgtag{\bgroup\FloatPositionToptrue}?><tabular xml:id="c09-tbl-0002"><title type="main">Combining Quantitative Value and Quantitative Momentum</title><table pgwide="1" frame="topbot" rowsep="0" colsep="0"><tgroup cols="4"><colspec colnum="1" colname="col1" align="left"/><colspec colnum="2" colname="col2" align="center"/><colspec colnum="3" colname="col3" align="center"/><colspec colnum="4" colname="col4" align="center"/><lwtablebody><?xmltex \pgtag{\tabcolsep=0pt\begin{tabular*}{\textwidth}{@{\extracolsep\fill}ld{6}d{5}d{5}@{\extracolsep\fill}}\firsttablerule}?><thead valign="bottom"><!--<row rowsep="1">--><?xmltex \pgtag{\icolcnt=1\relax}?><entry colname="col1" align="center" xml:id="c09-ent-0051"></entry><entry colname="col2" align="center" xml:id="c09-ent-0052" lwPstyle="TabularHead">Combination<?xmltex \pgtag{\\}?> w/Trend (Net)</entry><entry colname="col3" xml:id="c09-ent-0053" align="center" lwPstyle="TabularHead">Combination<?xmltex \pgtag{\\}?>(Net)</entry><entry colname="col4" xml:id="c09-ent-0054" align="center" lwPstyle="TabularHead">S&amp;P 500<?xmltex \pgtag{\\}?> TR Index</entry><!--</row>--></thead><!--<tbody valign="top">--><!--<row>--><?xmltex \\\tablerule\pgtag{\icolcnt=1\relax}?><entry colname="col1" xml:id="c09-ent-0055"><b>CAGR</b></entry>
<entry colname="col2" xml:id="c09-ent-0056">16.57&percnt;</entry>
<entry colname="col3" xml:id="c09-ent-0057">18.10&percnt;</entry>
<entry colname="col4" xml:id="c09-ent-0058">11.16&percnt;</entry><!--</row>-->
<!--<row>--><?xmltex \\\pgtag{\icolcnt=1\relax}?><entry colname="col1" xml:id="c09-ent-0059"><b>Standard Deviation</b></entry>
<entry colname="col2" xml:id="c09-ent-0060">17.97&percnt;</entry>
<entry colname="col3" xml:id="c09-ent-0061">21.38&percnt;</entry>
<entry colname="col4" xml:id="c09-ent-0062">15.45&percnt;</entry><!--</row>-->
<!--<row>--><?xmltex \\\pgtag{\icolcnt=1\relax}?><entry colname="col1" xml:id="c09-ent-0063"><b>Downside Deviation</b></entry>
<entry colname="col2"  xml:id="c09-ent-0064">13.31&percnt;</entry>
<entry colname="col3"  xml:id="c09-ent-0065">14.96&percnt;</entry>
<entry colname="col4"  xml:id="c09-ent-0066">11.05&percnt;</entry><!--</row>-->
<!--<row>--><?xmltex \\\pgtag{\icolcnt=1\relax}?><entry colname="col1" xml:id="c09-ent-0067"><b>Sharpe Ratio</b></entry>
<entry colname="col2" xml:id="c09-ent-0068">0.67</entry>
<entry colname="col3" xml:id="c09-ent-0069">0.66</entry>
<entry colname="col4" xml:id="c09-ent-0070">0.45</entry><!--</row>-->
<!--<row>--><?xmltex \\\pgtag{\icolcnt=1\relax}?><entry colname="col1" xml:id="c09-ent-0071"><b>Sortino Ratio (MAR</b> <math display="inline" overflow="scroll" xmlns="http://www.w3.org/1998/Math/MathML" xmlns:xlink="http://www.w3.org/1999/xlink"><mrow><mo mathvariant="bold">=</mo></mrow></math> <b>5&percnt;)</b></entry>
<entry colname="col2" xml:id="c09-ent-0072">0.90</entry>
<entry colname="col3" xml:id="c09-ent-0073">0.94</entry>
<entry colname="col4" xml:id="c09-ent-0074">0.62</entry><!--</row>-->
<!--<row>--><?xmltex \\\pgtag{\icolcnt=1\relax}?><entry colname="col1" xml:id="c09-ent-0075"><b>Worst Drawdown</b></entry>
<entry colname="col2"  xml:id="c09-ent-0076">&ndash;26.18&percnt;</entry>
<entry colname="col3"  xml:id="c09-ent-0077">&ndash;60.16&percnt;</entry>
<entry colname="col4"  xml:id="c09-ent-0078">&ndash;50.21&percnt;</entry><!--</row>-->
<!--<row>--><?xmltex \\\pgtag{\icolcnt=1\relax}?><entry colname="col1" xml:id="c09-ent-0079"><b>Worst Month Return</b></entry>
<entry colname="col2" xml:id="c09-ent-0080">&ndash;25.45&percnt;</entry>
<entry colname="col3" xml:id="c09-ent-0081">&ndash;26.56&percnt;</entry>
<entry colname="col4" xml:id="c09-ent-0082">&ndash;21.58&percnt;</entry><!--</row>-->
<!--<row>--><?xmltex \\\pgtag{\icolcnt=1\relax}?><entry colname="col1" xml:id="c09-ent-0083"><b>Best Month Return</b></entry>
<entry colname="col2" xml:id="c09-ent-0084">28.69&percnt;</entry>
<entry colname="col3" xml:id="c09-ent-0085">28.69&percnt;</entry>
<entry colname="col4" xml:id="c09-ent-0086">16.81&percnt;</entry><!--</row>-->
<!--<row>--><?xmltex \\\pgtag{\icolcnt=1\relax}?><entry colname="col1" xml:id="c09-ent-0087"><b>Profitable Months</b></entry>
<entry colname="col2" xml:id="c09-ent-0088">70.93&percnt;</entry>
<entry colname="col3" xml:id="c09-ent-0089">61.18&percnt;</entry>
<entry colname="col4" xml:id="c09-ent-0090">61.59&percnt;</entry><!--</row>-->
<?xmltex \pgtag{\\ \lasttablerule\end{tabular*}}?><!--</tbody>-->
</lwtablebody></tgroup>
</table>
</tabular><?xmltex \pgtag{\egroup}?>
<p xml:id="c09-para-0029">The trend&hyphen;following overlay doesn&apos;t enhance the risk&hyphen;adjusted metrics of the combination portfolio, but this analysis misses the dramatic shift in the tail&hyphen;risk profile of the combination system. The trend overlay limits the exposure of the equity portfolio to large drawdowns. For example, the maximum drawdown goes from 60.16 percent to 26.18 percent. Of course, there are no free lunches&mdash;the trend&hyphen;following investor gives up 1.5 percent per year in compounded annual returns and the chance of enduring a bout of poor relative underperformance is enhanced with a trend&hyphen;following approach. Figure<?xmltex \pgtag{\nobreak}?> <link href="c09-fig-0003"/> shows the histogram of five&hyphen;year spreads between the combination portfolio with trend following and the buy&hyphen;and&hyphen;hold combination portfolio.</p>
<?xmltex \OrgFixedPosition{c09-fig-0003}?>
<figure xml:id="c09-fig-0003">
<mediaResource href="urn:x-wiley:9781119237198:media:w9781119237198c09:c09f003" alt="image"/>
<caption>Histogram of 5&hyphen;Year Spreads</caption>
<?xmltex \pgtag{\bgroup\FloatPositionToptrue\putfigure{3}{c09/c09f003.eps}{}{}{}\egroup}?></figure>
<p xml:id="c09-para-0030">Figure<?xmltex \pgtag{\nobreak}?> <link href="c09-fig-0003"/> highlights the &ldquo;relative performance risk&rdquo; of trend following. On one hand, trend following protects against large tail&hyphen;risk events, but the system also enhances the tracking error relative to the index, which increases the chances that an investor will not be able to stick to the investment program over the long<?xmltex \pgtag{\nobreak}?> <?xmltex \pgtag{\hbox\bgroup}?>haul.<?xmltex \pgtag{\egroup}?></p>
<p xml:id="c09-para-0031">While not the focus of this book, we encourage investors concerned about large equity drawdowns to read more about trend following. We can also augment our rule of thumb to accommodate trend following:<?xmltex \OrgFixedPosition{c09-feafxd-0002}?>
<featureFixed xml:id="c09-feafxd-0002" lwtype="Extract"><title type="featureFixedName">Extract</title><p xml:id="c09-para-0032">Buy &apos;em cheap; buy &apos;em strong; and hold &apos;em long&hellip;</p>
<p xml:id="c09-para-0033"><?xmltex \pgtag{\noindent}?>but only when the trend is your friend.</p>
</featureFixed></p></section>
<section xml:id="c09-sec-0006"><title type="main">Career Risk Considerations</title><p xml:id="c09-para-0040">Trend following, which serves to minimize the expected impact of massive drawdowns, makes the potential for relative pain more likely. Downside&hyphen;protected strategies can underperform over five&hyphen;year periods on a compound return basis with higher frequency and with more depth than buy&hyphen;and&hyphen;hold approaches. So while there are huge potential benefits of trend following, there are associated career risk considerations. In the end, how much an investor dedicates to more active exposures really depends on the willingness of an investor to eat periods of relative underperformance. For some, relative performance is irrelevant; for others facing career risk concerns, relative performance rules the day. The irony of this discussion is that the efficient market hypothesis is right&mdash;there are no free lunches&mdash;but the explanation is wrong (i.e., stock prices always reflect fundamentals). We&apos;ve already highlighted that strategies like value and momentum are a reflection of a world with mispricing. However, there is still no free lunch. Markets are extremely competitive, and many investment risks, to include things like &ldquo;relative performance risk,&rdquo; are priced risks in the real world. Financial economic models might consider the relative risk premium &ldquo;alpha,&rdquo; but to many market participants, relative performance risk is a real, quantifiable risk that marginal price setters will pay someone to take off their hands.</p>
<p xml:id="c09-para-0041">Risk, it seems, is in the eyes of the beholder.</p></section>
<section xml:id="c09-sec-0005"><title type="main">What if I Can&apos;t Handle Poor Relative Performance?</title><p xml:id="c09-para-0034">Figure<?xmltex \pgtag{\nobreak}?> <link href="c09-fig-0001"/> highlights that even a concentrated value and momentum portfolio can sustain five&hyphen;year periods of underperformance (e.g., Internet bubble period and the post&hyphen;2008 financial crisis era). For many investors, this is simply too much pain to endure, and any excess expected returns associated with a willingness to bear that sort of &ldquo;relative pain&rdquo; are fairly earned by those who have the stomach to deal with it. And a trend&hyphen;following overlay only makes the chance of enduring a long&hyphen;stretch of relative pain even worse! The ultimate solution is to eliminate barriers and accept relative performance pain, but as we&apos;ve discussed throughout this book, career risk concerns and psychology problems prevent many investors from fully exploiting sustainable active strategies. After all, this is the reason certain active strategies work in the first place&mdash;they&apos;re tough to follow!</p>
<p xml:id="c09-para-0035">We recognize that the high&hyphen;conviction solution can never be deployed by a large swath of the investing public. Nonetheless, not all is lost, as investors have varying tolerances for relative performance pain. A majority of investors can&apos;t hold high conviction value and momentum, but some investors can add a small piece of high conviction value and momentum and bolt it on their passive allocation to the market. For example, consider a financial adviser who has a fairly sophisticated client base, but these clients have limited assessment horizons and cling to benchmarks. Large deviations from a benchmark&mdash;even with smarter clients&mdash;can create angry clients very quickly: &ldquo;Hey Mr. Adviser, why are we losing by 10 percentage points relative to the S&amp;P 500 index this quarter? You&apos;re fired!&rdquo; But maybe a smaller deviation (e.g., 2 percentage points off the S&amp;P 500) is less of an issue? Perhaps the adviser can survive the client performance meeting and explain why the risk of underperformance is the cost of admission to longer&hyphen;term expected outperformance. For investors in this situation, a core&hyphen;satellite approach may be<?xmltex \pgtag{\break}?> warranted.</p>
<p xml:id="c09-para-0036">The <i>core&hyphen;satellite approach</i> works as follows: The approach dedicates a large chunk of capital to a passive benchmark strategy (the &ldquo;core&rdquo;) and only adds a small component of an active strategy around the edges (the &ldquo;satellite&rdquo;). By construction, a core&hyphen;satellite approach will never deviate too far from a benchmark. For example, in Figure<?xmltex \pgtag{\nobreak}?> <link href="c09-fig-0004"/> we create a portfolio that is 80 percent allocated to the S&amp;P 500 and 20 percent allocated to the quantitative value and momentum portfolio described in the prior<?xmltex \pgtag{\break}?> section.</p>
<?xmltex \OrgFixedPosition{c09-fig-0004}?>
<figure xml:id="c09-fig-0004">
<mediaResource href="urn:x-wiley:9781119237198:media:w9781119237198c09:c09f004" alt="image"/>
<caption>Histogram of Five&hyphen;Year Spreads</caption>
<?xmltex \pgtag{\bgroup\FloatPositionToptrue\putfigure{4}{c09/c09f004.eps}{}{}{}\egroup}?></figure>
<p xml:id="c09-para-0037">The figure shows that the core&hyphen;satellite approach cannot eliminate relative pain. The core&hyphen;satellite investor would still need to endure pain during the Internet bubble period and the post&hyphen;2008 financial crisis period, but the pain is more tolerable. Of course, the downside of the core&hyphen;satellite approach is a much lower long&hyphen;term expected compounding rate than an undiluted combination approach (see the &ldquo;Combination (Net)&rdquo; column relative to the &ldquo;Core&hyphen;Satellite (Net)&rdquo; column in Table<?xmltex \pgtag{\nobreak}?> <link href="c09-tbl-0003"/>, which outlines the summary statistics from 1974 to 2014).</p><?xmltex \pgtag{\vfill\eject}?>
<?xmltex \OrgFixedPosition{c09-tbl-0003}?>
<?xmltex \pgtag{\bgroup\FloatPositionToptrue}?><tabular xml:id="c09-tbl-0003"><title type="main">Core&hyphen;Satellite Returns</title><table pgwide="1" frame="topbot" rowsep="0" colsep="0"><tgroup cols="5"><colspec colnum="1" colname="col1" align="left"/><colspec colnum="2" colname="col2" align="center"/><colspec colnum="3" colname="col3" align="center"/><colspec colnum="4" colname="col4" align="center"/><colspec colnum="5" colname="col5" align="center"/><lwtablebody><?xmltex \pgtag{\tabcolsep=0pt\begin{tabular*}{\textwidth}{@{\extracolsep\fill}ld{5}d{5}d{5}d{5}@{\extracolsep\fill}}\firsttablerule}?><thead valign="bottom"><!--<row rowsep="1">--><?xmltex \pgtag{\icolcnt=1\relax}?><entry colname="col1" align="center" xml:id="c09-ent-0091"></entry><entry colname="col2" xml:id="c09-ent-0092" align="center" lwPstyle="TabularHead">Core&hyphen;Satellite<?xmltex \pgtag{\\}?> (Net)</entry><entry colname="col3" align="center" xml:id="c09-ent-0093" lwPstyle="TabularHead">Combination<?xmltex \pgtag{\\}?> (Net)</entry><entry colname="col4" xml:id="c09-ent-0094" align="center" lwPstyle="TabularHead">Quantitative<?xmltex \pgtag{\\}?> Momentum<?xmltex \pgtag{\\}?> (Net)</entry><entry colname="col5" xml:id="c09-ent-0095" lwPstyle="TabularHead">S&amp;P 500<?xmltex \pgtag{\\}?> TR Index</entry><!--</row>--></thead><!--<tbody valign="top">--><!--<row>--><?xmltex \\\tablerule\pgtag{\icolcnt=1\relax}?><entry colname="col1" xml:id="c09-ent-0096"><b>CAGR</b></entry>
<entry colname="col2" xml:id="c09-ent-0097">12.66&percnt;</entry>
<entry colname="col3" xml:id="c09-ent-0098">18.10&percnt;</entry>
<entry colname="col4" xml:id="c09-ent-0099">17.38&percnt;</entry>
<entry colname="col5" xml:id="c09-ent-0100">11.16&percnt;</entry><!--</row>-->
<!--<row>--><?xmltex \\\pgtag{\icolcnt=1\relax}?><entry colname="col1" xml:id="c09-ent-0101"><b>Standard Deviation</b></entry>
<entry colname="col2" xml:id="c09-ent-0102">16.04&percnt;</entry>
<entry colname="col3" xml:id="c09-ent-0103">21.38&percnt;</entry>
<entry colname="col4" xml:id="c09-ent-0104">25.59&percnt;</entry>
<entry colname="col5" xml:id="c09-ent-0105">15.45&percnt;</entry><!--</row>-->
<!--<row>--><?xmltex \\\pgtag{\icolcnt=1\relax}?><entry colname="col1" xml:id="c09-ent-0106"><b>Downside Deviation</b></entry>
<entry colname="col2" xml:id="c09-ent-0107">11.48&percnt;</entry>
<entry colname="col3" xml:id="c09-ent-0108">14.96&percnt;</entry>
<entry colname="col4" xml:id="c09-ent-0109">18.09&percnt;</entry>
<entry colname="col5" xml:id="c09-ent-0110">11.05&percnt;</entry><!--</row>-->
<!--<row>--><?xmltex \\\pgtag{\icolcnt=1\relax}?><entry colname="col1" xml:id="c09-ent-0111"><b>Sharpe Ratio</b></entry>
<entry colname="col2" xml:id="c09-ent-0112">0.52</entry>
<entry colname="col3" xml:id="c09-ent-0113">0.66</entry>
<entry colname="col4" xml:id="c09-ent-0114">0.57</entry>
<entry colname="col5" xml:id="c09-ent-0115">0.45</entry><!--</row>-->
<!--<row>--><?xmltex \\\pgtag{\icolcnt=1\relax}?><entry colname="col1" xml:id="c09-ent-0116"><b>Sortino Ratio (MAR</b> <math display="inline" overflow="scroll" xmlns="http://www.w3.org/1998/Math/MathML" xmlns:xlink="http://www.w3.org/1999/xlink"><mrow><mo mathvariant="bold">=</mo></mrow></math> <b>5&percnt;)</b></entry>
<entry colname="col2" xml:id="c09-ent-0117">0.72</entry>
<entry colname="col3" xml:id="c09-ent-0118">0.94</entry>
<entry colname="col4" xml:id="c09-ent-0119">0.80</entry>
<entry colname="col5" xml:id="c09-ent-0120">0.62</entry><!--</row>-->
<!--<row>--><?xmltex \\\pgtag{\icolcnt=1\relax}?><entry colname="col1" xml:id="c09-ent-0121"><b>Worst Drawdown</b></entry>
<entry colname="col2" xml:id="c09-ent-0122">&ndash;51.86&percnt;</entry>
<entry colname="col3" xml:id="c09-ent-0123">&ndash;60.16&percnt;</entry>
<entry colname="col4" xml:id="c09-ent-0124">&ndash;67.72&percnt;</entry>
<entry colname="col5" xml:id="c09-ent-0125">&ndash;50.21&percnt;</entry><!--</row>-->
<!--<row>--><?xmltex \\\pgtag{\icolcnt=1\relax}?><entry colname="col1" xml:id="c09-ent-0126"><b>Worst Month Return</b></entry>
<entry colname="col2" xml:id="c09-ent-0127">&ndash;22.35&percnt;</entry>
<entry colname="col3" xml:id="c09-ent-0128">&ndash;26.56&percnt;</entry>
<entry colname="col4" xml:id="c09-ent-0129">&ndash;30.33&percnt;</entry>
<entry colname="col5" xml:id="c09-ent-0130">&ndash;21.58&percnt;</entry><!--</row>-->
<!--<row>--><?xmltex \\\pgtag{\icolcnt=1\relax}?><entry colname="col1" xml:id="c09-ent-0131"><b>Best Month Return</b></entry>
<entry colname="col2" xml:id="c09-ent-0132">16.52&percnt;</entry>
<entry colname="col3" xml:id="c09-ent-0133">28.69&percnt;</entry>
<entry colname="col4" xml:id="c09-ent-0134">34.67&percnt;</entry>
<entry colname="col5" xml:id="c09-ent-0135">16.81&percnt;</entry><!--</row>-->
<!--<row>--><?xmltex \\\pgtag{\icolcnt=1\relax}?><entry colname="col1" xml:id="c09-ent-0136"><b>Profitable Months</b></entry>
<entry colname="col2" xml:id="c09-ent-0137">61.79&percnt;</entry>
<entry colname="col3" xml:id="c09-ent-0138">61.18&percnt;</entry>
<entry colname="col4" xml:id="c09-ent-0139">61.79&percnt;</entry>
<entry colname="col5" xml:id="c09-ent-0140">61.59&percnt;</entry><!--</row>-->
<?xmltex \pgtag{\\ \lasttablerule\end{tabular*}}?><!--</tbody>-->
</lwtablebody></tgroup>
</table>
</tabular><?xmltex \pgtag{\egroup}?></section>
<?xmltex \pgtag{\tablenotecnt=6\def\itemwd{16.}}?><noteGroup xml:id="c09-ntgp-0001"><title type="main">Notes</title>
<note xml:id="c09-note-0001">Wesley Gray and Tobias Carlisle, <i>Quantitative Value: A Practitioner&apos;s Guide to Automating Intelligent Investment and Eliminating Behavioral Errors</i>,<?xmltex \pgtag{\break}?> (Hoboken, NJ: John Wiley &amp; Sons, 2012).</note>
<note xml:id="c09-note-0002">We increase the total fee from 1.80 percent in Chapter <exlink href="urn:x-wiley:9781119237198:xml-component:w9781119237198c08"/> to 2.00 percent in Chapter <exlink href="urn:x-wiley:9781119237198:xml-component:w9781119237198c09"/> to account for higher transaction costs associated with running an equal&hyphen;weight portfolio, as well as annually rebalancing between the quantitative momentum and quantitative value portfolios.</note></noteGroup>
</body>
</component>
