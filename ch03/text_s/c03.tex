%%%%% Please note that the below listed 2 lines needs to be moved to %%%%
%%%%% a new file 'c03.tml' and the same should be compiled to get the  %%%%
%%%%% typeset pages                                                  %%%%
\def\xmlfile{c03.tml}
\input xmltex
%%%%% END  %%%%%%%%%%%%%%%%%%%%%%%%
<?xml version="1.0" encoding="utf-8"?>
<!DOCTYPE component SYSTEM "file://chgnsm02/macdata/Books/ptg/Books-Documents/Approved-Documents/Guidelines/WileyML3G/WileyML_3G_v2.0/Wileyml3gv20-flat.dtd">
<component version="2.0" xmlns:cms="http://www.wiley.com/namespaces/wiley" xmlns:wiley="http://www.wiley.com/namespaces/wiley/wiley" type="bookChapter" xml:lang="en" xml:id="w9781119237198c03">
<?xmltex \pgtag{\IIIProofVersionInfo{c03}}?>
<?xmltex \pgtag{\setcounter{chapter}{2}\setcounter{page}{43}}?>
<?xmltex \pgtag{\def\Gpath{u:/books/Wiley/Pd/E-line/Reprint/Gray37198/figures/iround}%
}?>
<header xml:id="c03-hdr-0001">
<publicationMeta level="product">
<publisherInfo>
<publisherName>John Wiley &amp; Sons, Inc.</publisherName>
<publisherLoc>Hoboken, New Jersey</publisherLoc>
</publisherInfo>
<isbn type="print-13">9781119237198</isbn>
<titleGroup><title type="main" sort="QUANTITATIVE MOMENTUM">Quantitative Momentum</title></titleGroup>
<copyright ownership="publisher">Copyright &copy; 2016 by John Wiley &amp; Sons, Inc. All rights reserved.</copyright>
<numberingGroup>
<numbering type="edition" number="1">1st Edition</numbering>
</numberingGroup>
<creators><creator xml:id="cr-0001" creatorRole="author"><personName><givenNames>Wesley R.</givenNames> <familyName>Gray</familyName></personName></creator></creators>
<subjectInfo>
<subject href="http://psi.wiley.com/subject/ME20">n/a</subject>
</subjectInfo>
</publicationMeta>
<publicationMeta level="unit" position="40" type="chapter">
<idGroup>
<id type="unit" value="c03"/>
<id type="file" value="c03"/>
</idGroup>
<titleGroup><title type="name">Chapter</title></titleGroup>
<eventGroup>
<event type="xmlCreated" agent="SPi Global" date="2016-07-12"/>
</eventGroup>
<numberingGroup>
<numbering type="main">3</numbering>
</numberingGroup>
<objectNameGroup>
<objectName elementName="featureFixed">Extract</objectName>
</objectNameGroup>
</publicationMeta>
<contentMeta>
<titleGroup><title type="main">Momentum Investing Is Not Growth Investing</title></titleGroup>
</contentMeta>
</header>
<body sectionsNumbered="no" xml:id="c03-body-0001"><section type="opening" xml:id="c03-sec-0001"><p xml:id="c03-para-0001"><?xmltex \OrgFixedPosition{c03-blkfxd-0001}?><blockFixed type="standFirst" xml:id="c03-blkfxd-0001"><p xml:id="c03-para-0002">&ldquo;The dumbest reason in the world to buy a stock is because it&apos;s going up.&rdquo;</p>
<source>&mdash;Attributed to Warren Buffett<link href="c03-note-0001"/></source>
</blockFixed></p>
<p xml:id="c03-para-0003"><?xmltex \pgtag{\firstlet}?>We use the term <i>momentum</i> to mean a continuation of past relative returns&mdash;past winners tend to be future winners, while past losers tend to be future losers. Practitioners often refer to this class of strategies as relative strength strategies, which have been around for a long time. In fact, Robert Levy published a paper in 1967 called &ldquo;Relative Strength as a Criterion for Investment Selection.&rdquo; Mr. Levy outlines his conclusion: &ldquo;The profits attainable by purchasing the historically strongest stocks are superior to the profits from random selection.&rdquo;<sup>2</sup> Oddly enough, research on relative strength strategies went dormant following Levy&apos;s contribution. What happened? The efficient market hypothesis happened.</p></section>
<section xml:id="c03-sec-0002"><title type="main">The Efficient Market Mafia Kills<?xmltex \pgtag{\protect\break}?> Relative Strength</title><p xml:id="c03-para-0004">As we alluded to in Chapter <exlink href="urn:x-wiley:9781119237198:xml-component:w9781119237198c02"/>, the efficient market hypothesis (EMH) was developed at the University of Chicago in the 1960s and 1970s. The EMH hypothesis subsequently flourished across academia. Under the semi&hyphen;strong form interpretation of the EMH, asset prices reflect all publicly available information so that there is no way for investors to consistently outperform a randomly selected basket of securities after controlling for risk. Or as EMH proponent Burton Malkiel so eloquently put it in his 1973 classic, <i>A Random Walk Down Wall Street</i>: &ldquo;A blindfolded monkey throwing darts at a newspaper&apos;s financial pages could select a portfolio that would do just as well as one carefully selected by experts.&rdquo;<link href="c03-note-0003"/> Thus, from an EMH perspective, for all intents and purposes, Levy&apos;s evidence on the performance of relative strength strategies was an impossibility.</p>
<p xml:id="c03-para-0005">It seems that practitioners like Levy (who worked in the private sector at the time his paper was published), were overtaken by the cult of academics focused on pursuing the efficient market hypothesis. Practitioners were essentially banned from publishing in top&hyphen;tier academic finance journals and academics pursuing research interests that went counter to the EMH idea were driven from the emerging EMH temple.<link href="c03-note-0004"/> The subsequent 25 years of published academic research entered a dark age, and discussions of relative strength strategies were effectively banned, since space was primarily reserved for EMH cheerleaders.</p>
<p xml:id="c03-para-0006">Yet, all was not well in the ivory tower. Anomalies that were inconsistent with EMH began to emerge in the literature in the 1970s. For example, as previously mentioned, Ben Graham, among others, had shown that buying a basket of cheap stocks tended to outperform the market, and academics began to formally examine the value effect. Evidence related to value and other so&hyphen;called anomalies began to accumulate, hinting that there might be kinks in the EMH armor, but EMH proponents remained confident. However, around the same time that many EMH supporters were basking in glory, Daniel Kahneman, working with Amos Tversky, started exploring how human biases affected financial decision making. Kahneman and Tversky established some of the earliest connections between investors&apos; internal behavioral biases and many of the observed anomalies that were being identified in the academic finance literature.</p></section>
<section xml:id="c03-sec-0003"><title type="main">&ldquo;Momentum&rdquo; Rises from the<?xmltex \pgtag{\protect\nobreak}?> Ashes</title><p xml:id="c03-para-0007">Finally, in the early 1990s, Narasimhan Jegadeesh and Sheridan Titman revitalized the findings from Levy&apos;s 1967 paper in their pioneering 1993 article &ldquo;Returns to Buying Winners and Selling Losers: Implications for Market Efficiency.&rdquo; This paper essentially replicated the spirit of the analysis conducted by Levy in 1967, but with the benefit of more data, computational power, and willingness on behalf of the establishment to publish research that questioned EMH. By now, the cracks in the EMH armor were getting<?xmltex \pgtag{\nb}?> bigger.</p>
<p xml:id="c03-para-0008">Interestingly enough, Jegadeesh and Titman never mention the word <i>momentum</i> in their original paper, even though their paper is considered by many to be the seminal work on modern&hyphen;era stock selection momentum strategies. We posit that the term <i>momentum</i> was adopted after Mark Carhart published his University of Chicago dissertation in <i>The Journal of Finance in 1997</i>. In this paper, Carhart creates a <i>momentum factor,</i> which essentially reflected the relative strength of the stock selection strategies outlined in Jegadeesh and Titman&apos;s paper.<link href="c03-note-0005"/> Soon after Carhart&apos;s paper, <i>momentum</i> became the new academic term for the age&hyphen;old relative strength strategy. With the floodgates open, researchers published a flurry of papers on momentum strategies. The evidence was so overwhelming that the anomaly was crowned the &ldquo;premier anomaly&rdquo; by none other than Eugene Fama&mdash;an original architect of the EMH theory.<link href="c03-note-0006"/></p>
<p xml:id="c03-para-0009">Remarkably, while modern&hyphen;day academics refocused on the concepts of stock selection momentum, many practitioners continued to be stuck in a time warp. The reasons for this regressive behavior are likely related to the practitioner training pipeline. The academics, who train all the MBAs that go on to manage portfolios, were still being taught portfolio mathematics so they could solve asset allocation decisions. Stock picking training was a waste of time because it was a sucker&apos;s game under a strict interpretation of the EMH. And of course, for the MBA &ldquo;rebels,&rdquo; there was always the value anomaly to pursue, which had been popularized by the intense success of Warren Buffett, the folksy investment hero from Omaha, Nebraska. Unlike value, however, there were no vibrant champions for momentum investing&mdash;no Ben Graham, no Warren Buffett. To make matters worse, the heroes associated with the value investing school of thought were, ironically, <i>agreeing</i> with the EMH academics when it came to momentum. Their value investing approach was perfectly reasonable, but momentum investing was deemed a black art, a kind of voodoo magic, only practiced by fools and heretics. Of course, all of this flew in the face of the actual evidence, which suggests that momentum investing is an even better anomaly than value investing.<link href="c03-note-0007"/></p>
<p xml:id="c03-para-0010">EMH enjoyed great success for many years, with reams of academic papers showcasing how efficient markets had become. In many respects, EMH had won the argument&mdash;prices are generally efficient. But this price efficiency is why the evidence on momentum investing was so disheartening to the EMH school of thought. The value anomaly was one thing&mdash;perhaps investors could beat the market if they used their intellect, did their in&hyphen;depth homework, and understood the financial statements better than the next investor. But the momentum anomaly was saying something completely different: Price momentum had nothing to do with fundamentals, so even a halfwit could pursue a successful strategy focused solely on relative price performance, since this simple metric seemed to predict future prices. This finding conflicted with even the weakest form of EMH. Houston we have a problem.</p></section>
<section xml:id="c03-sec-0004"><title type="main">Behavioral Finance Theorists Explain Momentum</title><p xml:id="c03-para-0011">To the credit of academic researchers, the financial economics field moved forward and the behavioral finance paradigm arose, phoenix&hyphen;like, from the ashes of EMH. This new paradigm held tightly to the EMH as a baseline hypothesis, but relaxed assumptions regarding investor rationality and frictionless markets, in order to understand and explain how and why prices might deviate from their efficient levels. This framework laid the foundation for the sustainable active investing concepts outlined in Chapter <exlink href="urn:x-wiley:9781119237198:xml-component:w9781119237198c02"/>.</p>
<p xml:id="c03-para-0012">Hard&hyphen;core value investors have a different kind of angst regarding momentum, a kind of anxiety that is grounded in clouded reasoning and a religious zeal, as is evidenced by quotations from value investing books and personas. For example, Warren Buffett is reported to have said, &ldquo;The dumbest reason in the world to buy a stock is because it&apos;s going up.&rdquo; As a rule of thumb, Buffett&apos;s advice isn&apos;t a bad rule, and Warren Buffett is clearly an extraordinary investor whose insights are worthy of investigation. But rules of thumb don&apos;t always capture the nuance of a situation. Higher prices may not always be an unreliable signal. For example, what if the intrinsic value of a stock is higher than the new, higher price&mdash;is that not still a value investment? Or perhaps there is a genuine positive feedback loop associated with higher prices that in turn increases the intrinsic value of the firm? High price movements may lower the cost of capital for a firm, allow them to attract better human capital, or even generate free advertising, thus increasing fundamental value, albeit in a reflexive way. In short, growth stocks, defined as stocks with high prices to fundamentals, are generally a bad thing, but higher prices, per se, aren&apos;t always a bad thing. In fact, they are generally a positive development, all else equal.</p>
<p xml:id="c03-para-0013">Consider the following hypothetical scenario:
<list xml:id="c03-list-0001" style="bulleted"><listItem xml:id="c03-li-0001">Facebook has gone up 100 percent the past year and has a price to<?xmltex \pgtag{\break}?> earnings ratio of 15.</listItem>
<listItem xml:id="c03-li-0002">Google has gone down 50 percent and has a price to earnings ratio of<?xmltex \pgtag{\nb}?> 15.</listItem>
</list>
</p>
<p xml:id="c03-para-0014">Which is the better buy? For a classic value investor, these stocks are arguably the same from a valuation perspective since they both have the same price to earnings ratio of 15. However, based on psychology, some investors will &ldquo;feel&rdquo; like Google is a better opportunity, since value stocks are often those that have declined in price. What value investor wants to buy a stock that is up 100 percent? True value investors are genetically programmed to be suspicious of strong upward price moves&mdash;we know because we are value investors by nature! Strong upward price movements are typically a bad signal when it comes to a traditional value or distressed investing opportunity. Upward moves suggest things are not as cheap as they were before, and are more expensive now, at least comparatively. But this feeling of disgust that is associated with buying a high&hyphen;flying stock is not specific to value investors; this distrust is also felt more generally by all investors&mdash;nobody wants to be the sucker that bought after the price moved higher. In fact, people can feel a contrary urge. If you own a stock that has gone up in value, you may seek to realize the gain by selling it&mdash;after all, it feels good to realize a gain. This effect is often termed the <i>disposition effect</i>. There is strong empirical evidence to support the theory that the disposition effect is related to the momentum anomaly.<link href="c03-note-0008"/></p>
<p xml:id="c03-para-0015">Consider a stock that is at a 52&hyphen;week high&mdash;many investors interpret this to mean the stock is overvalued and unlikely to go higher, even if it may still be cheap on a fundamental basis. The mainstream interpretation is patently false: 52&hyphen;week&hyphen;high stocks greatly outperform 52&hyphen;week&hyphen;low stocks.<link href="c03-note-0009"/> But if many market participants have these kinds of biases, a reasonable hypothesis is that there will be price pressure&mdash;unrelated to fundamentals&mdash;that may prevent a security from reaching its true fundamental value because market participants perceive that, for some gut&hyphen;based reason, &ldquo;the stock has already gone up too much.&rdquo; This situation would be a case where momentum investing is essentially a cousin, not an enemy, of value investing. How so? Value investing&apos;s edge is often characterized as <i>pessimism</i> in the presence of <i>poor</i> short&hyphen;term fundamentals, which causes stocks to become too cheap relative to future expectations. Perhaps momentum investing&apos;s edge could be characterized as <i>pessimism</i> in the presence of <i>strong</i> short&hyphen;term fundamentals, which causes stocks to remain too cheap to future expectations.</p></section>
<section xml:id="c03-sec-0005"><title type="main">Wait a Minute: Momentum Investing Is Just Growth Investing, Which Doesn&apos;t Work!</title><p xml:id="c03-para-0016">Hold on, if we are arguing that stocks can be cheap due to pessimism related to <i>strong</i> short&hyphen;term fundamentals, isn&apos;t that &hellip; growth investing?</p>
<p xml:id="c03-para-0017"><i>No.</i></p>
<p xml:id="c03-para-0018"><?xmltex \pgtag{\changespaceskip{2.4}}?>But before clarifying, let&apos;s review the psychology behind value versus growth. In our prior discussion of value and growth, the evidence showed that value beats growth. The reason for this spread is partly attributable to mispricing from behavioral bias in the market. For example, the original Lakonishok, Shleifer, and Vishny study mentioned in Chapter <exlink href="urn:x-wiley:9781119237198:xml-component:w9781119237198c02"/> showed that price to fundamental ratios serve as a proxy for expectation errors exhibited in the market. Recall that investors thought high past earnings<?xmltex \pgtag{\nb}?> growth<?xmltex \pgtag{\nb}?> rates would continue for growth stocks, and low past earnings growth rates<?xmltex \pgtag{\nb}?> would<?xmltex \pgtag{\nb}?> continue for value stocks. The evidence showed this expected result did not in fact occur. Follow&hyphen;on studies debate this core result from Lakonishok, Vishny, and Shleifer&apos;s original findings,<link href="c03-note-0010"/> but these papers fail to address more recent work, including Daniel and Titman&apos;s 1997 paper on value characteristics and stock returns, Piotroski and So&apos;s 2012 paper on the interaction between value investing returns and fundamentals, and Jack&apos;s 117&hyphen;page dissertation, which is a deep&hyphen;dive into the concepts outlined in Piotroski and So&apos;s work. These more recent papers confirm that the value anomaly, while volatile and costly to exploit, is likely driven, in part,<?xmltex \pgtag{\break}?> by mispricing.<link href="c03-note-0011"/></p>
<p xml:id="c03-para-0019">Thus, on average, investors seem to over&hyphen;extrapolate good news from growth firms (firms with high price to fundamentals), driving them above intrinsic value, and do the opposite with value firms (firms with low price to fundamentals), driving them below intrinsic value. So in the value&hyphen;investing framework, growth investors seem to be too <i>optimistic</i>, given strong fundamentals. But are we now saying that momentum investors are too <i>pessimistic</i>, given strong prices? These positions seem to be in conflict, but we will explain.</p>
<p xml:id="c03-para-0020">Let us be clear: Momentum investing is <i>not</i> growth investing. Growth investing, in accordance with the studies mentioned, is characterized by securities that have high prices relative to past fundamentals (e.g., price&hyphen;to&hyphen;earnings ratio). We acknowledge that there are many alternative ways to define growth investing in practice (e.g., growth at a reasonable price), but we will stick with the academic convention for the purposes of our argument. In contrast to growth, we characterize momentum investing as securities that have strong relative performance to all other securities, <i>independent</i> of fundamentals. For example, a momentum strategy might consider the cumulative returns of prices over the last 12 months relative to other stocks, but earnings, or any other fundamental metric, would play <i>no part</i> in the analysis. To paraphrase Vince Lombardi: With momentum, prices aren&apos;t everything; they are the only thing.</p>
<p xml:id="c03-para-0021">We will argue that strong momentum signals, similar to low price to fundamental ratios (i.e., value measures), are a proxy for investor expectation errors, and help an informed investor systematically identify situations where behavioral bias is preventing securities, on average, from reaching perceived fundamental value. Think back to the poor poker players in the sustainable active investing framework in Chapter <exlink href="urn:x-wiley:9781119237198:xml-component:w9781119237198c02"/>. As a first step in identifying sustainable active strategies, we need some market participants to be less than rational in order to create a mispricing opportunity. We will come back to &ldquo;poor poker players&rdquo; and the mechanics of how and why momentum works later. However, this point regarding the difference between the signal that characterizes momentum (i.e., price&hyphen;only) versus growth (i.e.,<?xmltex \pgtag{\nb}?> price <?xmltex \pgtag{\bgroup\mbox}?>relative<?xmltex \pgtag{\egroup}?> to some fundamental) is extremely important to understand to ensure that readers are not confused and think that growth investing is the same thing as momentum investing.</p>
<p xml:id="c03-para-0022">The best way to make the point clear is with data. We examine the overlap between a portfolio of mid&hyphen;to&hyphen;large market capitalization firms selected on a generic momentum signal (top decile of firms with relative strongest 12&hyphen;month performance, skipping the previous month) and a portfolio of firms selected based on a generic price to fundamental signal (top decile of firms with the highest price&hyphen;to&hyphen;book ratio&mdash;or alternatively, the book&hyphen;to&hyphen;market ratio, otherwise known as growth firms) for the period between 1963 and 2013. Surprisingly, there is only a 21 percent overlap between the names in the high momentum portfolios and the names in the growth portfolio. Thus, many momentum<?xmltex \pgtag{\nb}?> stocks are not growth stocks, and many growth stocks are not momentum stocks. In fact, a high momentum stock can be a value stock, a growth stock, or anything in between.</p></section>
<section xml:id="c03-sec-0006"><title type="main">Digging Deeper into Growth versus Momentum</title><p xml:id="c03-para-0023">In the following analysis we dig a little deeper into the characteristics of growth firms and high momentum firms. Our data sample includes all firms on the New York Stock Exchange (NYSE), American Stock Exchange (AMEX), and NASDAQ with the required data on CRSP and Compustat, which are the academic gold standard for financial data analysis. We only examine firms with ordinary common equity on CRSP and eliminate all REITS, ADRS, closed&hyphen;end funds, utilities, and financial firms. We incorporate CRSP delisting return data using the technique of Beaver, McNichols, and Price.<link href="c03-note-0012"/> To be included in the sample, all firms must have a non&hyphen;zero market value of equity as of June 30 of year <i>t</i>. We use book to market (B/M) as our annual indicator of &ldquo;valuation&rdquo; based on academic convention. Book is computed on June 30 each year using the methodology from Fama and French<link href="c03-note-0013"/> and the market capitalization on June 30. All firms with negative book values are eliminated from the sample. We consider &ldquo;growth&rdquo; firms to be those with the most expensive B/M ratios (i.e., lower is more expensive). We calculate generic momentum by ranking all stocks monthly on their cumulative 12&hyphen;month returns, skipping the most recent month, similar to Fama and French.</p>
<p xml:id="c03-para-0024">The tests are focused on all mid&hyphen; and large&hyphen;cap stocks, defined as stocks with a market capitalization above the NYSE 40<sup>th</sup> percentile for market capitalization. This approach seeks to determine if the empirical results are applicable to the broader universe of stocks and are robust to size and <?xmltex \pgtag{\bgroup\mbox}?>liquidity<?xmltex \pgtag{\egroup}?> effects over time. Our choice to focus on more liquid firms means that our conclusions may not be applicable to small illiquid firms.</p>
<p xml:id="c03-para-0025">We follow a simulation approach that works as follows:
<list xml:id="c03-list-0002" style="bulleted"><listItem xml:id="c03-li-0003">Each month we randomly draw 30 &ldquo;growth stocks&rdquo; and 30 &ldquo;momentum&rdquo; stocks from our top decile of growth stocks and high momentum stocks.</listItem>
<listItem xml:id="c03-li-0004">Repeat every month from 1963 to 2013 to create a monthly rebalance portfolio of &ldquo;growth&rdquo; stocks and a monthly rebalanced portfolio of &ldquo;momentum&rdquo; stocks.</listItem>
<listItem xml:id="c03-li-0005">Calculate performance statistics on the growth strategy and the momentum strategy from 1963 to 2013.</listItem>
<listItem xml:id="c03-li-0006">Repeat the above steps 1,000 times.</listItem>
</list>
</p>
<p xml:id="c03-para-0026">The experiment above is equivalent to having a monkey, focused on growth stocks, throw 30 darts at the growth stock dartboard every month for 50 years, and another monkey, focused on momentum stocks, which will throw 30 darts at the momentum stock dartboard every month for 50 years. We&apos;ll then have both the growth and the momentum monkey do this exercise 1,000 times, so at the end we will have a sample of 1,000 separate portfolio manager monkeys from each<?xmltex \pgtag{\nobreak}?> <?xmltex \pgtag{\hbox\bgroup}?>camp.<?xmltex \pgtag{\egroup}?></p>
<p xml:id="c03-para-0027">Now some monkeys will perform well, and others will perform poorly, simply based on luck. But recall that momentum monkeys always make their picks the top momentum stock decile, and growth monkeys always make their picks from the top growth stock decile.</p>
<p xml:id="c03-para-0028">First, Figure<?xmltex \pgtag{\nobreak}?> <link href="c03-fig-0001"/> shows the distribution of compound annual growth rates for the growth monkeys and the momentum monkeys.</p>
<?xmltex \OrgFixedPosition{c03-fig-0001}?>
<figure xml:id="c03-fig-0001">
<mediaResource href="urn:x-wiley:9781119237198:media:w9781119237198c03:c03f001" alt="image"/>
<caption>CAGR: Growth Monkeys versus Momentum Monkeys</caption>
<?xmltex \pgtag{\bgroup\FloatPositionPagetrue\figbotskip=-12pt\putfigure{1}{c03/c03f001.eps}{}{}{}\egroup}?></figure>
<?xmltex \OrgFixedPosition{c03-fig-0002}?>
<figure xml:id="c03-fig-0002">
<mediaResource href="urn:x-wiley:9781119237198:media:w9781119237198c03:c03f002" alt="image"/>
<caption>Volatility: Growth Monkeys versus Momentum Monkeys</caption>
<?xmltex \pgtag{\bgroup\FloatPositionPagetrue\figbotskip=-24pt\putfigure{2}{c03/c03f002.eps}{}{}{}\egroup}?></figure>
<p xml:id="c03-para-0029">Relative to their monkey peers, some of the luckiest growth monkeys did well, averaging approximately 14 percent over the period, and some very unlucky <?xmltex \pgtag{\bgroup\mbox}?>momentum<?xmltex \pgtag{\egroup}?> <?xmltex \pgtag{\bgroup\mbox}?>monkeys<?xmltex \pgtag{\egroup}?> did poorly, averaging approximately 17 percent over the period. However, incredibly, over the 50&hyphen;year period there isn&apos;t a single dart&hyphen;throwing growth monkey who outperformed <i>any</i> dart&hyphen;throwing momentum monkey. This result is stunning. Typically, when one runs a thousand simulations, one identifies some overlap in the &ldquo;tails,&rdquo; or extreme ends of the distribution. Clearly, from a compound return perspective, momentum is different from growth.</p>
<p xml:id="c03-para-0030">Next, let&apos;s look at a comparison of the volatility of the growth monkey portfolios and the volatility of the momentum monkey portfolios. Perhaps the return outperformance of momentum versus growth is compensation for extra risk associated with a generic momentum strategy.</p>
<p xml:id="c03-para-0031">Figure<?xmltex \pgtag{\nobreak}?> <link href="c03-fig-0002"/> highlights little difference in annualized volatility for the portfolios created using either a growth monkey or a momentum monkey to pick stocks. The distribution is narrow.</p>
<?xmltex \pgtag{\vfill\eject}?>
<p xml:id="c03-para-0032">But perhaps volatility doesn&apos;t capture the true risk of the momentum strategy relative to the growth portfolios? We examine the worst drawdowns, or the worst peak&hyphen;to&hyphen;trough performance during the 50&hyphen;year time period, as an extreme tail event. Figure<?xmltex \pgtag{\nobreak}?> <link href="c03-fig-0003"/> tabulates the extreme loss scenarios across the thousand simulations for both the growth and momentum strategies.</p>
<p xml:id="c03-para-0033">Note that higher drawdowns are reflected on the left side of Figure<?xmltex \pgtag{\nobreak}?> <link href="c03-fig-0003"/>, while lower drawdowns are reflected on the right. Because the light gray bars are clustered on the left, the tail risk for growth, on average, is actually <i>higher</i> than for momentum, which is clustered on the right. There is some overlap&mdash;some simulation runs where momentum has larger drawdowns than growth&mdash;but these instances are few and far between. The overwhelming number of observations shows higher drawdowns for the growth monkeys than for the momentum monkeys.</p>
<?xmltex \OrgFixedPosition{c03-fig-0003}?>
<figure xml:id="c03-fig-0003">
<mediaResource href="urn:x-wiley:9781119237198:media:w9781119237198c03:c03f003" alt="image"/>
<caption>Drawdown: Growth Monkeys versus Momentum Monkeys</caption>
<?xmltex \pgtag{\bgroup\FloatPositionBottrue\putfigure{3}{c03/c03f003.eps}{}{}{}\egroup}?></figure>
<p xml:id="c03-para-0034">To summarize, growth investing, as measured by high price to fundamentals, is not the same as momentum investing, as measured by strong past relative performance. This conclusion is clearly seen in the historical characteristics associated with each of these strategies, which shows that growth and momentum are different animals.</p></section>
<?xmltex \pgtag{\vfill\eject}?>
<section xml:id="c03-sec-0007"><title type="main">But Why Does Momentum Work?</title><p xml:id="c03-para-0035"><?xmltex \OrgFixedPosition{c03-blkfxd-0002}?><?xmltex \pgtag{\Secfollowedepitrue}?><blockFixed type="standFirst" xml:id="c03-blkfxd-0002"><p xml:id="c03-para-0036">&ldquo;We discovered the world wasn&apos;t flat before we understood and agreed why.&rdquo;</p>
<source>&mdash;Cliff Asness et&nbsp;al.<link href="c03-note-0014"/></source>
</blockFixed></p>
<p xml:id="c03-para-0037"><?xmltex \pgtag{\looseness=-1}?>Cliff Asness&apos;s quote highlights that sometimes you can understand that something is true before you fully understand why it is true, and agree with others why it is true. So it goes with momentum investing, where the data are clear that it works, but we lack clarity on exactly &ldquo;why.&rdquo; We attempt to address this conundrum, but knowingly embrace the humility that our thoughts can only hope to be directionally correct, at best. Value investing, in contrast to momentum investing, is intuitive. The value approach is intuitive because it is assumed that market prices drift around a so&hyphen;called intrinsic value, which is informed by fundamentals. Classic value investors claim to earn their paycheck by timing the difference between fundamentals and market prices. But what if the market decides to never update their expectation about the intrinsic value of a firm (also known as a <i>value trap</i>)? Assuming free cash flow distributions are distributed in the distant future, a value investor won&apos;t win in this situation. The value investor, like all investors, needs market expectations to change in their favor for the strategy to work. Value investing doesn&apos;t work simply because the investor buys cheap. Value investing works because cheap price&hyphen;to&hyphen;fundamental ratios, the proxy for a systematic market expectation error, mean revert in favor of the value investor, on average. The core argument behind momentum investing works along the exact same lines. Momentum investing works because the relative strength indicator is a proxy for a systematic expectation error in the market that predictably reverts in the momentum investor&apos;s favor,<?xmltex \pgtag{\break{}}?> on average.</p>
<p xml:id="c03-para-0038">To understand why momentum works, we leverage our sustainable active framework to determine if a strategy will be successful over the long term. The building blocks to identify sustainable performance were as follows: (1) identify bad poker players, (2) understand the limitations of the best poker players to exploit the bad poker players, and (3) exploit the opportunity that presents itself. We showed that value, which has a strong historical track record, has characteristics that suggest the past track record could plausibly continue in the future.</p>
<p xml:id="c03-para-0039">Our analysis of the value anomaly through the lens of the sustainable active investing framework raises a natural question: Is momentum, like value, a sustainable investment approach? With our sustainable active framework in hand, we can tackle this difficult question. But first, we should establish beyond any doubt that momentum&mdash;which isn&apos;t the same as growth investing&mdash;has worked, historically. To keep things simple and in line with previous analysis, we consider &ldquo;momentum investing&rdquo; to be roughly approximated by the practice of purchasing portfolios of firms with strong relative performance over the past year. Using&thinsp;the momentum portfolio data from Ken French&apos;s website,<link href="c03-note-0015"/> we examine the returns from January 1, 1927, to December 31, 2014, for a high momentum portfolio (high momentum decile, value&hyphen;weight returns), a value portfolio (high B/M decile, value&hyphen;weight returns), a growth portfolio (low B/M decile, value&hyphen;weight returns), and the S&amp;P 500 total return index (SP500). Results are shown in Table<?xmltex \pgtag{\nobreak}?> <link href="c03-tbl-0001"/>. All returns are total returns and include the reinvestment of distributions (e.g., dividends). Results are gross of<?xmltex \pgtag{\nobreak}?> <?xmltex \pgtag{\hbox\bgroup}?>fees.<?xmltex \pgtag{\egroup}?></p>
<?xmltex \OrgFixedPosition{c03-tbl-0001}?>
<?xmltex \pgtag{\bgroup\tabbotskip=-3pt\FloatPositionBottrue}?><tabular xml:id="c03-tbl-0001"><title type="main">Momentum Performance (1927&ndash;2014)</title><table pgwide="1" frame="topbot" rowsep="0" colsep="0"><tgroup cols="5"><colspec colnum="1" colname="col1" align="left"/><colspec colnum="2" colname="col2" align="center"/><colspec colnum="3" colname="col3" align="center"/><colspec colnum="4" colname="col4" align="center"/><colspec colnum="5" colname="col5" align="center"/><lwtablebody><?xmltex \pgtag{\tabcolsep=0pt\begin{tabular*}{\textwidth}{@{\extracolsep\fill}ld{5}d{5}d{5}d{5}@{\extracolsep\fill}}\firsttablerule}?><thead valign="bottom"><!--<row rowsep="1">--><?xmltex \pgtag{\icolcnt=1\relax}?><entry colname="col1" xml:id="c03-ent-0001"></entry><entry colname="col2"  xml:id="c03-ent-0002" align="center" lwPstyle="TabularHead">Momentum</entry><entry colname="col3" align="center" xml:id="c03-ent-0003" lwPstyle="TabularHead">Value</entry><entry colname="col4" align="center" xml:id="c03-ent-0004" lwPstyle="TabularHead">Growth</entry><entry colname="col5" align="center" xml:id="c03-ent-0005" lwPstyle="TabularHead">SP500</entry><!--</row>--></thead><!--<tbody valign="top">--><!--<row>--><?xmltex \\\tablerule\pgtag{\icolcnt=1\relax}?><entry colname="col1" xml:id="c03-ent-0006"><b>CAGR</b></entry>
<entry colname="col2" xml:id="c03-ent-0007">16.85&percnt;</entry>
<entry colname="col3" xml:id="c03-ent-0008">12.41&percnt;</entry>
<entry colname="col4" xml:id="c03-ent-0009">8.70&percnt;</entry>
<entry colname="col5" xml:id="c03-ent-0010">9.95&percnt;</entry><!--</row>-->
<!--<row>--><?xmltex \\\pgtag{\icolcnt=1\relax}?><entry colname="col1" xml:id="c03-ent-0011"><b>Standard Deviation</b></entry>
<entry colname="col2" xml:id="c03-ent-0012">22.61&percnt;</entry>
<entry colname="col3" xml:id="c03-ent-0013">31.92&percnt;</entry>
<entry colname="col4" xml:id="c03-ent-0014">19.95&percnt;</entry>
<entry colname="col5" xml:id="c03-ent-0015">19.09&percnt;</entry><!--</row>-->
<!--<row>--><?xmltex \\\pgtag{\icolcnt=1\relax}?><entry colname="col1" xml:id="c03-ent-0016"><b>Downside Deviation</b></entry>
<entry colname="col2" xml:id="c03-ent-0017">16.71&percnt;</entry>
<entry colname="col3" xml:id="c03-ent-0018">21.34&percnt;</entry>
<entry colname="col4" xml:id="c03-ent-0019">14.41&percnt;</entry>
<entry colname="col5" xml:id="c03-ent-0020">14.22&percnt;</entry><!--</row>-->
<!--<row>--><?xmltex \\\pgtag{\icolcnt=1\relax}?><entry colname="col1" xml:id="c03-ent-0021"><b>Sharpe Ratio</b></entry>
<entry colname="col2" xml:id="c03-ent-0022">0.66</entry>
<entry colname="col3" xml:id="c03-ent-0023">0.41</entry>
<entry colname="col4" xml:id="c03-ent-0024">0.35</entry>
<entry colname="col5" xml:id="c03-ent-0025">0.41</entry><!--</row>-->
<!--<row>--><?xmltex \\\pgtag{\icolcnt=1\relax}?><entry colname="col1" xml:id="c03-ent-0026"><b>Sortino Ratio (MAR</b> <math display="inline" overflow="scroll" xmlns="http://www.w3.org/1998/Math/MathML" xmlns:xlink="http://www.w3.org/1999/xlink"><mrow><mo mathvariant="bold">=</mo></mrow></math> <b>5&percnt;)</b></entry>
<entry colname="col2" xml:id="c03-ent-0027">0.79</entry>
<entry colname="col3" xml:id="c03-ent-0028">0.54</entry>
<entry colname="col4" xml:id="c03-ent-0029">0.37</entry>
<entry colname="col5" xml:id="c03-ent-0030">0.45</entry><!--</row>-->
<!--<row>--><?xmltex \\\pgtag{\icolcnt=1\relax}?><entry colname="col1" xml:id="c03-ent-0031"><b>Worst Drawdown</b></entry>
<entry colname="col2" xml:id="c03-ent-0032">&ndash;76.95&percnt;</entry>
<entry colname="col3" xml:id="c03-ent-0033">&ndash;91.67&percnt;</entry>
<entry colname="col4" xml:id="c03-ent-0034">&ndash;85.01&percnt;</entry>
<entry colname="col5" xml:id="c03-ent-0035">&ndash;84.59&percnt;</entry><!--</row>-->
<!--<row>--><?xmltex \\\pgtag{\icolcnt=1\relax}?><entry colname="col1" xml:id="c03-ent-0036"><b>Worst Month Return</b></entry>
<entry colname="col2" xml:id="c03-ent-0037">&ndash;28.52&percnt;</entry>
<entry colname="col3" xml:id="c03-ent-0038">&ndash;43.98&percnt;</entry>
<entry colname="col4" xml:id="c03-ent-0039">&ndash;30.65&percnt;</entry>
<entry colname="col5" xml:id="c03-ent-0040">&ndash;28.73&percnt;</entry><!--</row>-->
<!--<row>--><?xmltex \\\pgtag{\icolcnt=1\relax}?><entry colname="col1" xml:id="c03-ent-0041"><b>Best Month Return</b></entry>
<entry colname="col2" xml:id="c03-ent-0042">28.88&percnt;</entry>
<entry colname="col3" xml:id="c03-ent-0043">98.65&percnt;</entry>
<entry colname="col4" xml:id="c03-ent-0044">42.16&percnt;</entry>
<entry colname="col5" xml:id="c03-ent-0045">41.65&percnt;</entry><!--</row>-->
<!--<row>--><?xmltex \\\pgtag{\icolcnt=1\relax}?><entry colname="col1" xml:id="c03-ent-0046"><b>Profitable Months</b></entry>
<entry colname="col2" xml:id="c03-ent-0047">63.16&percnt;</entry>
<entry colname="col3" xml:id="c03-ent-0048">60.51&percnt;</entry>
<entry colname="col4" xml:id="c03-ent-0049">59.09&percnt;</entry>
<entry colname="col5" xml:id="c03-ent-0050">61.74&percnt;</entry><!--</row>-->
<?xmltex \pgtag{\\ \lasttablerule\end{tabular*}}?><!--</tbody>-->
</lwtablebody></tgroup>
</table>
</tabular><?xmltex \pgtag{\egroup}?>
<p xml:id="c03-para-0040">Momentum stocks have outperformed value stocks, growth stocks, and the broader market by a large margin. The portfolio of momentum stocks earns a compound annual growth rate of 16.85 percent per year, whereas, the growth stock portfolio earns 8.70 percent per year&mdash;an 8 percent annual spread in performance. This historical spread in returns is why momentum has been deemed the premier anomaly by academic researchers. There are obviously important considerations we avoid at this stage of our discussion, such as transaction costs, but one fact is clear&mdash;<i>momentum is the performance king.</i> Next question, is this performance sustainable?</p><section xml:id="c03-sec-0008"><title type="main">Are Bad Players Creating the<?xmltex \pgtag{\protect\nobreak}?> Momentum Anomaly?</title><p xml:id="c03-para-0041">With value, the core behavioral bias described was representative bias, which drove a price overreaction to poor fundamentals that mean&hyphen;revert over time. This description is, of course, an oversimplification of the psychological factors at work, but the collective academic evidence generally seems to support the core thesis that the excess returns earned by value stocks are not solely driven by additional risk&mdash;mispricing plays some role in describing the excess returns. With momentum, the collective evidence points in the same direction as value&mdash;risk certainly plays some role in explaining the excess returns, but mispricing plays a role as well. The behavioral premise for momentum is that investors seem to <i>underreact</i> to positive news reflected in the strong relative performance. On the face of it, the behavior driving value and momentum appear to contradict one another: Value is driven by an overreaction problem, while momentum is driven by an underreaction problem. What gives?</p>
<p xml:id="c03-para-0042">A valid critique of behavioral finance researchers is that they want to have their cake and eat it, too. In one instance we can lean on underreaction bias and in the next instance we can lean on the overreaction bias. The behavioral formula is too easy: (1) Grab a psychology textbook, and (2) identify behavioral biases that fit the data. Eugene Fama issued a challenge to so&hyphen;called behavioral finance researchers in his 1998 paper &ldquo;Market Efficiency, Long&hyphen;Term Returns, and Behavioral Finance:&rdquo;<link href="c03-note-0016"/><?xmltex \OrgFixedPosition{c03-feafxd-0201}?>
<featureFixed xml:id="c03-feafxd-0201" lwtype="Extract"><title type="featureFixedName">Extract</title><p xml:id="c03-para-0043">Following the standard scientific rule, market efficiency can only be replaced by a better model&hellip;The alternative has a daunting task. It must specify what it is about investor psychology that causes simultaneous underreaction to some types of events and overreaction to others&hellip;</p>
</featureFixed></p>
<p xml:id="c03-para-0044">Three sets of authors in three different papers<link href="c03-note-0017"/><sup>&ndash;</sup><link href="c03-note-0019"/> immediately took on the challenge. Daniel et&nbsp;al. and Barberis et&nbsp;al. focus on models driven by documented psychological biases to derive predictions that hypothesize excess returns for both value and momentum strategies. Hong and Stein also tackle the problem, but from a slightly different angle. Whereas Daniel et&nbsp;al. and Barberis et&nbsp;al. focus on investor psychology issues for individual market participants, Hong and Stein focus on the interaction of different market participants, which are assumed to either be fundamental or technical investors, but few investors are both fundamental and technical. We recommend that interested readers explore all of these papers since all three theories probably play some role in explaining momentum, but we pay particular attention to the Barberis et&nbsp;al. paper because it is arguably the approach with the most empirical support.<link href="c03-note-0020"/></p>
<p xml:id="c03-para-0045">Barberis et&nbsp;al. conclude that value and momentum are driven by biases that mirror one another. Value, as discussed previously, is driven by an overreaction problem, in which humans are too quick to draw conclusions from a small amount of recent data. In contrast, momentum is driven by an underreaction issue, which is the opposite of overreaction. With underreaction, humans are slow to update their views based on new evidence, which could be due to a systematic behavior bias and/or due to the fact human beings simply have limited cognitive power (i.e., &ldquo;limited attention&rdquo; as it is called in academic literature). But what drives overreaction in one circumstance and underreaction in another?</p>
<p xml:id="c03-para-0046"><?xmltex \pgtag{\looseness=-1}?>The challenge with any behavioral theory is in understanding what triggers overreaction and what triggers underreaction; in other words, why do market participants engage in behavioral &ldquo;regime&hyphen;shifting,&rdquo; and can we understand how and why they do this? Barberis et&nbsp;al. rely on Griffin and Tversky&apos;s work,<link href="c03-note-0021"/> which leads them to assume that good earnings news, presented <i>outside</i> of, or in isolation from, a long sequence of good earnings news, leads to an underreaction (i.e., conservatism) and that good<?xmltex \pgtag{\nobreak}?> earnings<?xmltex \pgtag{\nobreak}?> news, presented <i>inside of,</i> or as part of, a long sequences of good earnings news, lead to an overreaction (i.e., representativeness). Experimental evidence strongly supports the Barberis et&nbsp;al. theories. In 2002, Robert Bloomfield and Jeffrey Hales conducted controlled trading experiments with Cornell MBA students and found that business students engage in behavioral regime&hyphen;shifting, depending on how they perceive new information.<link href="c03-note-0022"/></p>
<p xml:id="c03-para-0047">Let&apos;s put this issue in concrete, practical terms. Take a company with a long string of positive earnings announcements. What happens when investors see another strong positive earnings announcement? Investors will predict the trend will continue, since this conclusion is representative of the ongoing earnings trend observed. But investors overreact. They become overly optimistic and bullish, and because they expect this strong earnings growth to continue to occur in the future, they bid up the stock to excessively high levels that become disconnected from fundamentals. At this point, if there is a negative earnings event, investors are stunned since this event is inconsistent with their optimism, and they sell, causing prices to decline. This behavior is growth investing.</p>
<p xml:id="c03-para-0048">Now take a company with a more uneven recent earnings history. What happens in this scenario when investors see a positive earnings surprise? They are skeptical, conservative, and slow to update their beliefs. They are hesitant to be bullish. After all, what if this is just a proverbial blip on the radar, and future earnings will not be similarly strong? In this case, investors can be said to underreact to strong earnings and underweight the information content of the recent earnings. They are overly pessimistic and will not bid the stock price higher, even though the news may still suggest it is undervalued. Only over time do prices increase to fully reflect the new fundamental information. This behavior is momentum investing.</p>
<p xml:id="c03-para-0049">The conclusion by Barberis et&nbsp;al. is that both value and momentum effects are plausible under a wide variety of parameter values (i.e., either underreaction or overreaction clearly seems to prevail depending on recent trends of historical earnings over several periods).</p>
<?xmltex \pgtag{\vfill\eject}?>
<p xml:id="c03-para-0050">Frankly, there is no definitive conclusion on the behavioral biases that drive both value and momentum, and maybe there never will be. There does, however, seem to be a general consensus that both of these anomalies are driven, in part, by mispricing due to behavioral bias. Value and momentum signals are simply a proxy for behavioral biases that drive systematic investor expectation errors. The empirical evidence strongly supports this hypothesis and is codified in the title of a 2013 paper by Asness, Moskowitz, and Pedersen, &ldquo;Value and Momentum Everywhere.&rdquo;<link href="c03-note-0023"/></p>
<p xml:id="c03-para-0051">In a way, maybe we shouldn&apos;t be too concerned with the specific mechanism that causes the poor players to contribute to an anomaly like value or momentum. Maybe it doesn&apos;t matter that we all understand and agree on why momentum or value work. As investors we just care that it works. And since momentum seems to work well, and we have covered how poor players contribute to its cause, now we must address a basic question: Why hasn&apos;t the smart money already arbitraged the anomaly away?</p></section>
<section xml:id="c03-sec-0009"><title type="main">What Do the<?xmltex \pgtag{\protect\nobreak}?> Best Players Think about Momentum?</title><p xml:id="c03-para-0052">Similar to value investing, momentum investing requires a level of discipline that few investors possess. Momentum does not work all the time and can <i>fail spectacularly</i>. This harsh reality prevents many large pools of capital from dipping their toe too far into the momentum pool. Momentum is simply too dangerous.</p>
<p xml:id="c03-para-0053">To make the point that momentum can sometimes be hazardous to your wealth, we examine the&thinsp;pain&thinsp;of momentum investing during the 2008 financial crisis and the follow&hyphen;on period. We examine the returns from January 1, 2008, to December 31, 2009, for a momentum portfolio (high momentum decile, value&hyphen;weight returns), a growth portfolio (low B/M decile, value&hyphen;weight returns), a value portfolio (high B/M decile, value&hyphen;weight returns), and the S&amp;P 500 total return index (SP500). Results are shown in Table<?xmltex \pgtag{\nobreak}?> <link href="c03-tbl-0002"/>. All returns are total returns and include the reinvestment of distributions (e.g., dividends). Results are gross of<?xmltex \pgtag{\nobreak}?> <?xmltex \pgtag{\hbox\bgroup}?>fees.<?xmltex \pgtag{\egroup}?></p>
<?xmltex \OrgFixedPosition{c03-tbl-0002}?>
<?xmltex \pgtag{\bgroup\FloatPositionToptrue}?><tabular xml:id="c03-tbl-0002"><title type="main">Momentum Investing Can Underperform (2008&ndash;2009)</title><table pgwide="1" frame="topbot" rowsep="0" colsep="0"><tgroup cols="5"><colspec colnum="1" colname="col1" align="left"/><colspec colnum="2" colname="col2" align="center"/><colspec colnum="3" colname="col3" align="center"/><colspec colnum="4" colname="col4" align="center"/><colspec colnum="5" colname="col5" align="center"/><lwtablebody><?xmltex \pgtag{\tabcolsep=0pt\begin{tabular*}{\textwidth}{@{\extracolsep\fill}ld{4}d{4}d{4}d{4}@{\extracolsep\fill}}\firsttablerule}?><thead valign="bottom"><!--<row rowsep="1">--><?xmltex \pgtag{\icolcnt=1\relax}?><entry colname="col1" align="center" xml:id="c03-ent-0051"></entry><entry colname="col2" align="center" xml:id="c03-ent-0052" lwPstyle="TabularHead">Momentum</entry><entry colname="col3" align="center"  xml:id="c03-ent-0053" lwPstyle="TabularHead">Growth</entry><entry colname="col4" align="center" xml:id="c03-ent-0054" lwPstyle="TabularHead">Value</entry><entry colname="col5" align="center" xml:id="c03-ent-0055" lwPstyle="TabularHead">SP500</entry><!--</row>--></thead><!--<tbody valign="top">--><!--<row>--><?xmltex \\\tablerule\pgtag{\icolcnt=1\relax}?><entry colname="col1" xml:id="c03-ent-0056"><b>CAGR</b></entry>
<entry colname="col2" xml:id="c03-ent-0057">&ndash;17.65&percnt;</entry>
<entry colname="col3" xml:id="c03-ent-0058">&ndash;8.52&percnt;</entry>
<entry colname="col4" xml:id="c03-ent-0059">&ndash;6.69&percnt;</entry>
<entry colname="col5" xml:id="c03-ent-0060">&ndash;10.36&percnt;</entry><!--</row>-->
<!--<row>--><?xmltex \\\pgtag{\icolcnt=1\relax}?><entry colname="col1" xml:id="c03-ent-0061"><b>Standard Deviation</b></entry>
<entry colname="col2" xml:id="c03-ent-0062">26.03&percnt;</entry>
<entry colname="col3" xml:id="c03-ent-0063">23.45&percnt;</entry>
<entry colname="col4" xml:id="c03-ent-0064">45.60&percnt;</entry>
<entry colname="col5" xml:id="c03-ent-0065">23.24&percnt;</entry><!--</row>-->
<!--<row>--><?xmltex \\\pgtag{\icolcnt=1\relax}?><entry colname="col1" xml:id="c03-ent-0066"><b>Downside Deviation</b></entry>
<entry colname="col2" xml:id="c03-ent-0067">20.67&percnt;</entry>
<entry colname="col3" xml:id="c03-ent-0068">17.38&percnt;</entry>
<entry colname="col4" xml:id="c03-ent-0069">23.06&percnt;</entry>
<entry colname="col5" xml:id="c03-ent-0070">17.37&percnt;</entry><!--</row>-->
<!--<row>--><?xmltex \\\pgtag{\icolcnt=1\relax}?><entry colname="col1" xml:id="c03-ent-0071"><b>Sharpe Ratio</b></entry>
<entry colname="col2" xml:id="c03-ent-0072">&ndash;0.64</entry>
<entry colname="col3" xml:id="c03-ent-0073">&ndash;0.30</entry>
<entry colname="col4" xml:id="c03-ent-0074">0.05</entry>
<entry colname="col5" xml:id="c03-ent-0075">&hyphen;0.39</entry><!--</row>-->
<!--<row>--><?xmltex \\\pgtag{\icolcnt=1\relax}?><entry colname="col1" xml:id="c03-ent-0076"><b>Sortino Ratio (MAR</b> <math display="inline" overflow="scroll" xmlns="http://www.w3.org/1998/Math/MathML" xmlns:xlink="http://www.w3.org/1999/xlink"><mrow><mo mathvariant="bold">=</mo></mrow></math> <b>5&percnt;)</b></entry>
<entry colname="col2" xml:id="c03-ent-0077">&ndash;1.01</entry>
<entry colname="col3" xml:id="c03-ent-0078">&ndash;0.64</entry>
<entry colname="col4" xml:id="c03-ent-0079">&ndash;0.09</entry>
<entry colname="col5" xml:id="c03-ent-0080">&hyphen;0.76</entry><!--</row>-->
<!--<row>--><?xmltex \\\pgtag{\icolcnt=1\relax}?><entry colname="col1" xml:id="c03-ent-0081"><b>Worst Drawdown</b></entry>
<entry colname="col2" xml:id="c03-ent-0082">&ndash;51.25&percnt;</entry>
<entry colname="col3" xml:id="c03-ent-0083">&ndash;46.72&percnt;</entry>
<entry colname="col4" xml:id="c03-ent-0084">&ndash;61.04&percnt;</entry>
<entry colname="col5" xml:id="c03-ent-0085">&ndash;47.75&percnt;</entry><!--</row>-->
<!--<row>--><?xmltex \\\pgtag{\icolcnt=1\relax}?><entry colname="col1" xml:id="c03-ent-0086"><b>Worst Month Return</b></entry>
<entry colname="col2" xml:id="c03-ent-0087">&ndash;15.19&percnt;</entry>
<entry colname="col3" xml:id="c03-ent-0088">&ndash;16.13&percnt;</entry>
<entry colname="col4" xml:id="c03-ent-0089">&ndash;28.07&percnt;</entry>
<entry colname="col5" xml:id="c03-ent-0090">&ndash;16.70&percnt;</entry><!--</row>-->
<!--<row>--><?xmltex \\\pgtag{\icolcnt=1\relax}?><entry colname="col1" xml:id="c03-ent-0091"><b>Best Month Return</b></entry>
<entry colname="col2" xml:id="c03-ent-0092">11.09&percnt;</entry>
<entry colname="col3" xml:id="c03-ent-0093">9.92&percnt;</entry>
<entry colname="col4" xml:id="c03-ent-0094">36.64&percnt;</entry>
<entry colname="col5" xml:id="c03-ent-0095">9.42&percnt;</entry><!--</row>-->
<!--<row>--><?xmltex \\\pgtag{\icolcnt=1\relax}?><entry colname="col1" xml:id="c03-ent-0096"><b>Profitable Months</b></entry>
<entry colname="col2" xml:id="c03-ent-0097">50.00&percnt;</entry>
<entry colname="col3" xml:id="c03-ent-0098">54.17&percnt;</entry>
<entry colname="col4" xml:id="c03-ent-0099">62.50&percnt;</entry>
<entry colname="col5" xml:id="c03-ent-0100">54.17&percnt;</entry><!--</row>-->
<?xmltex \pgtag{\\ \lasttablerule\end{tabular*}}?><!--</tbody>-->
</lwtablebody></tgroup>
</table>
</tabular><?xmltex \pgtag{\egroup}?>
<p xml:id="c03-para-0054">On a relative basis, momentum underperformed by a substantial amount. When we look at risk&hyphen;adjusted statistics the performance is even worse. Clearly, following an active momentum strategy involves strong elements of investment advisor career risk, akin to active value strategies. But it gets worse &hellip;</p>
<p xml:id="c03-para-0055">In Table<?xmltex \pgtag{\nobreak}?> <link href="c03-tbl-0003"/> we examine returns over the financial crisis period and we include the follow&hyphen;on period: January 1, 2008, to December 31, 2014. Results are gross of fees. Simple passive index funds outperform momentum over a seven&hyphen;year period!<link href="c03-note-0024"/></p>
<?xmltex \OrgFixedPosition{c03-tbl-0003}?>
<?xmltex \pgtag{\bgroup\FloatPositionToptrue}?><tabular xml:id="c03-tbl-0003"><title type="main">Momentum Investing Can Underperform (2008&ndash;2014)</title><table pgwide="1" frame="topbot" rowsep="0" colsep="0"><tgroup cols="5"><colspec colnum="1" colname="col1" align="left"/><colspec colnum="2" colname="col2" align="center"/><colspec colnum="3" colname="col3" align="center"/><colspec colnum="4" colname="col4" align="center"/><colspec colnum="5" colname="col5" align="center"/><lwtablebody><?xmltex \pgtag{\tabcolsep=0pt\begin{tabular*}{\textwidth}{@{\extracolsep\fill}ld{3.4}d{3.4}d{3.4}d{3.4}@{\extracolsep\fill}}\firsttablerule}?><thead valign="bottom"><!--<row rowsep="1">--><?xmltex \pgtag{\icolcnt=1\relax}?><entry colname="col1" align="center" xml:id="c03-ent-0101"></entry><entry colname="col2" xml:id="c03-ent-0102" align="center" lwPstyle="TabularHead">Momentum</entry><entry colname="col3" align="center" xml:id="c03-ent-0103" lwPstyle="TabularHead">Growth</entry><entry colname="col4" align="center" xml:id="c03-ent-0104" lwPstyle="TabularHead">Value</entry><entry colname="col5" align="center" xml:id="c03-ent-0105" lwPstyle="TabularHead">SP500</entry><!--</row>--></thead><!--<tbody valign="top">--><!--<row>--><?xmltex \\\tablerule\pgtag{\icolcnt=1\relax}?><entry colname="col1" xml:id="c03-ent-0106"><b>CAGR</b></entry>
<entry colname="col2" xml:id="c03-ent-0107">6.55&percnt;</entry>
<entry colname="col3" xml:id="c03-ent-0108">8.69&percnt;</entry>
<entry colname="col4" xml:id="c03-ent-0109">8.45&percnt;</entry>
<entry colname="col5" xml:id="c03-ent-0110">7.44&percnt;</entry><!--</row>-->
<!--<row>--><?xmltex \\\pgtag{\icolcnt=1\relax}?><entry colname="col1" xml:id="c03-ent-0111"><b>Standard Deviation</b></entry>
<entry colname="col2" xml:id="c03-ent-0112">22.24&percnt;</entry>
<entry colname="col3" xml:id="c03-ent-0113">17.13&percnt;</entry>
<entry colname="col4" xml:id="c03-ent-0114">29.73&percnt;</entry>
<entry colname="col5" xml:id="c03-ent-0115">16.75&percnt;</entry><!--</row>-->
<!--<row>--><?xmltex \\\pgtag{\icolcnt=1\relax}?><entry colname="col1" xml:id="c03-ent-0116"><b>Downside Deviation</b></entry>
<entry colname="col2" xml:id="c03-ent-0117">17.03&percnt;</entry>
<entry colname="col3" xml:id="c03-ent-0118">12.92&percnt;</entry>
<entry colname="col4" xml:id="c03-ent-0119">20.78&percnt;</entry>
<entry colname="col5" xml:id="c03-ent-0120">13.30&percnt;</entry><!--</row>-->
<!--<row>--><?xmltex \\\pgtag{\icolcnt=1\relax}?><entry colname="col1" xml:id="c03-ent-0121"><b>Sharpe Ratio</b></entry>
<entry colname="col2" xml:id="c03-ent-0122">0.39</entry>
<entry colname="col3" xml:id="c03-ent-0123">0.56</entry>
<entry colname="col4" xml:id="c03-ent-0124">0.41</entry>
<entry colname="col5" xml:id="c03-ent-0125">0.50</entry><!--</row>-->
<!--<row>--><?xmltex \\\pgtag{\icolcnt=1\relax}?><entry colname="col1" xml:id="c03-ent-0126"><b>Sortino Ratio (MAR</b> <math display="inline" overflow="scroll" xmlns="http://www.w3.org/1998/Math/MathML" xmlns:xlink="http://www.w3.org/1999/xlink"><mrow><mo mathvariant="bold">=</mo></mrow></math> <b>5&percnt;)</b></entry>
<entry colname="col2" xml:id="c03-ent-0127">0.23</entry>
<entry colname="col3" xml:id="c03-ent-0128">0.37</entry>
<entry colname="col4" xml:id="c03-ent-0129">0.36</entry>
<entry colname="col5" xml:id="c03-ent-0130">0.27</entry><!--</row>-->
<!--<row>--><?xmltex \\\pgtag{\icolcnt=1\relax}?><entry colname="col1" xml:id="c03-ent-0131"><b>Worst Drawdown</b></entry>
<entry colname="col2" xml:id="c03-ent-0132">&ndash;51.25&percnt;</entry>
<entry colname="col3" xml:id="c03-ent-0133">&ndash;46.72&percnt;</entry>
<entry colname="col4" xml:id="c03-ent-0134">&ndash;61.04&percnt;</entry>
<entry colname="col5" xml:id="c03-ent-0135">&ndash;47.75&percnt;</entry><!--</row>-->
<!--<row>--><?xmltex \\\pgtag{\icolcnt=1\relax}?><entry colname="col1" xml:id="c03-ent-0136"><b>Worst Month Return</b></entry>
<entry colname="col2" xml:id="c03-ent-0137">&ndash;15.91&percnt;</entry>
<entry colname="col3" xml:id="c03-ent-0138">&ndash;16.13&percnt;</entry>
<entry colname="col4" xml:id="c03-ent-0139">&ndash;28.07&percnt;</entry>
<entry colname="col5" xml:id="c03-ent-0140">&ndash;16.70&percnt;</entry><!--</row>-->
<!--<row>--><?xmltex \\\pgtag{\icolcnt=1\relax}?><entry colname="col1" xml:id="c03-ent-0141"><b>Best Month Return</b></entry>
<entry colname="col2" xml:id="c03-ent-0142">14.93&percnt;</entry>
<entry colname="col3" xml:id="c03-ent-0143">11.21&percnt;</entry>
<entry colname="col4" xml:id="c03-ent-0144">36.64&percnt;</entry>
<entry colname="col5" xml:id="c03-ent-0145">10.93&percnt;</entry><!--</row>-->
<!--<row>--><?xmltex \\\pgtag{\icolcnt=1\relax}?><entry colname="col1" xml:id="c03-ent-0146"><b>Profitable Months</b></entry>
<entry colname="col2" xml:id="c03-ent-0147">61.90&percnt;</entry>
<entry colname="col3" xml:id="c03-ent-0148">61.90&percnt;</entry>
<entry colname="col4" xml:id="c03-ent-0149">59.52&percnt;</entry>
<entry colname="col5" xml:id="c03-ent-0150">63.10&percnt;</entry><!--</row>-->
<?xmltex \pgtag{\\ \lasttablerule\end{tabular*}}?><!--</tbody>-->
</lwtablebody></tgroup>
</table>
</tabular><?xmltex \pgtag{\egroup}?>
<?xmltex \pgtag{\vfill\eject}?>
<p xml:id="c03-para-0056">Ask yourself the same question we posed with the results from a<?xmltex \pgtag{\break}?> hypothetical value investor from 1994 to 1999:
<?xmltex \OrgFixedPosition{c03-feafxd-0001}?>
<featureFixed xml:id="c03-feafxd-0001" lwtype="Extract"><title type="featureFixedName">Extract</title><p xml:id="c03-para-0057">If your asset manager underperformed a benchmark for seven years, at times by double digits, would you fire them?</p>
</featureFixed></p>
<p xml:id="c03-para-0058">The answer is a resounding &ldquo;Yes!&rdquo; for most investors, which translates into a resounding, &ldquo;No way, Jose!&rdquo; for professional asset managers concerned about their careers. But the market frictions associated with exploiting a momentum strategy extend beyond career risk. Unlike value, which is a strategy that works when traded relatively infrequently (e.g., annual rebalanced portfolios have excess risk&hyphen;adjusted returns), momentum is a strategy that requires a higher degree of trading frequency to be effective (e.g., quarterly rebalanced portfolios have excess risk&hyphen;adjusted returns, but annually rebalanced portfolios do not). This trading frequency increases transaction costs, which can be prohibitive and reduce the profitability of the strategy, net of trading costs. While a plausible limit of arbitrage, Frazzini et&nbsp;al. address this question directly using data from over a trillion dollars in live transactions from AQR Capital Management and find that transaction costs for efficient institutional investors cannot &ldquo;explain away&rdquo; their unwillingness to pursue momentum strategies.<link href="c03-note-0025"/></p></section>
<section xml:id="c03-sec-0010"><title type="main">Momentum Is Similar to<?xmltex \pgtag{\protect\nobreak}?> Value, Not Growth</title><p xml:id="c03-para-0059">Momentum investing turns out to be more similar to value than to growth from a performance perspective and from the viewpoint of our sustainable active investing framework. On the performance front, both value and momentum have strong historical risk&hyphen;adjusted returns and have been extensively tested by academic researchers across different markets, asset classes, and time periods. To explain this anomalous performance, the academic consensus suggests that value and momentum premiums are driven by some combination of hidden systematic risk&hyphen;factors (a justified reason for higher expected returns) and elements of mispricing (an unwarranted reason for higher expected returns). On the mispricing front, value and momentum metrics serve as signals to identify stocks suffering from market expectations that eventually move in favor of the value or momentum investor. This mispricing aspect is paired with the harsh reality that there is a limited ability for large pools of smart money to arbitrage away value or momentum. Many of these capital pools are conflicted by the high volatility and extreme career risk associated with pursuing active value and momentum strategies. Presumably, value and momentum premiums will continue to have staying power, under the assumptions that (1) value and momentum are fundamentally riskier strategies, (2) investors continue to suffer from behavioral bias, and (3) large&hyphen;scale arbitrage is costly and difficult.<link href="c03-note-0026"/></p></section>
</section>
<section type="summary" xml:id="c03-sec-0011"><title type="main">Summary</title><p xml:id="c03-para-0060">In this chapter, we explored the history of momentum research from its early days as a respectable approach, through the dark ages following the golden age of the EMH, and the more recent resurgence in academic interest. We then explore the common misperception that growth investing, as defined as buying stocks with high prices to fundamentals, is the same as momentum investing, which is buying stocks with strong relative returns. Nothing could be further from the truth. Buying expensive stocks is not the same as buying stocks with strong relative performance: one strategy performs, the other does not. Next, we investigated momentum through the lens of sustainable active investing. Momentum strategy excess expected returns are plausibly driven by investor behavioral errors, combined with limits of arbitrage, and thus reasonably support an argument for long&hyphen;term sustainable performance. Assuming we have convinced you that momentum, like value, is arguably a sustainable anomaly, we now explore why these two particular anomalies are <i>really</i> interesting when used together. In the next chapter, we will explore why all portfolios should consider combining value and momentum systems.</p></section>
<?xmltex \pgtag{\tablenotecnt=6\def\itemwd{16.}}?><noteGroup xml:id="c03-ntgp-0001"><title type="main">Notes</title>
<note xml:id="c03-note-0001">Robert Bloch, <i>The Warren Buffett Bible</i> (New York: Skyhorse Publishing, 2015).</note>
<note xml:id="c03-note-0002">Robert Levy, &ldquo;Relative Strength as a Criterion for Investment Selection,&rdquo; <i>The Journal of Finance</i> 22 (1967): 595&ndash;610.</note>
<note xml:id="c03-note-0003">Burton G. Malkiel, <i>A Random Walk Down Wall Street</i> (New York: Norton, 1973).</note>
<note xml:id="c03-note-0004">Jensen and Bennington take Levy&apos;s work to task in &ldquo;Random Walks and Technical Theories: Some Additional Evidence,&rdquo; <i>The Journal of Finance</i> 25 (1970): 469&ndash;482.</note>
<note xml:id="c03-note-0005">Mark Carhart, &ldquo;On Persistence in Mutual Fund Performance,&rdquo; <i>The Journal of Finance</i> 52 (1997): 57&ndash;82.</note>
<note xml:id="c03-note-0006">Eugene F. Fama and Kenneth R. French, &ldquo;Dissecting Anomalies,&rdquo; <i>Journal of Financial Economics</i> 63 (2008): 1653&ndash;1678.</note>
<note xml:id="c03-note-0007">Ibid.</note>
<note xml:id="c03-note-0008">Justin Birru, &ldquo;Confusion of Confusions: A Test of the Disposition Effect and Momentum,&rdquo; <i>The Review of Financial Studies</i> 28 (2015): 1849&ndash;1873.</note>
<note xml:id="c03-note-0009">Thomas George and Chuan&hyphen;Yang Hwang, &ldquo;The 52&hyphen;Week High and Momentum Investing,&rdquo; <i>The Journal of Finance</i> 59 (2004): 2145&ndash;2176.</note>
<note xml:id="c03-note-0010">See James Davis, &ldquo;The Cross&hyphen;Section of Realized Stock Returns: The Pre&hyphen;COMPUSTAT Evidence,&rdquo; <i>The Journal of Finance</i> 49 (1994), 1579&ndash;1593, and Eugene F. Fama and Kenneth R. French, &ldquo;Size and Book&hyphen;to&hyphen;Market Factors in Earnings and Returns,&rdquo; <i>The Journal of Finance</i> 50 (1995): 131&ndash;155.</note>
<note xml:id="c03-note-0011">See Kent Daniel and Sheridan Titman, &ldquo;Evidence on the Characteristics of Cross Sectional Variation in Stock Returns,&rdquo; <i>The Journal of Finance</i> 52 (1997), 1&ndash;33, Joseph Piotroski and Eric So, &ldquo;Identifying Expectation Errors in Value/Glamour Strategies: A Fundamental Analysis Approach,&rdquo; <i>The Review of Financial Studies</i> 25 (2012), 2841&ndash;2875, and Jack Vogel, &ldquo;Essays on Empirical Asset Pricing,&rdquo; dissertation Drexel University, 2014.</note>
<note xml:id="c03-note-0012">Beaver, William, Maureen McNichols, and Richard Price, &ldquo;Delisting Returns and Their Effect on Accounting&hyphen;based Market Anomalies,&rdquo; <i>Journal of Accounting and Economics</i> 43 (2007): 341&ndash;368.</note>
<note xml:id="c03-note-0013">Fama and French.</note>
<note xml:id="c03-note-0014">Cliff Asness, Andrea Frazzini, Ron Israel, and Toby Moskowitz, &ldquo;Fact, Fiction, and Momentum Investing,&rdquo; <i>The Journal of Portfolio Management</i> 40 (2014): 75&ndash;92.</note>
<note xml:id="c03-note-0015">mba.tuck.dartmouth.edu/pages/faculty/ken.french/data&uscore;library.html, accessed 12/30/2015.</note>
<note xml:id="c03-note-0016">Eugene F. Fama, &ldquo;Market Efficiency, Long&hyphen;Term Returns, and Behavioral Finance,&rdquo; <i>Journal of Financial Economics</i> 49 (1998): 283&ndash;306.</note>
<note xml:id="c03-note-0017">Nicholas Barberis, Andrei Shleifer, and Robert Vishny, &ldquo;A Model of Investor Sentiment,&rdquo; <i>Journal of Financial Economics</i> 49 (1998): 307&ndash;343.</note>
<note xml:id="c03-note-0018">Harrison Hong and Jeremy Stein, &ldquo;A Unified Theory of Underreaction, Momentum Trading, and Overreaction in Asset Markets,&rdquo; <i>The Journal of Finance</i> 54 (1999): 2143&ndash;2184.</note>
<note xml:id="c03-note-0019">Kent Daniel, David Hirshleifer, and Avanidhar Subrahmanyam, &ldquo;A Theory of Overconfidence, Self&hyphen;Attribution, and Security Market Under&hyphen; and Over&hyphen;Reactions,&rdquo; <i>The Journal of Finance</i> 53 (1998): 1839&ndash;1885.</note>
<note xml:id="c03-note-0020">Although, there is research that goes against the Barberis et&nbsp;al. theory. For example, Alexander Hillert, Heiko Jacobs, and Sebastian Muller, &ldquo;Media Makes Momentum,&rdquo; <i>The Review of Financial Studies</i> 27 (2014): 3467&ndash;3501. The reality is that all of these theories probably play some role in explaining the insurmountable evidence related to momentum, and momentum is clearly created by some sort of behavioral bias, even if it doesn&apos;t take the form of the version outlined in Barberis et&nbsp;al.</note>
<note xml:id="c03-note-0021">Dale Griffin and Amos Tversky, &ldquo;The Weighing of Evidence and the Determinants of Confidence,&rdquo; <i>Cognitive Psychology</i> 24 (1992): 411&ndash;435.</note>
<note xml:id="c03-note-0022">Robert Bloomfield and Jeffrey Hales, &ldquo;Predicting the Next Step of a Random Walk: Experimental Evidence of Regime&hyphen;Shifting Beliefs,&rdquo; <i>Journal of Financial Economics</i> 65 (2002): 397&ndash;414.</note>
<note xml:id="c03-note-0023">Cliff Asness, Toby Moskowitz, and Lasse Pedersen, &ldquo;Value and Momentum Everywhere,&rdquo; <i>The Journal of Finance</i> 68 (2013): 929&ndash;985.</note>
<note xml:id="c03-note-0024">To make matters even more interesting, over the 2008 to 2014 period, growth investing, ironically, was the best performer.</note>
<note xml:id="c03-note-0025">Andrea Frazzini, Ronen Israel, and Toby Moskowitz, &ldquo;Trading Costs of Asset Pricing Anomalies,&rdquo; AQR working paper, 2014.</note>
<note xml:id="c03-note-0026">Assuming that some element of both the value and momentum premiums is due to risk, we could see swings in these premiums in the future if risk preferences change.</note></noteGroup>
</body>
</component>
